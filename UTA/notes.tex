\documentclass{scrartcl}

\usepackage{amsmath}
\usepackage{amssymb}

\title{Intersection Type Algebras}
\author{Aaron Frye, Andrew Polonsky}

% Macros for papers on lambda calculus and type theory

% PZC font
\DeclareFontFamily{OT1}{pzc}{}
\DeclareFontShape{OT1}{pzc}{m}{it}{<-> s * [1.10] pzcmi7t}{}
\DeclareMathAlphabet{\mathpzc}{OT1}{pzc}{m}{it}

% Signals
\newcommand{\hired}[1]{\textcolor{red}{#1}}
\newcommand{\hiblue}[1]{\textcolor{blue}{#1}}
\newcommand{\higreen}[1]{\textcolor{green}{#1}}
\newcommand{\hiorange}[1]{\textcolor{orange}{#1}}
\newcommand{\greyout}[1]{\textcolor{gray}{#1}}
\newcommand{\whiteout}[1]{\textcolor{white}{#1}}
\newcommand{\err}[1]{\hired{#1}}

% Misc.
\newcommand{\ul}[1]{\underline{#1}}
\newcommand{\ol}[1]{\overline{#1}}
\newcommand{\setof}[1]{\left\{#1\right\}}
\newcommand{\fv}{\mathsf{FV}}

% Grammar sorts
\newcommand{\lam}{\lambda}
\newcommand{\Lam}{\Lambda}
\newcommand{\vars}{\mathbb{V}}
\newcommand{\atoms}{\mathbb{A}}
\newcommand{\types}{\mathbb{T}}
\newcommand{\nat}{\mathbb{N}}

% Spaced equations
\newcommand{\eq}{\quad=\quad}
\newcommand{\of}{\quad:\quad}
\newcommand{\bnf}{\quad::=\quad}
\newcommand{\rrule}{\quad\longrightarrow\quad}
\newcommand{\then}{\ \Longrightarrow\ }

% Relations and arrows
\newcommand{\sse}{\subseteq}
\newcommand{\sqle}{\sqsubseteq}
\newcommand{\from}{\leftarrow}
\newcommand{\oto}{\leftrightarrow}
\newcommand{\RA}{\Rightarrow}
\newcommand{\thra}{\twoheadrightarrow}

% Standard combinators
\newcommand{\ic}{\mathtt{I}}
\newcommand{\kc}{\mathtt{K}}
\newcommand{\yc}{\mathtt{Y}}
\newcommand{\scomb}{\mathtt{S}}

% Sets of lambda terms
\newcommand{\NF}{\mathsf{NF}}
\newcommand{\WN}{\mathsf{WN}}
\newcommand{\SN}{\mathsf{SN}}

% Finite types
\newcommand{\bzero}{\boldsymbol{0}}
\newcommand{\bone}{\boldsymbol{1}}
\newcommand{\btwo}{\boldsymbol{2}}

% Terms for products and sums
\newcommand{\fst}{\mathsf{fst}}
\newcommand{\snd}{\mathsf{snd}}
\newcommand{\leftc}{\mathsf{left}}
\newcommand{\rightc}{\mathsf{right}}
\newcommand{\case}{\mathsf{case}}

% Elimination forms
\newcommand{\elimc}{\mathsf{elim}}
\newcommand{\iter}{\mathsf{iter}}
\newcommand{\rec}{\mathsf{rec}}

% Terms for booleans
\newcommand{\ttc}{\mathtt{tt}}
\newcommand{\ffc}{\mathtt{ff}}
\newcommand{\ite}{\mathtt{IfThenElse}}

% Systems
\newcommand{\lstar}{\lam\star}
\newcommand{\lameq}{\lam\simeq}
\newcommand{\lamC}{{\lam{C}}}


% \newcommand{\types}{\mathbb{T}}
\newcommand{\mcS}{\mathcal{S}}
\newcommand{\mcF}{\mathcal{F}}
\newcommand{\semof}[1]{{[\![{#1}]\!]}}
\newcommand{\ita}{\mathpzc{Ita}}
\newcommand{\msl}{\mathpzc{Msl}}

\begin{document}

\section*{01.17}
TODO:
\begin{itemize}
  \item Need to start writing the paper!
  \item Need a pretty-printer/helper functions for using the Agda code.
  \item Need to refactor/cleanup the code
  \item Work on the conjecture
  \begin{enumerate}
    \item One direction should be easy
    \item Is the other direction even true?
  \end{enumerate}
  \item Generalizing the code to preorders/intersection types
  \item Verify correctness of the code: prove it computes the most general unifier
  equivalent to the starting list of constraints
  \item Implementing the decision algorithm for the generated recursive type algebra
  \item How to implement/define the ``free'' ita/meet semilattices in type theory
  --- related to: How to define strict order between types.
  \item Implement/verify the decider for the subtyping relation between trees.
  \item To look into strong normalization criterion.
\end{itemize}

\section{Notes from jan. 2025}
Intersection type algebras are the simultaneous free algebra for the
monoid functor (whose free algebras are list algebras) and
icmonoid functor (whose free algebras are free semilattices).

Specifically, given a type $A$, let $\mathcal{P}_f(A)$ denote the final
powerset of $A$.  Define
\begin{align}
  \label{e:msl}
  \msl(A) &= \mathcal{P}_f(A) \\
  \label{e:ita}
  \ita(A) &= \mu X. A \times \msl(X)^*
\end{align}

That is, for any type $A$, $\ita(A)$ is the initial type satisfying
$\ita(A) = \msl(\ita(A))^* \times A$.

The distributivity is implicit in this representation, since the
type $\msl(\ita(A))$ forms an intersection type algebra in the sense of
BCD, which involves two binary operations, $\to$ and $\cap$.
Specifically, given $S_1,S_2 \sse_f (\ita(A)$, we
define
\begin{align*}
  (S_1 \cap S_2) &= S_1 \cup S_2 \\
  (S_1 \to S_2) &= \setof{((s:ts),a) \mid s \in S_1, (ts,a) \in S_2}
\end{align*}
This translation validates the distributivity rule, since
\begin{align*}
  S_1 \to (S_2 \cap S_3)
  &= S_1 \to (S_2 \cup S_3) \\
  &= \setof{(s:ts,a) \mid s \in S_1, (ts,a) \in S_2 \cup S_3}\\
  &= \setof{(s:ts,a) \mid s \in S_1, (ts,a) \in S_2}
  \cup \setof{(s:ts,a) \mid s \in S_1, (ts,a) \in S_3}\\
  &= (S_1 \to S_2) \cup (S_1 \to S_3) \\
  &= (S_1 \to S_2) \cap (S_1 \to S_3)
\end{align*}

With the monadic unit $\eta_A = \lam a{:}A. (\emptyset,a): A \to \msl(\ita(A))$,
and the initiality of the algebra of BCD types $\types_{\to,\cap}(A)$,
we obtain a map $dnf_A : \types_{\to,\cap)(A) \to \msl(\ita(A))}$.

The converse map $bcd_A : \msl(\ita(A)) \to \types_{\to,\cap}(A)$
is given recursively by
\begin{align*}
  bcd_A(\setof{T_1,\dots,T_m}) &= bcd'_A(T_1) \cap (bcd'_A(T_2) \cap
    \cdots (bcd'_A(T_{m-1}) \cap bcd'_A(T_m)) \cdots)\\
  bcd'(A)([S_1,\dots,S_n],a) &= bcd_A(S_1) \to (bcd_A(S_2) \to \cdots (bcd_A(S_n)\to a)
  \cdots)
\end{align*}

It is easy to verify by induction that $bcd_A$ is injective.
It therefore splits the map $dnf_A$, computing the canonical representative
for the BCD-equivalence class of every type in $\types_{\to,\times}(A)$ .

\emph{Need to choose an ordering in representation of subsets!}

\section{2024.09.12}

Questions we still need to figure out:
\begin{itemize}
  \item What to do about reflexive constraints?
  \item What if the list of types equal to the variable is empty?
  Should its solution be $\top$?
  \item If $e = e \to e$ is the "terminal" type algebra
  (any algebra can be mapped to it), admitting solution of any equation,
  does having a constraint like that in a given system mean we can simulate
  $\top = e$?
  \item If we have $e = e \to e$, and we also have a constraint $c = e \to c$,
  this is like saying $c = \top \to c$.  And $d = d \to e$ is "equivalent" to
   $d = d \to \top$?  Does this imply that every term can be typed as $d$ as well?
   (Probably the answer is no.)
   \item Supposing we substitute every variable that we can,
   being left with a system in which every type a variable is constrained to,
   contains that variable.  This might not yet be the proper SR, because we
   might still need to equate individual types in the list for each variable.
\end{itemize}

\section{2024.08.16}

A given set of equations can lead to multiple SR.  These are all equivalent.

The \emph{recursive type entailment problem} is as follows.
Given a set of equations between types $E$ and two types $A$ and $B$,
determine whether $E \vdash A = B$.

The \emph{unification problem} is: Given two types $A$ and $B$,
determine whether there exists a substitution $\sigma$
so that $\vdash A[\sigma] = B[\sigma]$.

The \emph{unification problem up to constraints} is: Given a set of equations $E$,
and two types $A$ and $B$, determine whether there exists a substitution $\sigma$
so that $E \vdash A[\sigma] = B[\sigma]$.


The \emph{general satisfiability problem} is as follows.

Given sets of equations $E_1$, $E_2$, determine whether there exists
a type algebra morphism $h : \types/E_1 \to \types/E_2$.

\end{document}
