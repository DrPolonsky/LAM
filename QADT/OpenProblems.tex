\documentclass{scrartcl}

\usepackage{amsmath}
\usepackage{amssymb}

\newcommand{\cat}{\mathbb{C}}
\newcommand{\op}{{op}}

\newcommand{\bone}{\boldsymbol{1}}
\newcommand{\then}{\quad\mathrel{\Longrightarrow}\quad}

\newcommand{\ints}{\mathbb{Z}}
\newcommand{\nat}{\mathbb{N}}

\begin{document}

\section{Open Questions}
\begin{enumerate}
  \item{Commutativity of interpretations of EnvIso and Iso for ADTS}
  \item{A type theory defining equality without triple equality}
  \begin{itemize}
    \item{Isomorphisms without triple equality}
    \item{}
  \end{itemize}
  \item{A self-consistent type system with types of equalities}
  \item{Sufficient conditions to generate an induced isomorphism from a rig isomorphism}
  \begin{itemize}
  	\item{When does there exist a rig isomorphism that generates an induced isomorphism?}
	\item{Why do some rig isomorphisms generate an induced isomorphism and some do not?}
  \end{itemize}
  \item{}
\end{enumerate}

\section{Closed Questions}

\begin{enumerate}
  \item What is a strong isomorphism between ADTs?
  
  A strong isomorphism, or rig isomorphism, is an isomorphism between two functors acting on an ADT that is a fixpoint of a polynomial functor.  If we say $\mu F$ is the min fixpoint of a polynomial functor $F$, and $G$ and $H$ are polynomial functors, then the isomorphism $G(\mu F) \simeq H(\mu F)$ can be generated for free when:
  \begin{itemize}
  \item The roots of $F(X) - X = 0$ are a sub-multiset of the roots of $G(X) - H(X) = 0$.
  \item $F(X)$ is a second degree or higher polynomial.
  \item The content of $F(X)$ divides the content of $G(X) - H(X)$
  \end{itemize}
  
  \item What does it mean for a functor to be a factor of another functor?
  
  A fixed point of the smaller degree functor is a fixed point of the larger degree functor.
  
  \item A conception of a negative set in these terms.
  \begin{itemize}
    \item Universal $U$, and an ``integer set'' is a function from $U$ to $\mathbb{Z}$.
    \item A contravariant functor $- : \cat \to \cat^\op$ that splits the identity
    ($- \circ -^\op = id$)
    \item A solution to $E(X)=X$, obtained by taking $X = Y + 1$ for some type $Y$.
  \end{itemize}
  \item All polynomials functors with degree at least two will have linear factors.
  Can it solve arbitrary linear equation?
  Are all linear datatypes isomorphic to each other?  Are they generated by $S = -1$?
  \item Retractions between datatypes!
  \item Isomorphisms induced by initial algebra maps.
  \item Let $H(X) = 1 + X + X^3$, then $X$ contains sixth roots of unity as well as
  a square root of unity.  $M(X) = 1 + X + X^2$ contains fourth roots of unity;
  does this explain why we can ``divide $H(X)$ by $M(X)$''?

  Is this the reason why a solution to $M(X)=X$, when squared, yields a solution to $H(X)$?
  Why does this happen more generally?
  
  A solution to $M$, when squared, yields a solution to $H$ because the roots of $M(X) - X = 0$ are a sub-multiset of the roots of $X^2 = H(X^2)$.
  
  \item Can any equation $X=P(X)$, with $P(X)$ be an \emph{arbitrary}
  polynomial --- containing both positive and negative coefficients ---
  be solved by substitution $X = (Y+k)$, for some $k \ge 0$.

  Can various solutions be compared?
  
  We can obtain \textit{a} fixpoint of a Functor by making this substitution but not the least fixed point.

  \item Generalize all of the above to \emph{systems of equations} of the form
  \begin{align*}
    P_1(X_1,\dots,X_k) &= Q_1(X_1,\dots,X_k)\\
    P_2(X_1,\dots,X_k) &= Q_2(X_1,\dots,X_k)\\
     &\quad \vdots \\
    P_k(X_1,\dots,X_k) &= Q_k(X_1,\dots,X_k)
  \end{align*}
\end{enumerate}

\end{document}
