\documentclass{scrartcl}

\begin{document}

\section{Relations, ARS}


\begin{enumerate}
  \item Define logical and structurual operations on relations, proving laws about them.
  \item Define closure operations on realtions, proving laws about them
  \item Define properties such as transtivity etc.
\end{enumerate}
 
\subsection{Well-foundedness}

Four notions:
\begin{itemize}
  \item Based on every element being accessible
  \item Every inductive relation is universally true
  \item Every non-empty set has a minimal element
  \item No infinite sequence is decreasing
\end{itemize}



\section{Open research questions}
\begin{enumerate}
  \item Sufficient conditions for reversing the implications between notions of 
  well-foundedness?
  \item Knaster-Tarski Lemma: Most general form? Alternative formulations?
  \item Syntax for closure operators allowing to prove the needed properties
  uniformly/generically
\end{enumerate}

\section{To Do}
\begin{enumerate}
  \item Clean up the current development
  \item More laws about closure operations: Monotonicity, idempotency
  \item Refactor the ARS formalized results with the new development
  \item Knaster--Tarski in a well-founded setting: what additional
  hypotheses are necessary?
\end{enumerate}

\subsection{Questions about well-foundedness}

\begin{enumerate}
  \item For a given element $x$, is the property of
  being accessible $\lnot\lnot$-closed? I.E.,
  Does $x$ being $\lnot\lnot$-accessible imply that $x$ is accessible?
  (Conjecture: No.)

  \item As a special case, does being weakly (accessible) well-founded imply being $\lnot\lnot$-wellfounded?
  (Conjecture: No.)

  Also: Same question about inductive notion of well-foundedness.

  \item ($\star$) Given a well-founded relation, does every non-empty subset
  have a minimal element?

  \emph{Problem.} Need to decide whether, for a given $x \in U$,
   the set $\{y | Ryx\}$ is empty.

   \item IF every non-empty subset has a minimal element, does this imply
   either of the weak forms of well-foundedness: $\mathtt{isWFacc-}$ or
   $\mathtt{isWFind-}$ ?

   \item Does $\mathtt{isWFmin-}$ imply $\mathtt{isWFacc-}$ or $\mathtt{isWFind-}$?

   \emph{Problem.} Need to go from $\lnot (\forall y. R y d \to \phi y)$
   and $\forall y. R y d \to \lnot \lnot \phi y$ to $\bot$.
 
   In terms of accessibility, it should suffices to assume accessibility is
   $\lnot\lnot$-closed, since being related to y is not relevant for that. (?)

   \item Does sequential well-foundedness (no decreasing sequence) imply
   any of the other notions, e.g., \texttt{WFmin-} ?

   \emph{Note.}  This seems to require the most classical assumptions:
   $\lnot\lnot$-closure of $\phi$, relativized De Morgan law,
   Markov's principle, etc.
\end{enumerate}

\textsc{Remark.}
$\lnot\lnot$-closure of accessibility should solve problems 1 and 4.


\end{document}
