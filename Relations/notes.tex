\documentclass{scrartcl}

\usepackage{xcolor}

\newcommand{\hired}[1]{{\color{red}{#1}}}

\begin{document}

\section*{Notes}

\section{Plan}
\begin{itemize}
  \item Start by discussing motivation
  \item Then contributions
  \item Plan of the paper
  \item At some point, there should be like a summary of what's been done and what are some interesting discoveries
\end{itemize}

In the conclusion, we should recap the main takeaways from each piece.
The key role of finitely branching condition in making many results effective.
That it also collapses many well-foundedness properties.
The ``sufficiency frontier'' for completeness (termination and confluence).

Also discuss open problems, unprovable implications, etc.
Maybe mention the Kleene tree idea, or at least include the reference to the Bezem-Coquand paper.

\section{Motivation}

The standard presentation in Terese makes liberal use of classical logic.
Should stress effectivity of type-theoretic formalization.
\section{Contributions}

\begin{itemize}
  \item Formalization of ``elementary ARS theory'' as presented in
  Chapter 1 of [TeReSe].
  \item An ontology of termination and confluence properties, and a detailed analysis of logical relations between them.
  \item Defining several new ARS concepts that serve to streamline and generalize a number of relationships.
  \item With the above, obtain marginal improvements to classical confluence and termination criteria
  \item Collecting implications between constructive notions of well-foundedness
\end{itemize}

\section{Formalization principles}

\begin{itemize}
  \item No function extensionality
  \item No univalence, uniqueness of identity proofs, cubical, or any other
  assumptions related to equality
  \item Minimize the use of classical logic; every place where it was necessary,
  make decidability hypotheses explicit in every place where it could not be avoided.
  \item The vast majority of results are completely constructive and therefore,
  they compute.
  \item The goal is to have the results of basic abstract rewriting
  formalized in a canonical way, so that they can be used in building
  libraries of formalized programming language theory
  \item Since this application ....
  \item In other words, we wanted to stay as close as possible to the spirit of type theory based on the Curry--Howard isomorphism, so that in every explicitly given situation, our proofs will compute.
\end{itemize}

One big item in this development was studying various incaranations of the concept of well-foundedness in constructive context.
(Strong normalization is well-foundedness of the converse relation.)
The ``standard'' constructive definition does not even allow one to show that SN implies WN, see Example \ref{ex:undec}.  We therefore looked at a number of variations of this notion, their classical counterparts all being logically equivalent.  This gave rise to a rich set of concepts, see Figure \ref{f:wf}.

Along the way, we found marginal improvements to a number of standard theorems.
For example, the termination assumption of the famous Newman's Lemma can be slightly weakened....

There are also large classes of ARSs where some needed classical principles are simply valid.  For example, finitely branching relations encompass most ARSs resulting from first-order term rewrite systems.  For such relations, the implication SN to WN requires nothing else beyond plain decidability of
the relation itself $Rxy \lor \lnot Rxy$.

The main conclusion of our work is that most of basic ARS results can be made completely effective, at least for the most basic instance of first-order TRSs.
(This also includes the usual lambda calculus.)

For more exotic rewrite systems, like $\lambda \bot$ or coinductive rewriting,
these decidability assumptions no longer hold, and the utility of ARS results diminishes proportionally.

Finally, our Agda implementation provides a standard ``entry point'' for using these results in formal language development.

\end{document}


META
This is a report about the work that was completed.
Putting ARS on a constructive foundation -- stress that.
