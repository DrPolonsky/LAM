\documentclass{scrartcl}

\usepackage{hyperref}
% \usepackage{tikz}
\usepackage{amsmath}
\usepackage{amsthm}
\usepackage{pifont}
\usepackage{mathrsfs}
\usepackage{amssymb}
\usepackage{xcolor}
\usepackage{colortbl}
\usepackage{tikz-cd}
\usepackage{caption}
\usepackage{newunicodechar}


\usepackage[utf8]{inputenc}
\usepackage{ucs}
% \DeclareUnicodeCharacter{03A3}{\ensuremath{\Sigma}}



\usetikzlibrary{positioning,shapes.geometric,fit,arrows.meta}


\captionsetup{justification=centering}



\definecolor{darkgreen}{rgb}{0.0, 0.5, 0.0}

\newcommand{\tto}{\twoheadrightarrow}
\newcommand{\sse}{\subseteq}
\newcommand{\bset}{\mathbf{Set}}
\newcommand{\nat}{\mathbb{N}}

% Comments
\newcommand{\sacomment}[1]{\textcolor{green}{#1}}
\newcommand{\apcomment}[1]{\textcolor{blue}{#1}}
\newcommand{\greyout}[1]{\textcolor{gray}{#1}}
\newcommand{\err}[1]{\textcolor{red}{#1}}

% Theorem style
\newtheorem{notation}[theorem]{Notation}
% \newtheorem{thm}{Theorem}
% \newtheorem{dfn}[thm]{Definition}
% \newtheorem{prop}[thm]{Proposition}
% \newtheorem{cor}[thm]{Corollary}
% % \newtheorem{lemma}[thm]{Lemma}
% % \newtheorem{rmk}[thm]{Remark}
% % \newtheorem{expl}[thm]{Example}
% \newtheorem{notn}[thm]{Notation}
% %\theoremstyle{nonumberplain}
% %\theoremsymbol{\Box}
% % \newtheorem{proof}{Proof}

\newcommand{\RP}{\mathrm{RP}}
\newcommand{\gRP}{\mathbf{RP}}
\newcommand{\RPm}{\mathrm{RP{-}}}
\newcommand{\gRPm}{\mathbf{RP{-}}}
\newcommand{\NF}{\mathrm{NF}}
\newcommand{\MF}{\mathrm{MF}}
\newcommand{\UN}{\mathrm{UN}}
\newcommand{\gUN}{\mathbf{UN}}
\newcommand{\UNto}{\mathrm{UN}^{\to}}
\newcommand{\gUNto}{\mathbf{UN}^{\to}}
\newcommand{\SN}{\mathrm{SN}}
\newcommand{\gSN}{\mathbf{SN}}
\newcommand{\decSN}{\mathrm{dec(SN)}}
\newcommand{\SM}{\mathrm{SM}}
\newcommand{\SMseq}{\mathrm{SMseq}}
\newcommand{\gSM}{\mathbf{SM}}
\newcommand{\WN}{\mathrm{WN}}
\newcommand{\gWN}{\mathbf{WN}}
\newcommand{\SMandWN}{\mathrm{SM\land WN}}
\newcommand{\gSMandWN}{\mathbf{SM\land WN}}
\newcommand{\WM}{\mathrm{WM}}
\newcommand{\gWM}{\mathbf{WM}}
\newcommand{\WNFP}{\mathrm{WNFP}}
\newcommand{\NP}{\mathrm{NP}}
\newcommand{\gNP}{\mathbf{NP}}
\newcommand{\NPe}{\mathrm{NP_=}}
\newcommand{\gNPe}{\mathbf{NP_=}}
\newcommand{\WMFP}{\mathrm{WMFP}}
\newcommand{\MP}{\mathrm{MP}}
\newcommand{\gMP}{\mathbf{MP}}
\newcommand{\CR}{\mathrm{CR}}
\newcommand{\CRs}{\mathrm{CR^{\le 1}}}
\newcommand{\gCRs}{\mathbf{CR^{\le 1}}}
\newcommand{\gCR}{\mathbf{CR}}
\newcommand{\WCR}{\mathrm{WCR}}
\newcommand{\gWCR}{\mathbf{WCR}}
\newcommand{\Inc}{\mathrm{Inc}}
\newcommand{\gInc}{\mathbf{Inc}}
\newcommand{\BP}{\mathrm{BP}}
\newcommand{\gBP}{\mathbf{BP}}
\newcommand{\FB}{\mathrm{FB}}
\newcommand{\CP}{\mathrm{CP}}
\newcommand{\gCP}{\mathbf{CP}}



\newcommand{\from}{\leftarrow}



\newcommand{\red}[1]{\textcolor{red}{#1}}
\newcommand{\blue}[1]{\textcolor{blue}{#1}}

% Reduction relation macros
\newcommand{\rstep}{\mathbin{\longrightarrow_R}}
\newcommand{\mstep}{\mathbin{\longrightarrow_R^*}}
\newcommand{\estep}{\mathbin{\longrightarrow_R^=}}
\newcommand{\rrstep}{\mathbin{\longrightarrow_R^r}}
\newcommand{\brstep}{\mathbin{\longleftarrow_R^r}}
\newcommand{\bstep}{\mathbin{\longleftarrow_R}}
\newcommand{\bmstep}{\mathbin{\longleftarrow_R^*}}

% Unicode characters

\newunicodechar{∀}{\ensuremath{\forall}}
\newunicodechar{→}{\ensuremath{\rightarrow}}
\newunicodechar{ℕ}{\ensuremath{\mathbb{N}}}
\newunicodechar{↔}{\ensuremath{\leftrightarrow}}
\newunicodechar{⊆}{\ensuremath{\sse}}
\newunicodechar{∧}{\ensuremath{\land}}
\newunicodechar{₌}{\ensuremath{_=}}
\newunicodechar{𝓟}{\ensuremath{\mathcal{P}}}
\newunicodechar{∈}{\ensuremath{\in}}
\newunicodechar{φ}{\ensuremath{\phi}}
\newunicodechar{Σ}{\ensuremath{\Sigma}}
\newunicodechar{₁}{\ensuremath{{}_1}}


% Misc Macros
\newcommand{\terese}{[TeReSe]}
% \newcommand{\ul}[1]{\underline{#1}}
\newcommand{\ule}[1]{\underline{#1:}}
% \newcommand{\setof}[1]{\{#1\}}
\newcommand{\setof}[1]{\left\{#1\right\}}


\newcommand{\tclos}[1]{{#1}^{\scriptscriptstyle{+}}}


\usepackage{pifont}% http://ctan.org/pkg/pifont
\newcommand{\cmark}{\quad\ding{51}}%
\newcommand{\xmark}{\quad\ding{55}}%

\begin{document}

This document aims to bring together the various notes from report, rewriting qs, and counterexamples.

\section{Todo}
\subsection*{Week starting 11th November}
\begin{enumerate}
  \item[\cmark] In the report, give a classical proof that the $Inc$ Property
  implies that no $R$-increasing sequence is bounded.

  Conclude that every relation satisfying $Inc$ also satisfies $RP$.

  Therefore, the proof of Theorem 1.2.3(iii) using $RP$ instead of $Inc$
  is generalizing the claim of the theorem (as stated in [TeReSe]).
  \item Complete the proof in ARS.agda of Theorem 1.2.3.(iii)
  \item Organize notes.tex into three sections: TODO, Open Qs, What we've proved. Make sure all notes from the other
  tex files are sorted here correctly.
  \item Tidy up and standardize the notation used in ars.agda and other agda files
  \item Carry on working through report.tex.
  \item
\end{enumerate}
\subsection*{General todo}
\begin{enumerate}

  \item  Make the first argument to "being a normal form" implicit?
  \item Define WNFP to be:
  \[\forall a b c \to b \in NF \to a R^* b \to a R^* c \to c R^* b \]
  \item Does WNFP and WN imply CR?  Looks like it!
  \item  Another interesting one? $WN^*$: hereditarily $WN$ ($WN$ for the reduction graph starting from a given element $x$.)

  Could even consider $SN^*$, $CR^*$ etc.
  \item Define: Inc
  \item Define: Ind
  \item Define: $\mathrm{CR}^{\le 1}$
  \item capture all classical properties/asssumptions we might want to use and justify their use.
  \item Principles we are considering
  \begin{itemize}
    \item Decidability of equality on $A$
    \item Decidability of the relation $R$
    \item When relevant, decidability of the given precicate $P$
    \item Not-not-closure of being accessible
    \item Not-not-closure of $R$
    \item Not-not-closure of $P$
    \item Decidability of whether a given element is an $R$-normal form
    \item When relevant, local/element-wise versions of the above
    \item Markov's principle:
    \begin{itemize}
      \item In $wfMin_0 \to wfSeq$.

      Given a sequence $s : \nat \to A$, let $E(a) = \exists n \in \nat. s n \equiv a$.

      Need: $\forall a. \lnot\lnot E(a) \to E(a)$, i.e.,
      \[\lnot\lnot \big(\exists n \in \nat. s n \equiv a\big)
            \to \big(\exists n \in \nat. s n \equiv a\big)\]

    \end{itemize}
  \end{itemize}
  \item Write a guide to the code
  \item More laws about closure operations: Monotonicity, idempotency

\end{enumerate}
\newpage
\section{Open Questions }
\begin{enumerate}
  \item
     \[ WNg \land UN \to CRelem :
     \forall (R : \mathscr{R} A) \to WN R \to \forall x \to is R -UN x \to is R -CR x \]
     \item $WN R \to UN R \to \omega-bdd R \to SN R$ (1.2.3.ii-)
      \item Finish formulating the ``compactness property".  Two candidates:
       \item ``Every cocone for a given infinite sequence loops back to some point in the sequence"
       \item ``If an $R^*$-sequence has a cocone then it's constant after some point on.''
       \item is $\omega$-bounded implied by WN and the following confluence property:
       $a \to^* n \in NF$, and $a \to b$ then $b \to^* n $.
  \item (2024.10.10) Is being $R$-WN inductive?
  \item (2024.10.10) Does $R$ being finitely branching imply
   that, for an inductive $\phi$, if $\lnot \phi(x)$ then
   $\exists y. Rxy \land \lnot \phi(y)$?

  \item If WCR holds, is WN$\downarrow\subseteq$WN ?
  \item Same question for next bullet in 2.2
  \item Same for bullet 3 in 2.2 (but pointwise!)
  \item \item Look at restricting WFmin to $\lnot\lnot$-closed predicates
  \item Status of important hypotheses:
    \begin{itemize}
      \item Deciding whether a given element $x$ reduces to a given element $y$
      (Is $R$ itself decidable?)
      \item Deciding whether a given element $x$ reduces to some element $y$
      (Is being an $R$-normal form decidable?)

      {\textcolor{red}{This is needed to show that SN $\Rightarrow$ WN.}}

      \item If a given element $x$ is \emph{not} a normal form,
      exhibiting some element $y$ that it reduces to.
      (This is related to DeMorgan/Markov/etc. properties that came up in WF file.)

      \item Does UN-lemma \emph{require} the decidability assumption?
    \end{itemize}
  \item META questions: Which of the ARS theorems can be made ``local/pointwise'',
  for example,
  \begin{itemize}
    \item NL+: $WCR(R) \land SN(x) \to CR(x)$?  (Yes?)
    \item NL++: $WCR(x) \land SN(x) \to CR(x)$? (No.)
  \end{itemize}
  \item Can "sequential" properties like RP, RP-, isWFseq, isWFseq-, $\omega$-bounded, etc.,
    be related to the cofinality property.
    \item What happens if properties involving sequences are reformulated for
  $R^r$? (Cannot decide whether the given sequence is finite or infinite.)
  \item Knaster-Tarski Lemma: Most general form? Alternative formulations?
  \item Knaster--Tarski in a well-founded setting: what additional
  hypotheses are necessary?
\end{enumerate}
\subsection*{well-foundedness}
\begin{enumerate}
  \item For a given element $x$, is the property of
  being accessible $\lnot\lnot$-closed? I.E.,
  Does $x$ being $\lnot\lnot$-accessible imply that $x$ is accessible?
  (Conjecture: No.)

  \item As a special case, does being weakly (accessible) well-founded imply being $\lnot\lnot$-wellfounded?
  (Conjecture: No.)

  Also: Same question about inductive notion of well-foundedness.

  \item ($\star$) Given a well-founded relation, does every non-empty subset
  have a minimal element?

  \emph{Problem.} Need to decide whether, for a given $x \in U$,
   the set $\{y | Ryx\}$ is empty.

   \item IF every non-empty subset has a minimal element, does this imply
   either of the weak forms of well-foundedness: $\mathtt{isWFacc-}$ or
   $\mathtt{isWFind-}$ ?

   \item Does $\mathtt{isWFmin-}$ imply $\mathtt{isWFacc-}$ or $\mathtt{isWFind-}$?

   \emph{Problem.} Need to go from $\lnot (\forall y. R y d \to \phi y)$
   and $\forall y. R y d \to \lnot \lnot \phi y$ to $\bot$.

   In terms of accessibility, it should suffices to assume accessibility is
   $\lnot\lnot$-closed, since being related to y is not relevant for that. (?)

   \item Does sequential well-foundedness (no decreasing sequence) imply
   any of the other notions, e.g., \texttt{WFmin-} ?

   \emph{Note.}  This seems to require the most classical assumptions:
   $\lnot\lnot$-closure of $\phi$, relativized De Morgan law,
   Markov's principle, etc.
   \textsc{Remark.}
   $\lnot\lnot$-closure of accessibility should solve problems 1 and 4.



\end{enumerate}

\newpage
\section{What has been proven}
\begin{itemize}
  \item \sacomment{List all the constructive proofs we've put together!}
  \item SN implies bounded
  \item RP- implies RP
  \item The infinite haircomb refutes $WN \land RP \to \omega$-bounded.  But it is not WCR.
  \item We have formalized proposition 1.1.9, 1.1.10, 1.1.11 from [TeReSe].
  \item We have a proof for Theorem 1.2.1 (Newlman's Lemma) (though we have just one proof of multiple possible proofs).
  \item We have a proof for all parts of Theorem 1.2.2
  \item We have a proof for Theorem 1.2.3 i, iv.
  \item Reduction closure properties for SN, NF, UN, recurrent property.

  So, does $WN, RP, WCR$ imply $\omega$-bounded?
  Answer: yes, that is what we have proved classically.
\end{itemize}




\newpage






\end{document}
