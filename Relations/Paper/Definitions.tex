\section{Definitions}
\label{sec:Definitions}
The following definitions are provided to ensure clarity and precision
in our discussion. Each subsection specifies the Agda file(s) which contain formalizations of the given definitions.

\begin{definition}
    An abstract rewrite system (\emph{ARS}) is a structure $\mathcal{A} = (A, R_\alpha)$ where
     $A$ is a set of elements and $R_\alpha$ is a set of binary relations on $A$.
\end{definition}

For the rest of this section, let $A$ be a fixed set, and $R$ be a relation on $A$.


% \greyout{We denote by $\tilde R$ the relational converse of $R$ ($(x,y) \in \tilde R \iff (y,x) \in R$),
% and by $R;R'$ the relational composition of $R$ and $R'$.}

\subsection{Closure operators (\texttt{ClosureOperators.agda})}
\begin{notation}
  The following notions are standard.  For precise definitions,
  see \texttt{ClosureOperators.agda}.
  \begin{enumerate}
    \item $R^r$ denotes the \emph{reflexive closure} of $R$.
    \item $R^s$ denotes the \emph{symmetric closure} of $R$.
    \item $\tclos{R}$ denotes the \emph{transitive closure} of $R$.
    \item $R^* = (R^r)\tclos{}$ denotes the \emph{reflexive-transitive closure} of $R$.
    \item $R^= = (R^s)^*$ denotes the \emph{equivalence relation generated by} $R$.
  \end{enumerate}
\end{notation}




In the context of abstract rewriting, we will often write $a \rstep b$
in place of $(a,b) \in R$.  Similarly, we will write $a \mstep b$ for $(a,b) \in R^*$,
$a \bstep b$ for $(b,a) \in R$.  When the relation $R$ is clear from the context,
we may drop the corresponding subscript.

We next define rewriting-theoretic concepts like confluence and normalization relative to a fixed element of $A$; global versions of these properties are obtained by universal quantification.
\subsection{Normalization (\texttt{ARS-Properties.agda})}
\begin{definition} The following definitions relate to normalization. Let $A$ and $R$ be as above, and let $a \in A$.
  \begin{description}
    \item[$a \in \NF_R$] \emph{$a$ is a normal form} if $(a,b) \notin R$ for any $b \in A$ 
    \item[$a \in \WN_R$] \emph{$a$ is weakly normalizing} if $a \mstep b$ for some $b \in \NF_R$.
    \item[$a \in \SN_R$] \emph{$a$ is strongly normalizing} if $a$ is accessible with respect to the converse relation $\bar{R} = {(y,x) | (x,y) \in R}$. See Section \ref{sec:Well-foundedness}. 
  \end{description}
\end{definition} 

 

\subsection{Confluence (\texttt{ARS-Properties.agda})}
\begin{definition} The following definitions relate to confluence. Let $A$ and $R$ be as above, and let $a \in A$.
    \begin{description}
        \item[$a \in \WCR_R$] \emph{$a$ is weakly Church-Rosser (or weakly confluent)} if $c \bstep a \rstep b \implies c \mstep d \bmstep b$.
        \item[$a \in \CR_R$] \emph{$a$ is Church-Rosser (or confluent)} if $c \bmstep a \mstep b \implies c \mstep d \bmstep b$.
        \item[$a \in \CRs_R$] \emph{$a$ is subcommutative} if $c \bmstep a \mstep b \implies c \rrstep d \brstep b$.
        \item[$a \in \NPe_R$] \emph{$a$ has the equivalence normal form property} if $a \estep b \implies a\mstep b$ for any $b \in \NF$.
        \item[$a \in \NP_R$] \emph{$a$ has the normal form property} if $c \bmstep a \mstep b \implies c \mstep b$ for any $b \in \NF$. 
        \item[$a \in \UN_R$] \emph{$a$ has the unique normal form property} if $a \estep b \implies a \equiv b$  for any $a, b \in \NF$.
        \item[$a \in \UNto_R$] \emph{$a$ has the unique normal form property with respect to reduction} if $c \bmstep a \mstep b  \implies b \equiv c$  for any $b, c \in \NF$.
    \end{description}
\end{definition}
For each of the subsets defined above we denote in bold face the proposition that the subset contains all of $A$.
That is, for $\mathrm{X} \in \{\WCR , \CR , \NP , \UN \}$ we write $\mathbf{X}$ for the statement that $x \in \mathrm{X}_R$ for all $x \in A$.


In \terese $\NF$ is used to denote $\NPe$, whereas we have chosen to use $\NF$ to denote normal form. 

\begin{proposition}\hfill 
    \begin{enumerate}
        \item $\gNP \iff \gNPe$ 
        \item $\gUN \implies \gUNto$
    \end{enumerate}
\end{proposition}
\begin{proof} \hfill
    \begin{enumerate}
        \item \verb|NP₌↔NP : R isNP₌ ↔ R isNP| in \texttt{ARS-Implications.agda}.
        \item \verb|UN→UN→ : R isUN → R isUN→| in \texttt{ARS-Implications.agda}. 
    \end{enumerate} 
\end{proof}

The implication $\UNto \implies \UN$ does not hold, as can be seen in Example \ref{CE:un} in Subsection \ref{subsec:counterexamples}.

\subsection{Sequences and Recurrence (\texttt{ARS-Properties.agda}, \texttt{Seq.agda})}
The following definitions are key to formalizing Theorem 1.2.3 in \terese.

\begin{notation}
    $R \models \mathbf{X}$ is $\forall a.\; a \in \mathrm{X}_R$. When $R$ is clear from the context, we will write $\gNP$ for $R \models \gNP$, etc. 
\end{notation}

\begin{definition} \hfill
    \begin{description}
        \item[$R \models Inc$] \emph{$R$ is increasing} if there exists a ``size function'' $|-| : A \to \nat$ satisfying, for all $a, b \in A$,
        $Rab \to |a| < |b|$  
    \end{description}
\end{definition}

Having a size function that maps to the natural numbers is computationally taxing to satisfy. 
Further, it takes us outside of purely relational theory. The following definitions are for properties which allow us to forgo $\Inc$, for further discussion see Subsection \ref{subsec:def}.
\begin{definition} \label{def:rp} Let $A$ and $R$ be as above, and let $a \in A$.
    \begin{description} 
        \item[$s : \nat \to A$] \emph{A sequence $s \in A$} is a function from the natural numbers to $A$.  
        \item[$s \in R-increasing$] \emph{A sequence $s \in A$ is $R$-increasing} if every term is $R$-related to its preceding term.
        \item[$a$ is an $s$-bound] if $s (i) \mstep a$ for all $i \in \nat$. 
        \item[$a \in \BP_R$] if all $R$-increasing sequence $s$ with $s (0) = a$ there exists an $s$-bound.  
        \item[$R \models \RPm$] \emph{$R$ has the weak recurrence property} if for every increasing sequence $s$ with bound $b$ there exists an $i \in \nat$ such that $b \mstep s (i)$.
        
        \item[$R \models \RP$] \emph{$R$ has the recurrence property} if every bounded, increasing sequence $s$ with an $s$-bound, there exists an $i \in \nat$ such that 
        $s (i) \mstep x \implies x \mstep s (i)$ for all $x\in A$. See Definition \ref{def:mf}. 
    \end{description}
\end{definition}
In \terese $\mathsf{Ind}$ is used to denote $\BP$. 

The following is immediate.
\begin{proposition}\hfill 
    \begin{enumerate}
        \item $\gInc \implies \gRPm$
        \item $\gRPm \implies \gRP$
    \end{enumerate}
\end{proposition}


It follows that every relation that is increasing in the sense of \terese
vacuously satisfies the RP condition.

We investigated $\RP$ and thought that with $\RPm$ we had a candidate for a weaker definition that 
did not rely on the notion of $\MF$. We show that this weaker definition is 
in fact an equivalent definition in the following proof in \texttt{ARS-Implications.agda}: 

\verb|RP-↔RP : R is RP- ↔ R is RP|. 

\subsection{Finitely branching (\texttt{FinitelyBranching.agda})}

\begin{definition} Let $A$ and $R$ be as above, and let $a \in A$.
    \begin{description}
        \item[$a \in \FB$] \emph{$a$ is finitely branching} if there is a finitely many $b \in A$ such that $a \rstep b$.  
    \end{description}
\end{definition}
Our Agda formulation of the above definition makes use of the List type constructor. 
\verb|FB a = Σ[ xs ∈ List A ] (∀ b → R b a → b ∈List xs)|

$\FB$ is a property that plays an important role in our discussion of well-foundedness in Section \ref{sec:Well-foundedness}.