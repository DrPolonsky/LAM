\section{Definitions}
\label{sec:Definitions}

The following definitions are provided to ensure clarity and precision
in our discussion.
 When useful, we also include the name and type of
definition as we have implemented it in Agda, allowing the report to guide the code
and vice versa.



\begin{definition}
    An abstract rewrite system (\emph{ARS}) is a structure $\mathcal{A} = (A, R_\alpha)$ where
     $A$ is a set of elements and $R_\alpha$ is a set of binary relations on $A$.
\end{definition}

For the rest of this section, let $A$ be a fixed set, and $R$ be a relation on $A$.
(In the Agda code supplement, this is represented by a type $A : \bset$,
and a binary type family $R : A \to A \to \bset$.)

\greyout{We denote by $\tilde R$ the relational converse of $R$ ($(x,y) \in \tilde R \iff (y,x) \in R$),
and by $R;R'$ the relational composition of $R$ and $R'$.}


\begin{notation}
  The following notions are standard.  For precise definitions,
  see \texttt{ClosureOperators.agda}.
  \begin{enumerate}
    \item $R^r$ denotes the \emph{reflexive closure} of $R$.
    \item $R^s$ denotes the \emph{symmetric closure} of $R$.
    \item $R^+$ denotes the \emph{transitive closure} of $R$.
    \item $R^* = (R^r)^+$ denotes the \emph{reflexive-transitive closure} of $R$.
    \item $R^= = (R^s)^*$ denotes the \emph{equivalence relation generated by} $R$.
  \end{enumerate}
\end{notation}




In the context of abstract rewriting, we will often write $a \rstep_R b$
in place of $(a,b) \in R$.  Similarly, we will write $a \mstep b$ for $(a,b) \in R^*$,
$a \bstep b$ for $(b,a) \in R$.  When the relation $R$ is clear from the context,
we may drop the corresponding subscript.

We first define rewriting-theoretic concepts like confluence and normalization relative to a fixed element of $A$; global versions of these properties are obtained by universal quantification.

\begin{definition}  Let $A$ and $R$ be as above, and let $a \in A$.
  \begin{description}
    \item[$a \in \NF_R$] \emph{$a$ is a normal form} if $(a,b) \notin R$ for any $b \in A$.
    \item[$a \in \WN_R$] \emph{$a$ is weakly normalizing} if $a \mstep b$ for some $b \in \NF_R$.
    \item[$a \in SN_R$]  \emph{$a$ is $R$-accessible}, see Definition \ref{d:WFacc}.
  \end{description}

    An element $a \in A$ is \emph{R-weakly confluent} ($WCR_R$) if all single step relations from $a$ can always converge
    via multi-step relations to some common reduct.

    A relation $R$ is \emph{weakly confluent} (AKA weakly Church-Rosser (WCR)) if every $a \in A$ is R-weakly confluent.
\end{definition}

\begin{definition}
    An element $a \in A$ is \emph{R-confluent} ($CR_R$) if all multi-step relations from $a$ can always converge
    via multi-step relations to some common reduct.
    A relation $R$ is \emph{confluent} (AKA Church-Rosser (CR)) if every $a \in A$ is R-confluent.
\end{definition}

\begin{definition}
  \label{d:WFacc}
  An element is \emph{accessible} if ...
\end{definition}

\begin{definition}
    An element $a \in A$ is \emph{$R$-weakly normalizing} ($WN_{R}$)if $a \mstep b$ for some normal form $b \in A$.

    A relation $R$ is weakly normalizing (WN) if every $a \in A$ is $R$- weakly normalizing.
\end{definition}

\begin{definition}
    An element $a \in A$ is \emph{$R$-strongly normalizing} ($SN_R$) if every sequence of relations starting from $a$ is finite.

    A relation $R$ is strongly normalizing (SN) if every
    $a \in A$ is $R$-strongly normalizing.
\end{definition}

The above definition of SN is taken from [TeReSe].
When working in Agda, we make use of the alternative definition presented in [TeReSe] ($\leftarrow$ is accessibly well-founded).
The \ref{sec:Well-foundedness} section clarifies this definition.

\begin{definition}
    An element $a \in A$ has the \emph{$R$-weak normal form property} ($WNFP_{R}$) if $a \mstep b$ and
    $a \mstep c \implies c \mstep b$, for all $c \in A$ and all normal form $b \in A$.

    A relation $R$ has the \emph{weak normal form property} (WNFP) if every $a \in A$ has the $R$-weak normal form property.
\end{definition}

\begin{definition}
    The \emph{equivalence relation} of $R$ is the smallest relation $\estep$ that contains $R$ and is reflexive, transitive, and symmetric.
\end{definition}

\begin{definition}
An element $a \in A$ has the \emph{$R$-normal form property} ($NFP_{R}$) if
$a \estep b$  $\implies$
$a \mstep b$, for any normal form $b \in A$.

A relation $R$ has the \emph{normal form property} (NFP) if every $a \in A$ has the $R$-normal form property.
\end{definition}

In ARS-Implications.agda we show the equivalence of WNFP and NFP when applied globally in the function $\mathtt{NP \leftrightarrow WNFP}$.

In the next definition we deonte two elements $a, b \in A$ being equivalent with $a \equiv b$.
\begin{definition}
An element $a \in A$ has the \emph{{R}-unique normal form property} ($UN_{R}$) if
$a \estep b$  $\implies$
$a \equiv b$, for any normal form $b \in A$.

A relation $R$ has the \emph{unique normal form property} ($UN$) if every $a \in A$ has the $R$-unique normal form property.
\end{definition}


\begin{definition}
An element $a \in A$ has the \emph{${R}$-unique normal form property with respect to reduction} ($UN^ \to _{R}$) if
$b \bmstep  \cdot  \mstep c$  $\implies$ $b \equiv c$, for all normal forms $b,c \in A$.

A relation $R$ has the \emph{unique normal form property with respect to reduction} ($UN^\to$) if every $a \in A$ has the
$R$-unique normal form property with respect to reduction.
\end{definition}
In ARS-Implications.agda we show that $UN$ implies $UN\to$ but the inverse implication doesn't hold, as seen
in counterexample (\sacomment{TODO: put counterexample here when section is complete}).

\begin{definition}
    A \emph{sequence (in $A$)} is a function from the natural numbers to $A$:
    \begin{align*}
      &s : \nat \to A \\
      &s = (s_0,s_1,s_2,\dots)
    \end{align*}
\end{definition}

\begin{definition}
    A sequence is \emph{$R$-increasing} if every term is $R$-related to its preceding term.
\end{definition}

\begin{definition}
    A sequence is \emph{bounded} if there exists an element $a \in A$ to which all elements of the sequence reduce.
\end{definition}

\begin{definition}
    An element $a \in A$ has the \emph{minimal form property} ($MF$) if $a \mstep b \implies b \mstep a$ for all $b \in A$.
\end{definition}

Note that being a normal form is a trivial case of the minimal form property (as we prove in ARS-implications.agda
function $\mathtt{NF\subseteq MF : \forall {x} \to NF x \to MF x}$).

\begin{definition}
    An element $a \in A$ is \emph{$R$-weakly minimal} ($WM_{R}$)if $a \mstep b$ for some minimal form $b \in A$.

    A relation $R$ is weakly minimal (WM) if every $a \in A$ is $R$-weakly minimal.
\end{definition}

\begin{definition}
    An element $a \in A$ is \emph{$R$-strongly minimal} ($SM_R$) if either $a$ itself is a minimal form or every element one $R$ step
    from $a$ is also strongly minimal.

    A relation $R$ is strongly minimal (SM) if every
    $a \in A$ is $R$-strongly minimal.
\end{definition}

\begin{definition}
   An element $a \in A$ has the \emph{${R}$-weak minimal form property} ($WMFP_R$) if
   $a \mstep b$ and $a \mstep c$  $\implies$
   $c \mstep b$, for all $c \in A$ and all $b \in A$ with the minimal form property.

   A relation $R$ has the \emph{weak minimal form property} (WMFP) if every $a \in A$ has the $R$-weak minimal form property.
\end{definition}

Definitions to add:
\begin{enumerate}
    \item Finite branching
    \item Any other terms from well foundedness
    \item The other terms we have for strongly
\end{enumerate}
