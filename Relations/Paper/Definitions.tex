\section{Definitions}
\label{sec:Definitions}

The following definitions are provided to ensure clarity and precision 
in our discussion. When useful, we also include the name and type of
definition as we have implemented it in Agda, allowing the report to guide the code
and vice versa.

\begin{definition}
    An abstract rewrite system (\emph{ARS}) is a structure $\mathcal{A} = (A, R_\alpha)$ where
     $A$ is a set of elements and $R_\alpha$ is a set of binary relations on $A$.
\end{definition}

We denote a relation $R$ (from the set $R_\alpha$) from an element $a \in A$ to an element $b \in A$ as $a\rstep b$.
A multi-step relation is denoted $a \mstep b$.

\begin{definition}
    An element $a \in A$ is \emph{R-weakly confluent} ($WCR_R$) if all single step relations from $a$ can always converge
    via multi-step relations to some common reduct.
    
    A relation $R$ is \emph{weakly confluent} (AKA weakly Church-Rosser (WCR)) if every $a \in A$ is R-weakly confluent.
\end{definition}

\begin{definition}
    An element $a \in A$ is \emph{R-confluent} ($CR_R$) if all multi-step relations from $a$ can always converge
    via multi-step relations to some common reduct.
    A relation $R$ is \emph{confluent} (AKA Church-Rosser (CR)) if every $a \in A$ is R-confluent.
\end{definition}

\begin{definition}
    \emph An element $a \in A$ is a {normal form} if there exists no relation to any other element in $A$.
\end{definition}
  
\begin{definition}
    An element $a \in A$ is \emph{$R$-weakly normalizing} ($WN_{R}$)if $a \mstep b$ for some normal form $b \in A$.
  
    A relation $R$ is weakly normalizing (WN) if every $a \in A$ is $R$- weakly normalizing.
\end{definition}

\begin{definition}
    An element $a \in A$ is \emph{$R$-strongly normalizing} ($SN_R$) if every sequence of relations starting from $a$ is finite.
  
    A relation $R$ is strongly normalizing (SN) if every
    $a \in A$ is $R$-strongly normalizing.
\end{definition}

The above definition of SN is taken from [TeReSe]. 
When working in Agda, we make use of the alternative definition presented in [TeReSe] ($\leftarrow$ is accessibly well-founded). 
The \ref{sec:Well-foundedness} section clarifies this definition. 

\begin{definition}
    An element $a \in A$ has the \emph{$R$-weak normal form property} ($WNFP_{R}$) if $a \mstep b$ and 
    $a \mstep c \implies c \mstep b$, for all $c \in A$ and all normal form $b \in A$.
    
    A relation $R$ has the \emph{weak normal form property} (WNFP) if every $a \in A$ has the $R$-weak normal form property.    
\end{definition}

\begin{definition}
    The \emph{equivalence relation} of $R$ is the smallest relation $\estep$ that contains $R$ and is reflexive, transitive, and symmetric.
\end{definition}

\begin{definition}
An element $a \in A$ has the \emph{$R$-normal form property} ($NFP_{R}$) if
$a \estep b$  $\implies$
$a \mstep b$, for any normal form $b \in A$.

A relation $R$ has the \emph{normal form property} (NFP) if every $a \in A$ has the $R$-normal form property.
\end{definition}

In ARS-Implications.agda we show the equivalence of WNFP and NFP when applied globally in the function $\mathtt{NP \leftrightarrow WNFP}$.

In the next definition we deonte two elements $a, b \in A$ being equivalent with $a \equiv b$.
\begin{definition}
An element $a \in A$ has the \emph{{R}-unique normal form property} ($UN_{R}$) if
$a \estep b$  $\implies$
$a \equiv b$, for any normal form $b \in A$.

A relation $R$ has the \emph{unique normal form property} ($UN$) if every $a \in A$ has the $R$-unique normal form property.
\end{definition}


\begin{definition}
An element $a \in A$ has the \emph{${R}$-unique normal form property with respect to reduction} ($UN^ \to _{R}$) if
$b \bmstep  \cdot  \mstep c$  $\implies$ $b \equiv c$, for all normal forms $b,c \in A$.

A relation $R$ has the \emph{unique normal form property with respect to reduction} ($UN^\to$) if every $a \in A$ has the
$R$-unique normal form property with respect to reduction.
\end{definition}
In ARS-Implications.agda we show that $UN$ implies $UN\to$ but the inverse implication doesn't hold, as seen 
in counterexample (\sacomment{TODO: put counterexample here when section is complete}).

\begin{definition}
    A \emph{sequence (in $A$)} is a function from the natural numbers to $A$:
    \begin{align*}
      &s : \nat \to A \\
      &s = (s_0,s_1,s_2,\dots)
    \end{align*}
\end{definition}
  
\begin{definition}
    A sequence is \emph{$R$-increasing} if every term is $R$-related to its preceding term.
\end{definition}
  
\begin{definition}
    A sequence is \emph{bounded} if there exists an element $a \in A$ to which all elements of the sequence reduce.
\end{definition}

\begin{definition}
    An element $a \in A$ has the \emph{minimal form property} ($MF$) if $a \mstep b \implies b \mstep a$ for all $b \in A$.
\end{definition}

Note that being a normal form is a trivial case of the minimal form property (as we prove in ARS-implications.agda 
function $\mathtt{NF\subseteq MF : \forall {x} \to NF x \to MF x}$).

\begin{definition}
    An element $a \in A$ is \emph{$R$-weakly recurrent} ($WR_{R}$)if $a \mstep b$ for some recurrent element $b \in A$.
  
    A relation $R$ is weakly recurrent (WR) if every $a \in A$ is $R$-weakly recurrent.
\end{definition}

\begin{definition}
    An element $a \in A$ is \emph{$R$-strongly recurrent} ($SR_R$) if either $a$ itself is recurrent or every element one $R$ step
    from $a$ is also strongly recurrent.
  
    A relation $R$ is strongly recurrent (SR) if every
    $a \in A$ is $R$-strongly recurrent.
\end{definition}

\begin{definition}
   An element $a \in A$ has the \emph{${R}$-weak minimal form property} ($WMFP_R$) if
   $a \mstep b$ and $a \mstep c$  $\implies$
   $c \mstep b$, for all $c \in A$ and all $b \in A$ with the minimal form property.

   A relation $R$ has the \emph{weak minimal form property} (WMFP) if every $a \in A$ has the $R$-weak minimal form property.
\end{definition}








