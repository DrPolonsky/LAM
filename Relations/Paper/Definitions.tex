\section{Definitions}
\label{sec:Definitions}
The following definitions are provided to ensure clarity and precision
in our discussion. Each subsection specifies the Agda file(s) which contain formalizations of the given definitions.

\begin{definition}
    An abstract rewrite system (\emph{ARS}) is a structure $\mathcal{A} = (A, R_\alpha)$ where
     $A$ is a set of elements and $R_\alpha$ is a set of binary relations on $A$.
\end{definition}

For the rest of this section, let $A$ be a fixed set, and $R$ be a relation on $A$.
(In the Agda code supplement, this is represented by a type $A : \bset$,
and a binary type family $R : A \to A \to \bset$.)

\greyout{We denote by $\tilde R$ the relational converse of $R$ ($(x,y) \in \tilde R \iff (y,x) \in R$),
and by $R;R'$ the relational composition of $R$ and $R'$.}

\subsection{Closure operators (\texttt{ClosureOperators.agda})}
\begin{notation}
  The following notions are standard.  For precise definitions,
  see \texttt{ClosureOperators.agda}.
  \begin{enumerate}
    \item $R^r$ denotes the \emph{reflexive closure} of $R$.
    \item $R^s$ denotes the \emph{symmetric closure} of $R$.
    \item $R^+$ denotes the \emph{transitive closure} of $R$.
    \item $R^* = (R^r)^+$ denotes the \emph{reflexive-transitive closure} of $R$.
    \item $R^= = (R^s)^*$ denotes the \emph{equivalence relation generated by} $R$.
  \end{enumerate}
\end{notation}




In the context of abstract rewriting, we will often write $a \rstep b$
in place of $(a,b) \in R$.  Similarly, we will write $a \mstep b$ for $(a,b) \in R^*$,
$a \bstep b$ for $(b,a) \in R$.  When the relation $R$ is clear from the context,
we may drop the corresponding subscript.

We next define rewriting-theoretic concepts like confluence and normalization relative to a fixed element of $A$; global versions of these properties are obtained by universal quantification.
\subsection{Normalization (\texttt{ARS-Properties.agda})}
\begin{definition} The following definitions relate to normalization. Let $A$ and $R$ be as above, and let $a \in A$.
  \begin{description}
    \item[$a \in \NF_R$] \emph{$a$ is a normal form} if $(a,b) \notin R$ for any $b \in A$ 
    \item[$a \in \WN_R$] \emph{$a$ is weakly normalizing} if $a \mstep b$ for some $b \in \NF_R$.
    \item[$a \in \SN_R$]  \emph{$a$ is strongly normalizing} if every reduction sequence starting from $a$ is well-founded\footnote{See Section \ref{sec:Well-foundedness} for definitions of being well-founded. Note that we are 
    making use of the alternative definition of $\SN$ provided in \terese.}. \sacomment{Check this defn}
  \end{description}
\end{definition} 

 
\begin{definition} The folowing definitions relate to being minimal. Let $A$ and $R$ be as above, and let $a \in A$.
    \begin{description}
        \item[$a \in \MF_R$] \emph{$a$ is a minimal form} if $a \mstep b \implies b \mstep a$ for any $b \in A$.
        \item[$a \in \WM_R$] \emph{$a$ is weakly minimalizing} if $a \mstep b$ for some $b \in \MF_R$.
        \item[$a \in \SM_R$]  \emph{$a$ is strongly minimalizing} if either $a \in MF_R$, or every element one $R$ step from $a$ is also strongly minimalizing.
    \end{description}
\end{definition}
We found that there was a strong interrelationship between the normalizing properties and the minimal properties. This relationship is 
captured in Figure \ref{fig:norm-hierarchy}.

\begin{figure}[h]   
    \centering
        \begin{tikzpicture}[auto,
            arrowstyle/.style={->, line width=1pt, >={Latex[length=3mm, width=2mm]}, shorten >=2pt}]
            % A box style
            \tikzstyle{boxnode} = [draw, rectangle, text centered,minimum width=1.2cm, inner sep=3pt]   
            % Define the nodes
            \node (SN) [boxnode] at (4,0) {$\SN$};
            \node (WN) [boxnode, above right=0.5cm and 0.5cm of SN] {$\WN$};
            \node (SR) [boxnode, below right=0.5cm and 0.5cm of SN] {$\SM$};
            \node (WR) [boxnode, below right=0.5cm and 0.5cm of WN] {$\WM$};
            
            % Draw the arrows to form a diamond
            \draw [arrowstyle] (SN) -- (WN);
            \draw [arrowstyle] (SN) -- (SR);
            \draw [arrowstyle] (WN) -- (WR);
            \draw [arrowstyle] (SR) -- (WR);
        \end{tikzpicture}
        \caption{A hierarchy of normalizing and minimal properties.}
        \label{fig:norm-hierarchy}
\end{figure}

The implications will be explored in detail in Section \ref{sec:Implications}. 

A trivial implication not captured in the above figure is covered in Proposition \ref{prop:nftomf}. 
\begin{proposition}\label{prop:nftomf}
    $a\in NF_R \implies a \in MF_R$ 
\end{proposition}    
\begin{proof}
    See, \verb|NF ⊆ MF : ∀x → NFx → MFx| in \texttt{ARS-Implications.agda}.
\end{proof}
\subsection{Confluence (\texttt{ARS-Properties.agda})}
\begin{definition} The following definitions relate to confluence. Let $A$ and $R$ be as above, and let $a \in A$.
    \begin{description}
        \item[$a \in \WCR_R$] \emph{$a$ is weakly Church-Rosser (or weakly confluent)} if $c \bstep a \rstep b \implies c \mstep d \bmstep b$.
        \item[$a \in \CR_R$] \emph{$a$ is Church-Rosser (or confluent)} if $c \bmstep a \mstep b \implies c \mstep d \bmstep b$.
        \item[$a \in \NPe_R$] \emph{$a$ has the equivalence normal form property} if $a \estep b \implies a\mstep b$ for any $b \in \NF$.
        \item[$a \in \NP_R$] \emph{$a$ has the normal form property} if $c \bmstep a \mstep b \implies c \mstep b$ for any $b \in \NF$\footnote{In \terese $\NF$ is used to denote $\NPe$, whereas we have chosen to use $\NF$ to denote normal form.}. 
        \item[$a \in \MP_R$] \emph{$a$ has the minimal form property} if $c \bmstep a \mstep b \implies c \mstep b$ for any $b \in \MF$.
        \item[$a \in \UN_R$] \emph{$a$ has the unique normal form property} if $a \estep b \implies a \equiv b$  for any $a, b \in NF$.
        \item[$a \in \UNto_R$] \emph{$a$ has the unique normal form property with respect to reduction} if $c \bmstep a \mstep b  \implies b \equiv c$  for any $b, c \in NF$.
    \end{description}
\end{definition}

Again, we found that there was a strong interrelationship between the properties, captured in Figure \ref{fig:conf-hierarchy}, which are explored in Section \ref{sec:Implications}.

\begin{figure}[h] 
    \centering

        \begin{tikzpicture}[auto,
          arrowstyle/.style={->, line width=1pt, >={Latex[length=3mm, width=2mm]}, shorten >=2pt}]
          % A box style
          \tikzstyle{boxnode} = [draw, rectangle, text centered,minimum width=1.2cm, inner sep=3pt]
      
          % Place the nodes vertically (top to bottom)
          \node (Confluent) [boxnode] at (-4,0) {$\CR$};
          \node (RP)        [boxnode, right=.5cm of Confluent] {$\MP$};
          \node (WN)        [boxnode, right=.5cm of RP] {$\NP$};
          \node (UN)        [boxnode, right=.5cm of WN] {$\UNto$};
      
          % Draw arrows downwards
          \draw[arrowstyle] (Confluent) -- (RP);
          \draw[arrowstyle] (RP) -- (WN);
          \draw[arrowstyle] (WN) -- (UN);
       
          \end{tikzpicture}
          \caption{A hierarchy of confluent properties.}
          \label{fig:conf-hierarchy}
    
\end{figure}

\begin{proposition}
    $\NPe \leftrightarrow \NP$ 
\end{proposition}
\begin{proof}
    See, \verb|NP₌↔NP : R isNP₌ ↔ R isNP| in \texttt{ARS-Implications.agda}.
\end{proof}
\begin{proposition}
    $\UN \implies \UNto $
\end{proposition}
\begin{proof}
    See, \verb|UN→UN→ : R isUN → R isUN→| in \texttt{ARS-Implications.agda}
\end{proof}
The implication $\UNto \implies \UN$ does not hold, as can be seen in CE-\ref{CE:un} in Subsection \ref{subsec:counterexamples}.
\subsection{Sequences and Recurrence (\texttt{ARS-Properties.agda}, \texttt{Seq.agda})}
\begin{definition} The following definitions relate to the minimal definitions and are key to formalizing Theorem 1.2.3 in \terese. Let $A$ and $R$ be as above, and let $a \in A$.
    \begin{description}
        \item[$s : \nat \to A$] \emph{A sequence $s \in A$} is a function from the natural numbers to $A$.  
        \item[$s \in R-increasing$] \emph{A sequence $s \in A$ is $R$-increasing} if every term is $R$-related to its preceding term.
        \item[$a \in \BP$] \emph{$a$ is a bound} if there exists a sequence $s$ such that $s (i) \mstep a$ for any $i \in \nat$. 
        \item[$R\, is\, \RP$] \emph{$R$ has the recurrence property} if every bounded, increasing sequence has an element $a \in MF$.
        \item[$R\, is\, \RPm$] \emph{$R$ has the weak recurrence property} if for every increasing sequence $s$ with bound $b$ there exists an element $i$ of the sequence such that $a \mstep s (i)$.
        \item[$R\, is\, Inc$] \emph{$R$ is increasing} if there exists a ``size function'' $|-| : A \to \nat$ satisfying, for all $a, b \in A$,
        \[ Rab \to |a| < |b| \]       
    \end{description}
\end{definition}
Proposition \ref{prop:SMRP} connects $\SM$ with $\RP$ and $\BP$.
\begin{proposition}\label{prop:SMRP}
    $SM \leftrightarrow RP \land BP$
\end{proposition}
\begin{proof}
    See,\\ \verb|RP∧BP→SMseq : R isRP → R isBP → ∀ {x : A} → SMseq R x| \\ \verb|RisSMseq→RisRP : (∀ {x : A} → SMseq R x) → R isRP|
    \\ \verb|RisSMseq→RisBP : (∀ {x : A} → SMseq R x) → R isBP| in \texttt{ARS-Implications.agda}.
\end{proof}

\begin{proposition}
    If $R$ is increasing, then no infinite $R$-increasing sequence is bounded.
\end{proposition}
\begin{proof}
    Let $(a_i) = a_0 \to a_1 \to a_2 \to \cdots$ be an $R$-increasing sequence
    and suppose that $(a_i)$ is bounded.  Let $b$ be the bound, so that $a_i \to b$ for all $i$.

    If $m = |b|$ then we have
    \[ 0 \le |a_0| < |a_1| < |a_2| < \cdots < |a_m| < |b| = m \]
    % \[ m > |a_m| > |a_{m-1}| > \cdots > |a_0| \ge 0 \]
    which is a contradiction, since there are only $m-1$ integers strictly less than $m$.
\end{proof}
It follows that every relation that is increasing in the sense of \terese
vacuously satisfies the RP condition.

\begin{corollary}
    If $R$ is increasing, then $R$ has the recurrence property.
\end{corollary}
We discuss in Section \ref{sec:Formalization} the motivation for finding an alternative for the increasing property found in \terese. 

\sacomment{Anything else to define such as finitely branching?}
