\section{Hierarchy}
\label{sec:Hierarchy}

\newenvironment{counterexample}[1][]{%
    \refstepcounter{CEcounter} % Increment the counter
    \noindent \scriptsize\textbf{{\theCEcounter } }  #1\par
}

% From the perspective of programming language theory, the two properties of abstract rewriting 
% systems that are most interesting are $\SN$ and $\CR$. The following hierarchies 
% show chains of implications starting from these properties.
This section examines the hierarchy of ARS properties. 
In programming language theory, $\SN$ and $\CR$ constitute the essential properties that 
jointly establish completeness. 
Consequently, our analysis centers on two primary categories of properties: those associated 
with termination and those associated with confluence.
% Possibly redundant introductory paragraph

As discussed in Subsection \ref{subsec:def} we put forward $\RP$ as an alternative for $\Inc$. It is through examining $\RP$ that 
we came to the 'minimal` family of properties below. These properties help to augment our ARS hierarchy. 

\begin{definition}\label{def:mf} \hfill
    \begin{description}
        \item[\ule{$a \in \MF_R$}] \emph{$a$ is a minimal form} if $a \mstep b \implies b \mstep a$ for any $b \in A$.
        \item[\ule{$a \in \WM_R$}] \emph{$a$ is weakly minimalizing} if $a \mstep b$ for some $b \in \MF_R$.
        \item[\ule{$a \in \SMseq_R$}] \emph{$a$ is sequentially strongly minimalizing} if for every $R$-increasing sequence $s$ starting at $a$, there exists 
        an $i \in \nat$ such that $s (i) \in \MF$. 
        \item[\ule{$a \in \SM_R$}]  \emph{$a$ is strongly minimalizing} if either $a \in MF_R$, or every element one $R$ step from $a$ is strongly minimalizing. (This 
        definition is to be understood inductively).
        \item[\ule{$a \in \MP_R$}] \emph{$a$ has the minimal form property} if $c \bmstep a \mstep b \implies c \mstep b$ for any $b \in \MF$.

    \end{description} 
\end{definition}

Trivially, the following holds. 
\begin{proposition}\label{prop:nftomf}
    $\NF \implies \MF$ 
\end{proposition}    
\begin{proof}
    See, \verb|NF ⊆ MF : ∀x → NFx → MFx| in \texttt{ARS-Implications.agda}.
\end{proof}

$\SMseq$ is only classically equivalent to $\SM$. The 
following proposition relates these new definitions with our previous $\RP$ definitions.
\begin{proposition}\label{prop:SMRP}
    $SMseq \leftrightarrow RP \land BP$
\end{proposition}
\begin{proof}
    See,\\ \verb|RP∧BP→SMseq : R isRP → R isBP → ∀ {x : A} → SMseq R x| \\ \verb|RisSMseq→RisRP : (∀ {x : A} → SMseq R x) → R isRP|
    \\ \verb|RisSMseq→RisBP : (∀ {x : A} → SMseq R x) → R isBP| in \texttt{ARS-Implications.agda}.
\end{proof}


Proposition \ref{prop:nftomf} suggests that we consider $\MF$ as a termination property in its own right. This leads us to 
explore the following broader taxonomy of concepts relating to termination and confluence.

\begin{center}
    \begin{tikzpicture}[auto,
      arrowstyle/.style={->, line width=1pt, >={Latex[length=3mm, width=2mm]}, shorten >=2pt}]
      % A box style
      \tikzstyle{boxnode} = [draw, rectangle, text centered, minimum width=1.2cm, inner sep=3pt]
      
      % Title nodes
      \node[text width=4cm, align=center, font=\bfseries] at (-1.5,1) {Confluence Hierarchy};
      \node[text width=4cm, align=center, font=\bfseries] at (5,1) {Termination Hierarchy};
  
      % Place the nodes vertically (top to bottom)
      \node (Confluent) [boxnode] at (-4,0) {$\CR$};
      \node (RP)        [boxnode, right=.5cm of Confluent] {$\MP$};
      \node (WN)        [boxnode, right=.5cm of RP] {$\NP$};
      \node (UN)        [boxnode, right=.5cm of WN] {$\UNto$};
  
      % Draw arrows downwards
      \draw[arrowstyle] (Confluent) -- (RP);
      \draw[arrowstyle] (RP) -- (WN);
      \draw[arrowstyle] (WN) -- (UN);
   
        
      % Define the nodes
      \node (NF) [boxnode] at (3,0) {$\NF$};
      \node (MF) [boxnode] at (3,-1) {$\MF$};
      \node (SN) [boxnode] at (5,0) {$\SN$};
      \node (SR) [boxnode] at (5,-1) {$\SM$};
      \node (WN) [boxnode] at (7,0) {$\WN$};
      \node (WR) [boxnode] at (7,-1) {$\WM$};
      
      % Draw the arrows to form a diamond
      \draw [arrowstyle] (NF) -- (SN);
      \draw [arrowstyle] (NF) -- (MF);
      \draw [arrowstyle] (MF) -- (SR);
      \draw [arrowstyle] (SN) -- (WN);
      \draw [arrowstyle] (SN) -- (SR);
      \draw [arrowstyle] (WN) -- (WR);
      \draw [arrowstyle] (SR) -- (WR);
      
      % Optional: Add a separation line
      \draw[dashed] (2.0,-1.5) -- (2.0,1.5);
    \end{tikzpicture}
\end{center}

The implications in this diagram are formalized in \texttt{ARS-Implications.agda}. 
The hierarchy for confluencce requires no classical assumptions, however this is not the case for the hierarchy of terminating properties.
The implication from $\SN$ to $\WN$ relies on the classical decidability of the property 
of being $R$-minimal, as demonstrated in our formalization:

\verb|SNdec→WN : (~R R) isMinDec → SN ⊆ WN| 
 
The same property is similarly required in order to show that $\SM$ implies $\WM$.

Having established the hierarchies of ARS properties, we now examine which property combinations are sufficient to reverse these implications and achieve completeness. The following tables illustrate, for each property combination, whether confluence (left box) and strong normalization (right box) are obtained. Where properties cannot be derived, we provide counterexamples demonstrating the necessity of additional conditions.

The first table explores where the properties hold locally, the second explores where the properties hold globally. Counterexamples in the 
first table assume $\WCR$ and in the second table assume $\gWCR$ (as a reminder, bold text is used to indicate a property holds globally). 

\clearpage
\renewcommand*{\thefootnote}{\fnsymbol{footnote}}



\begin{table}[h!]
    \centering
    \renewcommand\arraystretch{1.2}
    \begin{tabular}{!{\vrule width 1.5pt}
        >{\columncolor{gray!30}}l
        !{\vrule width 1.5pt}c|c
        !{\vrule width 1.5pt}c|c
        !{\vrule width 1.5pt}c|c
        !{\vrule width 1.5pt}c|c
        !{\vrule width 1.5pt}c|c!{\vrule width 1.5pt}}
        \Xhline{1.5pt}
        \rowcolor{gray!30} 
        & \multicolumn{2}{c!{\vrule width 1.5pt}}{$\WM$}
        & \multicolumn{2}{c!{\vrule width 1.5pt}}{$\WN$}
        & \multicolumn{2}{c!{\vrule width 1.5pt}}{$\SM$}
        & \multicolumn{2}{c!{\vrule width 1.5pt}}{$\SMandWN$}
        & \multicolumn{2}{c!{\vrule width 1.5pt}}{$\SN$} \\
        \Xhline{1.5pt}
        $\UNto$ & CE-\ref{CE:4} & CE-\ref{CE:11} & CE-\ref{CE:4} & CE-\ref{CE:6} & $\CR$\footnotemark[1] & CE-\ref{CE:8} & $\CR$\footnotemark[1] & CE-\ref{CE:4} & $\CR$\footnotemark[1] & $\SN$ \\
        \hline
        $\NP$ & CE-\ref{CE:9} & CE-\ref{CE:8} & $\CR$ & CE-\ref{CE:6} & $\CR$\footnotemark[1] & CE-\ref{CE:8} & $\CR$ & $\SN$ & $\CR$\footnotemark[2] & $\SN$ \\
        \hline
        $\MP$ & $\CR$ & CE-\ref{CE:8} & $\CR$ & CE-\ref{CE:6} & $\CR$\footnotemark[2] & CE-\ref{CE:8} & $\CR$ & $\SN$ & $\CR$\footnotemark[2] & $\SN$ \\
        \hline
        $\CR$ & $\CR$ & CE-\ref{CE:8} & $\CR$ & CE-\ref{CE:6} & $\CR$ & CE-\ref{CE:8} & $\CR$ & $\SN$ & $\CR$ & $\SN$ \\
        \hline
        \Xhline{1.5pt}
    \end{tabular}
    \caption{Local implications}
\end{table}

\vspace{-1cm}
% \\[-0.5cm]
\begin{table}[h!]
    \centering
    \renewcommand\arraystretch{1.2}
    \begin{tabular}{!{\vrule width 1.5pt}
        >{\columncolor{gray!30}}l
        !{\vrule width 1.5pt}c|c
        !{\vrule width 1.5pt}c|c
        !{\vrule width 1.5pt}c|c
        !{\vrule width 1.5pt}c|c
        !{\vrule width 1.5pt}c|c!{\vrule width 1.5pt}}
        \Xhline{1.5pt}
        \rowcolor{gray!30} 
        & \multicolumn{2}{c!{\vrule width 1.5pt}}{$\gWM$}
        & \multicolumn{2}{c!{\vrule width 1.5pt}}{$\gWN$}
        & \multicolumn{2}{c!{\vrule width 1.5pt}}{$\gSM$}
        & \multicolumn{2}{c!{\vrule width 1.5pt}}{$\gSMandWN$}
        & \multicolumn{2}{c!{\vrule width 1.5pt}}{$\gSN$} \\
        \Xhline{1.5pt}
        $\gUNto$ & CE-\ref{CE:2} & CE-\ref{CE:8} & $\gCR$ & CE-\ref{CE:6} & $\gCR$\footnotemark[1] & CE-\ref{CE:8} & $\gCR$ & $\gSN$ & $\gCR$\footnotemark[2] & $\gSN$ \\
        \hline
        $\gNP$ & CE-\ref{CE:3} & CE-\ref{CE:8} & $\gCR$ & CE-\ref{CE:6} & $\gCR$\footnotemark[1] & CE-\ref{CE:8} & $\gCR$ & $\gSN$ & $\gCR$\footnotemark[2] & $\gSN$ \\
        \hline
        $\gMP$ & $\gCR$ & CE-\ref{CE:8} & $\gCR$ & CE-\ref{CE:6} & $\gCR$\footnotemark[2] & CE-\ref{CE:8} & $\gCR$ & $\gSN$ & $\gCR$\footnotemark[2] & $\gSN$ \\
        \hline
        $\gCR$ & $\gCR$ & CE-\ref{CE:8} & $\gCR$ & CE-\ref{CE:7} & $\gCR$ & CE-\ref{CE:8} & $\gCR$ & $\gSN$ & $\gCR$ & $\gSN$ \\
        \Xhline{1.5pt}
    \end{tabular}
    \caption{Global implications}
\end{table}
\footnotetext[1]{This implication also requires $\gWCR$.}
\footnotetext[2]{This implication also requires either $\gWCR$ or the classical property required to go from 
$\SN \to \WN$ or $\SM \to \WM$.}
\renewcommand*{\thefootnote}{\arabic{footnote}}

The table should be interpreted as follows: for example, if a relation possesses the property $\NP$, then, according to the table, obtaining the property $\CR$ requires, at a minimum, the additional property $\WN$. Counterexample~\ref{CE:9} demonstrates that the combination of $\WM$ and $\NP$ is insufficient to guarantee $\CR$.

\subsection{Counterexamples}\label{subsec:counterexamples}
% Setting up the ce counter 
\newcounter{CEcounter}
\renewcommand{\theCEcounter}{\arabic{CEcounter}}

    \begin{center}
        \begin{tabular}{ccc}  % Three columns per row
            % First Counterexample
            \begin{minipage}{0.3\textwidth}
                \centering
                \begin{counterexample}\label{CE:1}
                    \begin{tikzcd}[row sep=small, column sep=small]
                        a & b \arrow[l] \arrow[r, bend left] & c \arrow[l, bend left] \arrow[r] & d
                    \end{tikzcd} \\
                \end{counterexample}
            \end{minipage}
            &
            % Second Counterexample
            \begin{minipage}{0.3\textwidth}
                \centering
                \begin{counterexample}\label{CE:2}
                    \begin{tikzcd}[row sep=small, column sep=small]
                        a & b \arrow[l] \arrow[r, bend left] & c \arrow[l, bend left] \arrow[r] & d \arrow[loop right]
                    \end{tikzcd} \\
                \end{counterexample}
            \end{minipage}
            &
            % Third Counterexample
            \begin{minipage}{0.3\textwidth}
                \centering
                \begin{counterexample}\label{CE:3}
                    \begin{tikzcd}[row sep=small, column sep=small]
                        a \arrow[loop left] & b \arrow[l] \arrow[r, bend left] & c \arrow[l, bend left] \arrow[r] & d \arrow[loop right]
                    \end{tikzcd} \\
                \end{counterexample}
            \end{minipage}
            \\  % New row
            \\
            
            % Fourth Counterexample
            \begin{minipage}{0.3\textwidth}
                \centering
                \begin{counterexample}\label{CE:4}
                    \begin{tikzcd}[row sep=small, column sep=small]
                        & a \arrow[d] \arrow[dl] \\ e & b \arrow[l] \arrow[r] & c \arrow[r, bend left] & d \arrow[l, bend left]
                    \end{tikzcd} \\
                \end{counterexample}
            \end{minipage}
            &
            % Fifth Counterexample
            % \begin{minipage}{0.3\textwidth}
            %     \centering
            %     \begin{counterexample}\label{CE:5}
            %         \begin{tikzcd}[row sep=small, column sep=small]
            %             d & a \arrow[l, two heads] \arrow[r, two heads] & b \arrow[r, bend left] & c \arrow[l, bend left]
            %         \end{tikzcd} \\
            %     \end{counterexample}
            % \end{minipage}
            
            % Sixth Counterexample - This one shows why UN -> doesn't imply UN
            \begin{minipage}{0.3\textwidth}
                \centering
                \begin{counterexample}\label{CE:5}
                    \begin{tikzcd}[row sep=small, column sep=small]
                        n & c \arrow[l] \arrow[r] & a \arrow[r, bend left] & b \arrow[l, bend left]
                        & d \arrow[l] \arrow[r] & m
                    \end{tikzcd} \\
                \end{counterexample}
            \end{minipage}
        % \\ % New row
        % \\
        &
        \begin{minipage}{0.3\textwidth}
            \centering
            \begin{counterexample}\label{CE:6}
            \begin{tikzcd}[row sep=small, column sep=small]
                f_0 \arrow[r] \arrow[dr] & f_1 \arrow[r] \arrow[d] & f_2 \arrow[dl] \arrow[r] & \dots \arrow[dll] \\
                & n 
            \end{tikzcd} \\
            \end{counterexample}
        \end{minipage}
        \\ % new row
        \\
        
        \begin{minipage}{0.3\textwidth}
            \centering
            \begin{counterexample}\label{CE:7}
            \begin{tikzcd}[row sep=small, column sep=small]
                f_0 \arrow[r] \arrow[d] & f_1 \arrow[r] \arrow[d] & f_2 \arrow[r] \arrow[d] & \dots \arrow[d] \\
                n_0  & n_1  & n_2 & \dots 
            \end{tikzcd} \\
            \end{counterexample}
        \end{minipage}
        
        &
        \begin{minipage}{0.3\textwidth}
            \centering
            \begin{counterexample}\label{CE:8}
            \begin{tikzcd}[row sep=small, column sep=small]
                a \arrow[r, bend left] & b \arrow[l, bend left]
            \end{tikzcd} \\
        \end{counterexample}
    \end{minipage}
    % \\ % New row
    % \\
    &
    \begin{minipage}{0.3\textwidth}
        \centering
        \begin{counterexample}\label{CE:9}
            \begin{tikzcd}[row sep=small, column sep=small]
                d  \arrow[d] \arrow[rr, bend left] & a \arrow[r] \arrow[l]  &  
                b \arrow[r, bend left] & c \arrow[l, bend left] \\  
                f_0 \arrow[r] & f_1 \arrow[r] & f_2 \arrow[r] & \dots  
            \end{tikzcd} \\
        \end{counterexample}
    \end{minipage}
    % New row
    \\
    \\
    \begin{minipage}{0.3\textwidth}
        \centering
        \begin{counterexample}\label{CE:10}
            \begin{tikzcd}[row sep=small, column sep=small]
                c \arrow[r, bend left] & b \arrow[l, bend left]  & a \arrow[l] \arrow[r] 
                & d \arrow[r, bend left] & e \arrow[l, bend left]
            \end{tikzcd} \\
        \end{counterexample}
    \end{minipage}
    &
    &
    \begin{minipage}{0.3\textwidth}
        \centering
        \begin{counterexample}\label{CE:11}
            \begin{tikzcd}[row sep=small, column sep=small]
                c & a \arrow[r, bend left] \arrow[l] & b \arrow[l, bend left]
            \end{tikzcd} \\
        \end{counterexample}
    \end{minipage}
    
    
\end{tabular}
\end{center}

\begin{remark} The following are the key takeaways from the implication tables. \hfill
    \begin{enumerate}
        \item The property $\gWCR$ can sometimes be a substitute for the classical 
        property required for $\SN \implies \WN$ (and similarly $\SM \implies \WM$).   
        \item $\gWN \land \gUNto \implies \gCR$ does not require any classical assumptions
         or $\gWCR$ for the implication to $\gCR$, unlike $\gSN \land \gUNto$. 
        \item $\gWN \land \gUNto \implies \gCR$ only holds when the properties are global. The 
        counterexample provided illustrates why $\WCR \land \WN \land \UNto \nRightarrow \CR$.
        \item We have a generalization of Newman's Lemma as $\SM \land \gWCR \implies \CR$. This is explored in Subsection \ref{subsec:newnewman}.
        \item Both $\SM$ and $\WN$ are required for $\SN$, this connection is explored in Subsection \ref{subsec:SMWNSN}.
    \end{enumerate}
\end{remark}

\subsection{Generalized Newman's Lemma}\label{subsec:newnewman}
One interesting outcome from investigating the normalization taxonomy was a generalization of Newman's Lemma. 

\begin{proposition}
    $\gWCR \land \SM \implies \CR$
\end{proposition}
\begin{proof}
    See, \verb|LocalNewmansLemmaRecurrent : R isWCR → SM ⊆ CR|
\end{proof}

This proof follows the same steps as the second proof of Newman's Lemma found in \terese, but 
uses the weaker assumption $\SM$ in place of $\SN$. 

% The tables show a number of changes when moving from properties holding locally to globally. 
% Globally $\WN \land \UNto \implies \CR$ holds whereas it does not locally. Similarly, $\SM \land \WN \implies \SN$ holds 
% globally but not locally. 
\subsection{Relation between $\SM$, $\WN$, and $\SN$} \label{subsec:SMWNSN}
One key takeaway from the tables is that we obtain completeness when we have the properties $\SM$, $\WN$, and $\NP$. We also 
obtain completeness when we have $\gWN$ and $\gSM$. 
To show that $\gSM \land \gWN \implies \gSN$ we built on our proof that $\SM \land \WN \land \NP \implies \SN$. If $\gWN$ holds 
then our proof no longer requires the property $\NP$. This progression of proofs can be seen in the functions:

\verb|WN∧NP∧SM→SN : ∀ {x} → WN x → NP x → SM x → SN x| 

\verb|isWN∧SM→SN : R isWN → ∀ {x} → SM x → SN x|

\verb|isWN∧isSM→isSN : R isWN → R isSM → R isSN| 


