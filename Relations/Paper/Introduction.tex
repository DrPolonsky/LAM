\section{Introduction}
\label{sec:Introduction}

% \section{Plan}
% \begin{itemize}
%   \item Start by discussing motivation
%   \item Then contributions
%   \item Plan of the paper
%   \item At some point, there should be like a summary of what's been done and what are some interesting discoveries
% \end{itemize}
%


We present a formalization of the basic Abstract Rewriting Systems (ARS) theory in the Agda proof assistant.
This is part of a larger effort to develop a library of formalized programming language theory (PL)
at Appalachian State University.
Since Term Rewriting Systems (TRS) play a foundational role in programming languages,
while ARS encompass the basic facts about rewriting relations in general,
establishing these facts is part of the necessary infrastructure required to
pursue the broader project.  The present contribution can therefore be seen as
``bootstrapping'' Appstate's new formalized PL repository.

This broader goal also drives the main design choices in our approach.
Since rewriting theory is concerned with a fine-grained analysis of computation,
the most natural vehicle for formalizing it is type theory based on the
Curry--Howard isomorphism, which Agda implements.  Proofs in this language
are automatically effective, and implicitly carry the code implementing the
transformations from the hypotheses to the conclusion.  For example,
a proof that a given TRS is terminating automatically renders a function that
computes a normal form of any term in its domain.

Staying within this paradigm requires the proofs to be constructive.
Moreover, it requires us to avoid any axioms or postulates, such as function
extensionality, uniqueness of identity proofs, univalence, etc.
While most of ARS results are indeed constructive, standard presentations,
 including \hired{[Terese]}, make liberal use of classical logic.
Our paper can thus be read as a thoroughly constructive
development of elementary ARS theory.

Before we proceed further, let us consider two formalization scenarios where
our contribution can be useful.

\subsection*{Example 1. Formalization of typed lambda calculus.}

Suppose one is formalizing the standard metatheory of
a typed lambda calculus.  To show confluence, one possible approach is to
first establish the Church--Rosser theorem for the untyped lambda calculus,
then invoke the Subject Reduction Theorem.
However, suppose one has already independently verified Strong Normalization.
Invoking Newman's lemma, one can then conclude confluence from the weak
diamond property, which is generally a much simpler fact to prove.

\subsection*{Example 2. Quotients of types.}
Suppose one wishes to construct the initial model of an algebraic theory.
The classical definition involves the quotient of a free algebra by
the equivalence relation induced by the theory.
The quotient is problematic operation in type theory.
While there are various solutions, such as Higher Inductive Types,
or Quotient Inductive--Inductive Types, these require one to extend the
basic type theory with new constructs.

However, if the algebraic theory in question admits a convergent presentation,
by a confluent and terminating relation $R : A \to A \to Set$, one can define the type of
$R$-normal forms as
\[  (\Sigma n : A) (\Pi x : A) \lnot P n x \]
In many applications, this type can substitute the true quotient satisfactorily.
(In univalent or cubical type theories, one can moreover show this type to be
a set, provided the domain $A$ is.)

As these examples illustrate, a fully effective implementation of basic ARS results
can be very helpful in many settings arising in programming language theory.

\subsection{Contributions}
Our contributions
\begin{itemize}
  \item Formalization of ``elementary ARS theory'' as presented in
  Chapter 1 of [TeReSe].
  \item An ontology of termination and confluence properties, and a detailed analysis of logical relations between them.
  \item Defining several new ARS concepts that serve to streamline and generalize a number of relationships.
  \item With the above, obtain marginal improvements to classical confluence and termination criteria
  \item Collecting implications between constructive notions of well-foundedness
\end{itemize}

\subsection{Formalization principles}
\hired{One of our design goals is to stay within pure type theory as much as possible,
therefore we adopt the following principles.}


\begin{itemize}
  \item No function extensionality
  \item No univalence, uniqueness of identity proofs, cubical, or any other
  assumptions related to equality
  \item Minimize the use of classical logic; every place where it was necessary,
  make decidability hypotheses explicit in every place where it could not be avoided.
  \item The vast majority of results are completely constructive and therefore,
  they compute.
  \item The goal is to have the results of basic abstract rewriting
  formalized in a canonical way, so that they can be used in building
  libraries of formalized programming language theory
  \item Since this application ....
  \item In other words, we wanted to stay as close as possible to the spirit of type theory based on the Curry--Howard isomorphism, so that in every explicitly given situation, our proofs will compute.
\end{itemize}

\subsection{Misc.}
One big item in this development was studying various incaranations of the concept of well-foundedness in constructive context.
(Strong normalization is well-foundedness of the converse relation.)
The ``standard'' constructive definition does not even allow one to show that SN implies WN, see Example \ref{ex:undec}.  We therefore looked at a number of variations of this notion, their classical counterparts all being logically equivalent.  This gave rise to a rich set of concepts, see Figure \ref{f:wf}.

Along the way, we found marginal improvements to a number of standard theorems.
For example, the termination assumption of the famous Newman's Lemma can be slightly weakened....

There are also large classes of ARSs where some needed classical principles are simply valid.  For example, finitely branching relations encompass most ARSs resulting from first-order term rewrite systems.  For such relations, the implication SN to WN requires nothing else beyond plain decidability of
the relation itself $Rxy \lor \lnot Rxy$.

The main conclusion of our work is that most of basic ARS results can be made completely effective, at least for the most basic instance of first-order TRSs.
(This also includes the usual lambda calculus.)

For more exotic rewrite systems, like $\lambda \bot$ or coinductive rewriting,
these decidability assumptions no longer hold, and the utility of ARS results diminishes proportionally.

Finally, our Agda implementation provides a standard ``entry point'' for using these results in formal language development.

{ \color{gray}
In the conclusion, we should recap the main takeaways from each piece.
\begin{itemize}
  \item ARS results are effective for finitely branching and decidable relations
  --- covers first-order TRSs and the lambda calculus.
  \item Identification of ``sufficiency frontier'' for completeness (termination and confluence).
  It also collapses many well-foundedness properties.
  \item Also discuss open problems, unprovable implications, etc.
  Maybe mention the Kleene tree idea, or at least include the reference to the Bezem--Coquand paper.
  \item \emph{Formalizing classical ARS theory in the Curry--Howard framework
  lead to improvements of classical ARS results.}
\end{itemize}
}


\subsection{OLD>DELETE}

Term Rewriting Systems (TRS) play a foundational role in the theory of
programming languages.

- Operational semantics
- Reduction strategies

At Appalachian State University, we are building a library of formalized theory,
lambda calculus, and type theory. This is a first contribution in this effort.

ARS is Establishing those properties of relations relevant for rewriting theory.

Quotients.

As a motivating example, consider the problem of representing quotients in type theory.

Solutions relying on homotopy type theory require computational interpretation
of univalence, which is a high price to pay for a basic structural operation
undergirding many concepts in mathematics.
Satisfactory solution is provided by cubical type theory, but this solution
is very complex for something so basic.

However

Term rewriting systems allow to present an algebraic theory with a
set of generators and relations from which decidability of equality
follows immediately.
Moreover, the fact that one can always find canonical representatives
allows to efficiently work with them.


- Formalization of "Abstract Rewriting" content from TeReSe,
including background on logic, relations, etc.
- Identifying constructive formulations of classical concepts and proofs
- Studying implications between notions related to confluence and termination
- Isolating classical principles from proofs and eliminating them where possible.
-
