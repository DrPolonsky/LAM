\section{Formalization of \terese}
\label{sec:Formalization}
This sections is a guide to our formalization of \terese $\,$ and explains the deviations we take from the text.

\subsection{Formalization of Definitions (\texttt{Base.agda})}\label{subsec:def}

As outlined in the previous section, \texttt{Properties.agda}, \texttt{Seq.agda}, and \texttt{ClosureOperators.agda} define most standard termination and confluence properties. The file \texttt{Base.agda} introduces additional definitions which are required for the formalization of the propositions found in Chapter 1 of \terese.


The following is a summary of the differences in definitions we have used as mentioned in Section \ref{sec:Definitions}.
\begin{itemize}
    \item $\SN$ : We define this via the notion of accessibility rather than the non-existence of infinite reduction sequences. Both definitions articulate normalization as well-foundedness of the converse relation. However, different notions of well-foundedness are used in the two definitions. The differences are highlighted in Section~\ref{sec:Well-foundedness}.
     See Section \ref{sec:Well-foundedness} for further discussion.
    \item $\NPe$ : We use the equivalent $\NP$ definition.
    \item $\Inc$ : We use the more general property $\RP$.
\end{itemize}

As mentioned, the $\Inc$ property can impose a computationally demanding requirement. However, $\Inc$ is used only in Theorem 1.2.3
and the only crucial aspect of $\Inc$ needed for the proof is that no $R$-increasing
sequence can have an upper bound.
We establish that the weaker property $\RP$ (or its equivalent $\RPm$) is sufficient for replacing $\Inc$ in the theorem (see Subsection \ref{subsec:theorems}).
This led us to examine the condition on $s (i)$ appearing inside $\RP$ more closely. This condition provides a generalization of the notion of a normal form,
which we explore in detail in Section \ref{sec:Hierarchy}.

\subsection{\texttt{Propositions.agda}}
There are three sets of propositions given in \terese: \texttt{1.1.9}, \texttt{1.1.10}, \texttt{1.1.11}.
The formalization of the proofs are in \texttt{Propositions.agda}. These formalizations are straightforward
and do not merit further discussion.

\subsection{\texttt{Theorems.agda}} \label{subsec:theorems}

\begin{theorem}(Newman's Lemma)
  $\gSN \land \gWCR \implies \gCR$
\end{theorem}

\begin{proof}
  There are three proofs of Newman's Lemma given in \terese.
  The first and third of these proofs make use of classical reasoning. The second proof is amenable to a
  constructive formalization as shown in the following function in \texttt{ARS-NewmansLemma.agda}:

  \verb|NewmansLemma : R isSN → R isWCR → R isCR|.
\end{proof}

  We show in Subsection \ref{subsec:newnewman} a generalization of Newman's Lemma using a condition distilled from $\gRP$.
  The formalizations of the remaining theorems are all found in \texttt{Theorems.agda}.

\begin{theorem}[Thm 1.2.2 in \terese]
\begin{enumerate}[label=\roman*.]
  \item[]
  \item $\gCR \implies \gNPe \implies \gUN$
  \item $\gWN \land \gUN \implies \gCR$
  \item $\gCRs \implies \gCR$
\end{enumerate}
\end{theorem}

\begin{proof}
    See \texttt{module Theorem-1-2-2} in \texttt{Theorems.agda}, \\
    \verb|CR→NP : R isCR → R isNP₌|\\
    \verb|NP→UN : R isNP₌ → R isUN|\\
    \verb|ii     : R isWN × R isUN → R isCR|\\
    \verb|ii+    : R isWN × R |\texttt{isUN}$^{\to}$ \verb|→ R isCR|\\
    \verb|iii    : subcommutative R → R isCR|
\end{proof}
\begin{remark}
    In $\mathtt{UN^{→}∧WN→CR}$ a more general proof of ($ii$) is given using $\UNto$.
\end{remark}


\begin{theorem}[Thm. 1.2.3 in \terese]
  \begin{enumerate}[label=\roman*.]
    \item[]
    \item $\gWN \land \gUN \implies \gBP$
    \item $\gBP \land \gInc \implies \gSN$
    \item $\gWCR \land \gWN \land \gInc \implies \gSN$
    \item $\gCR \iff \gCP$ $($the forward implication only holds if the ARS is countable.$)$
  \end{enumerate}
\end{theorem}

\begin{proof}
    See \texttt{module Theorem-1-2-3} in \texttt{Theorems.agda},\\
    \verb|i     : R isWN → R isUN → R isBP|\\
    $\mathtt{i^\to}$\hspace{5mm} \verb|: R isWN → R |\texttt{isUN}$^{\to}$ \verb|→ R isBP|\\
    \verb|iiSeq : R isWN → R isUN → R isRP → isWFseq- (~R R)|\\
    \verb|iii   : R isWN → R isWCR → R isRP- → dec SN → R isSN|\\
    \verb|iv    : R isCP → R isCR|
\end{proof}
\begin{remark} \hfill
  \begin{description}
    \item[($i$)] In $\mathtt{i^\to}$ we again provide a slight improvement by using $\gUNto$ rather than $\gUN$.
    \item[($ii$)] This proof is inherently classical. For the closest constructive analogue to this proof we have written $\mathtt{iiSeq}$, where we replace $\gBP$ with the stronger properties $\gWN$ and $\gUN$ and we replace $\gInc$ with the weaker property $\gRP$.
    With the original property $\gBP$ and $gRP$ we can imply the related property $gSM$ ($\gSM$ is defined in Section~\ref{sec:Hierarchy} and its relation to $gSN$ is expanded upon). The definition and further discussion of $\isWFseqm$ are provided in Section~\ref{sec:Well-foundedness}.
    \item[($iii$)] In $\mathtt{iii}$, we replace $\gInc$ with the more general property $\gRPm$. We also assume that $\gSN$ is decidable, as this allows the construction of a witness for whether an element possesses the property $\SN$, which is essential for our proof.
  \end{description}
\end{remark}
