\section{Conclusion and Further Work}
\label{sec:Conclusion}

Recap the main takeaways from each piece.
\begin{itemize}
  \item ARS results are effective for finitely branching and decidable relations
  --- covers first-order TRSs and the lambda calculus.
  \item Identification of ``sufficiency frontier'' for completeness (termination and confluence).
  It also collapses many well-foundedness properties.
  \item Also discuss open problems, unprovable implications, etc.
  Maybe mention the Kleene tree idea, or at least include the reference to the Bezem--Coquand paper.
  \item Throughout this effort, we found that the classical theorems often assume
  more than is necessary, and these distinctions becomes especially pronounced
  in the context of constructive logic.
  \item \emph{Formalizing classical ARS theory in the Curry--Howard framework
  lead to improvements of classical ARS results.} In the process, new properties were identified,
   offering a clearer structural understanding of how different properties interact
    to establish completeness. These insights contribute to a more refined and
     expressive formulation of ARS theory.
  \item We believe the fact that $\isWFseq$ does not imply $\isWFacc$
  can be established by the methods of Bezem et al.

\end{itemize}
