\section{Conclusion and Further Work}
\label{sec:Conclusion}

In conclusion, we have formalized the basic results of ARS theory as
presented in \cite{Terese}.  Our investigations show that much of this
theory can be made effective for most languages one typically encounters
in programming language theory.
Throughout this effort, we found that the classical theorems often assume
  more than is necessary, and these distinctions become especially pronounced
  in the context of constructive logic.
  Seeking to optimize hypotheses in every proof, we obtaining marginal improvements of the standard
termination and confluence criteria.
We also identified several new ARS properties
and established their relationship with the standard notions.
This provided us with a
clearer structural understanding of how different properties interact
with one another, contributing to a more refined and
expressive formulation of ARS theory.

Our work lays the ground for further extensions.  It makes standard ARS results
ready for immediate use in formalizations of term rewriting systems and typed
lambda calculi.

% In the future, we would like to formalize more advanced ARS results.
% We would


% Recap the main takeaways from each piece.
% \begin{itemize}
%   \item ARS results are effective for finitely branching and decidable relations
%   --- covers first-order TRSs and the lambda calculus.
%   \item Identification of ``sufficiency frontier'' for completeness (termination and confluence).
%   It also collapses many well-foundedness properties.
%   \item Also discuss open problems, unprovable implications, etc.
%   Maybe mention the Kleene tree idea, or at least include the reference to the Bezem--Coquand paper.
%   \item Throughout this effort, we found that the classical theorems often assume
%   more than is necessary, and these distinctions becomes especially pronounced
%   in the context of constructive logic.
%   \item \emph{Formalizing classical ARS theory in the Curry--Howard framework
%   lead to improvements of classical ARS results.} In the process, new properties were identified,
%    offering a clearer structural understanding of how different properties interact
%     to establish completeness. These insights contribute to a more refined and
%      expressive formulation of ARS theory.
%   \item We believe the fact that $\isWFseq$ does not imply $\isWFacc$
%   can be established by the methods of Bezem et al.
%
% \end{itemize}
