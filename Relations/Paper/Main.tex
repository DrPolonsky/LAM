\documentclass[letterpaper,USenglish,cleveref, autoref, thm-restate]{lipics-v2021}
 
\usepackage{hyperref}
% \usepackage{tikz}
\usepackage{amsmath}
\usepackage{amsthm}
\usepackage{pifont}
\usepackage{mathrsfs}
\usepackage{amssymb}
\usepackage{xcolor}
\usepackage{colortbl}
\usepackage{tikz-cd}
\usepackage{caption}
\usepackage{newunicodechar}


\usepackage[utf8]{inputenc}
\usepackage{ucs}
% \DeclareUnicodeCharacter{03A3}{\ensuremath{\Sigma}}



\usetikzlibrary{positioning,shapes.geometric,fit,arrows.meta}


\captionsetup{justification=centering}



\definecolor{darkgreen}{rgb}{0.0, 0.5, 0.0}

\newcommand{\tto}{\twoheadrightarrow}
\newcommand{\sse}{\subseteq}
\newcommand{\bset}{\mathbf{Set}}
\newcommand{\nat}{\mathbb{N}}

% Comments
\newcommand{\sacomment}[1]{\textcolor{green}{#1}}
\newcommand{\apcomment}[1]{\textcolor{blue}{#1}}
\newcommand{\greyout}[1]{\textcolor{gray}{#1}}
\newcommand{\err}[1]{\textcolor{red}{#1}}

% Theorem style
\newtheorem{notation}[theorem]{Notation}
% \newtheorem{thm}{Theorem}
% \newtheorem{dfn}[thm]{Definition}
% \newtheorem{prop}[thm]{Proposition}
% \newtheorem{cor}[thm]{Corollary}
% % \newtheorem{lemma}[thm]{Lemma}
% % \newtheorem{rmk}[thm]{Remark}
% % \newtheorem{expl}[thm]{Example}
% \newtheorem{notn}[thm]{Notation}
% %\theoremstyle{nonumberplain}
% %\theoremsymbol{\Box}
% % \newtheorem{proof}{Proof}

\newcommand{\RP}{\mathrm{RP}}
\newcommand{\gRP}{\mathbf{RP}}
\newcommand{\RPm}{\mathrm{RP{-}}}
\newcommand{\gRPm}{\mathbf{RP{-}}}
\newcommand{\NF}{\mathrm{NF}}
\newcommand{\MF}{\mathrm{MF}}
\newcommand{\UN}{\mathrm{UN}}
\newcommand{\gUN}{\mathbf{UN}}
\newcommand{\UNto}{\mathrm{UN}^{\to}}
\newcommand{\gUNto}{\mathbf{UN}^{\to}}
\newcommand{\SN}{\mathrm{SN}}
\newcommand{\gSN}{\mathbf{SN}}
\newcommand{\decSN}{\mathrm{dec(SN)}}
\newcommand{\SM}{\mathrm{SM}}
\newcommand{\SMseq}{\mathrm{SMseq}}
\newcommand{\gSM}{\mathbf{SM}}
\newcommand{\WN}{\mathrm{WN}}
\newcommand{\gWN}{\mathbf{WN}}
\newcommand{\SMandWN}{\mathrm{SM\land WN}}
\newcommand{\gSMandWN}{\mathbf{SM\land WN}}
\newcommand{\WM}{\mathrm{WM}}
\newcommand{\gWM}{\mathbf{WM}}
\newcommand{\WNFP}{\mathrm{WNFP}}
\newcommand{\NP}{\mathrm{NP}}
\newcommand{\gNP}{\mathbf{NP}}
\newcommand{\NPe}{\mathrm{NP_=}}
\newcommand{\gNPe}{\mathbf{NP_=}}
\newcommand{\WMFP}{\mathrm{WMFP}}
\newcommand{\MP}{\mathrm{MP}}
\newcommand{\gMP}{\mathbf{MP}}
\newcommand{\CR}{\mathrm{CR}}
\newcommand{\CRs}{\mathrm{CR^{\le 1}}}
\newcommand{\gCRs}{\mathbf{CR^{\le 1}}}
\newcommand{\gCR}{\mathbf{CR}}
\newcommand{\WCR}{\mathrm{WCR}}
\newcommand{\gWCR}{\mathbf{WCR}}
\newcommand{\Inc}{\mathrm{Inc}}
\newcommand{\gInc}{\mathbf{Inc}}
\newcommand{\BP}{\mathrm{BP}}
\newcommand{\gBP}{\mathbf{BP}}
\newcommand{\FB}{\mathrm{FB}}
\newcommand{\CP}{\mathrm{CP}}
\newcommand{\gCP}{\mathbf{CP}}



\newcommand{\from}{\leftarrow}



\newcommand{\red}[1]{\textcolor{red}{#1}}
\newcommand{\blue}[1]{\textcolor{blue}{#1}}

% Reduction relation macros
\newcommand{\rstep}{\mathbin{\longrightarrow_R}}
\newcommand{\mstep}{\mathbin{\longrightarrow_R^*}}
\newcommand{\estep}{\mathbin{\longrightarrow_R^=}}
\newcommand{\rrstep}{\mathbin{\longrightarrow_R^r}}
\newcommand{\brstep}{\mathbin{\longleftarrow_R^r}}
\newcommand{\bstep}{\mathbin{\longleftarrow_R}}
\newcommand{\bmstep}{\mathbin{\longleftarrow_R^*}}

% Unicode characters

\newunicodechar{∀}{\ensuremath{\forall}}
\newunicodechar{→}{\ensuremath{\rightarrow}}
\newunicodechar{ℕ}{\ensuremath{\mathbb{N}}}
\newunicodechar{↔}{\ensuremath{\leftrightarrow}}
\newunicodechar{⊆}{\ensuremath{\sse}}
\newunicodechar{∧}{\ensuremath{\land}}
\newunicodechar{₌}{\ensuremath{_=}}
\newunicodechar{𝓟}{\ensuremath{\mathcal{P}}}
\newunicodechar{∈}{\ensuremath{\in}}
\newunicodechar{φ}{\ensuremath{\phi}}
\newunicodechar{Σ}{\ensuremath{\Sigma}}
\newunicodechar{₁}{\ensuremath{{}_1}}


% Misc Macros
\newcommand{\terese}{[TeReSe]}
% \newcommand{\ul}[1]{\underline{#1}}
\newcommand{\ule}[1]{\underline{#1:}}
% \newcommand{\setof}[1]{\{#1\}}
\newcommand{\setof}[1]{\left\{#1\right\}}


\newcommand{\tclos}[1]{{#1}^{\scriptscriptstyle{+}}}


\bibliographystyle{plainurl}% the mandatory bibstyle  

  
\title{Formalization of Abstract Rewriting}  
 

\author{Andrew Polonsky}{Appalachian State University, USA}
{polonskya@appstate.edu} 
{} 
{}%TODO mandatory, 
% please use full name; only 1 author per \author macro; 
% first two parameters are mandatory, other parameters can be empty. 
% Please provide at least the name of the affiliation and the country. 
% The full address is optional. Use additional curly braces to indicate the correct
%  name splitting when the last name consists of multiple name parts.
\author{Samuel Arkle}{Appalachian State University, USA (Recent graduate)}
{arklesd@appstate.edu}{}{}

\authorrunning{A. Polonsky and S. Arkle}

\Copyright{Andrew Polonsky and Samuel Arkle}

\ccsdesc[100]{\textcolor{red}{Replace ccsdesc macro with valid one}} %TODO mandatory: Please choose ACM 2012 classifications from https://dl.acm.org/ccs/ccs_flat.cfm 

\keywords{Dummy keyword} %TODO mandatory; please add comma-separated list of keywords

\supplement{https://github.com/DrPolonsky/LAM/tree/main/Relations}% TODO: set a repo just for this submissionoptional, e.g. related research data, source code, ... hosted on a repository like zenodo, figshare, GitHub, ...

 
\begin{document}
\maketitle

\begin{abstract}
    We present a constructive formalization of Abstract Reduction Systems (ARS)
     in the Agda proof assistant, focusing on the results given in Term Rewriting 
     Systems (Terese). We define a taxonomy of concepts related to termination and 
     confluence and investigate their relationships between each other and their 
     classical counterparts. We identify, and eliminate where possible, the use of classical 
     logic in the proofs of standard ARS results. Our analysis leads to refinements and mild
      generalizations of classical termination and confluence criteria.
\end{abstract}
\section{Introduction}
\label{sec:Introduction}

% \section{Plan}
% \begin{itemize}
%   \item Start by discussing motivation
%   \item Then contributions
%   \item Plan of the paper
%   \item At some point, there should be like a summary of what's been done and what are some interesting discoveries
% \end{itemize}
%


We present a formalization of the basic Abstract Rewriting Systems (ARS) theory in the Agda proof assistant.
This work is part of a larger effort to develop a library of formalized programming language theory (PL)
at Appalachian State University.
Since Term Rewriting Systems (TRS) play a foundational role in programming languages,
while ARS encompass the basic facts about rewriting relations in general,
establishing these facts is part of the necessary infrastructure required to
pursue the broader project.  The present contribution can therefore be seen as
``bootstrapping'' Appstate's new formalized PL repository.

This broader goal also drives the main design choices in our approach.
Since rewriting theory is concerned with a fine-grained analysis of computation,
the most natural vehicle for formalizing it is type theory based on the
Curry--Howard isomorphism, which Agda implements.  Proofs in this language
are automatically effective, and implicitly carry the code implementing the
transformations from the hypotheses to the conclusion.  For example,
a proof that a given TRS is terminating automatically renders a function that
computes a normal form of any term in its domain.

Staying within this paradigm requires the proofs to be constructive.
Moreover, it requires us to avoid any axioms or postulates, such as function
extensionality, uniqueness of identity proofs, univalence, etc.
While most of ARS results are indeed constructive, standard presentations,
 including \cite{Terese}, make liberal use of classical logic.
Our paper can thus be read as a thoroughly constructive
development of elementary ARS theory.

\subsection{Motivation}

Before we proceed further, let us consider two formalization scenarios where
a constructive ARS library can be useful.

\begin{example}[Formalization of a typed lambda calculus]
Suppose one is formalizing the standard metatheory of
a typed lambda calculus.  To show confluence, one can
first establish the Church--Rosser theorem (CR) for the untyped lambda calculus,
and then proceed via the Subject Reduction theorem.
However, suppose that Strong Normalization had already been verified independently.
Invoking Newman's lemma, one can conclude confluence from the weak
diamond property --- which is generally easier to prove than full CR.
\end{example}

\begin{example}[Quotients of types]
Suppose one wishes to construct the initial model of an algebraic theory $T$
over a first-order signature $S$.
(For example, the reader could think here of the free commutative semiring over a finite set of generators.)
The classical definition involves the quotient of the term algebra $S^*$ by
the equivalence relation $\approx_T$ induced by $T$.
Direct encoding of quotients is problematic in pure type theory.
While there are a number of solutions, such as Higher Inductive Types \cite{HoTT},
Quotient Inductive--Inductive Types \cite{QIIT}, Cubical Type Theory \cite{CTT}, etc.,
they involve extending type theory with new language constructs,
incurring additional costs in complexity.

Now, suppose $T$ admits a convergent presentation.
That is, suppose that the equivalence relation on $S^*$ coincides
with conversion generated by a confluent and terminating rewriting system
${\cdot}{\to_R}{\cdot} : S^* \to S^* \to \bset$.
Then one can replace $S^*/{\approx_T}$ by the type of $R$-normal forms.
(In the semiring example, these would be appropriately sorted polynomial expressions.)
This can be defined as a pure dependent type
\[ NF(R) \quad := \quad (\Sigma n : S^*) (\Pi x : S^*) \lnot (n \to_R x) \]
This type encodes the quotient in the sense that the type of functions $S^*/{\approx_T} \to A$ is isomorphic to plain functions from $NF(R)$ to $A$.
Indeed, the elimination rule for the quotient type requires every equation in $T$
to be validated in order to define a function $f : S^*/{\approx_T} \to A$.
Any such function trivially specializes to $NF(R)$.  In the other direction,
a plain function from $NF(R)$ to $A$
can be extended to all of $S^*$ by induction on normalizing reduction sequences.
Confluence ensures that $T$-equal elements are always mapped to the same value.

It follows that $S^*/{\approx_T}$ is isomorphic to $NF(R)$.

It also follows as a corollary that equality in the initial model is decidable.
\end{example}

As these examples illustrate, a fully effective implementation of basic ARS results
can be helpful in many settings arising in programming language theory.

\subsection{Formalization principles}
With the explicit goal of having the results of basic abstract rewriting
formalized in a canonical way, so that they can be used in building
libraries of formalized PL theory, we adopted the following
principles during our development.

\begin{itemize}
  \item No use of function extensionality;
  \item No use of univalence, uniqueness of identity proofs, axiom K, or any other
  assumptions related to equality;
  \item Minimize the use of classical logic, and make decidability hypotheses explicit in every place where classical logic could not be avoided;
  \item Stay as faithful as possible to the spirit of type theory based on the Curry--Howard isomorphism, so that in every explicit application, our proofs would compute.
\end{itemize}

\subsection{Decidability}
In several places, we had to assume certain decidability hypotheses in order to make
the proofs go through.  These were always flagged explicitly, and an effort was made to
have as few of them as possible.

Since Strong Normalization is a central concept in ARS theory,
special attention in our work was given to various incaranations of the concept of well-foundedness in the constructive context.
(Strong normalization is well-foundedness of the converse relation.)
The standard constructive definition does not even allow one to show that SN implies WN, see Example \ref{ex:undec} \apcomment{Refer to DecidableCounterexamples.agda?}.
  We therefore looked at a number of variations of this notion, their classical counterparts all being logically equivalent.  This gave rise to a rich set of concepts, see Figure \ref{f:wf}.

We have also identified classes of ARSs where the needed classical principles are simply valid.  In particular, for finitely branching relations, the implication SN to WN requires nothing else beyond plain decidability of the relation itself: $Rxy \lor \lnot Rxy$.  For example, an ARS induced by a first-order TRS with a finite set of rules is both finitely branching and decidable.

The main conclusion of our work is that most of the main ARS results can be made completely effective, at least for the practical examples encountered in first-order TRSs and lambda calculi.

For more exotic rewrite systems --- like $\lambda \bot$ (underpinning sensible lambda theories), or coinductive rewriting ---
these decidability assumptions no longer hold, and the utility of the ARS framework diminishes proportionally.

\subsection{Contributions}
Our contributions are as follows.
\begin{itemize}
  \item Formalization of elementary ARS theory as presented in
  \cite{Terese}, including relevant background;
  % Chapter 1 of
  \item An ontology of termination and confluence properties, and a detailed analysis of logical relationships between them;
  \item Definitions of new ARS concepts that refine our understanding of these relationships;
  \item Marginal improvements to classical confluence and termination criteria;
  \item An examination of several distinct notions of well-foundedness
  in the constructive setting.
\end{itemize}

\subsection{Plan of the paper}

The structure of the paper roughly corresponds with the progression of the Agda code.
% Agda files.
% (with the exception of section \ref{sec:Well-foundedness}, which is put at the end)
In the next section, we lay out the main properties of relations studied in ARS theory,
setting the ground for what follows.  In Section \ref{sec:Formalization}, we
outline our formalization of the ARS chapter from \cite{Terese}.
Here we also discuss how this effort suggested new ARS properties that helped us overcome some obstacles we encountered along the way.  These properties fit naturally within the rewriting paradigm.  Thus in Section \ref{sec:Implications}, we dedicate some effort to investigating them further.  The result is a broader ontology of ARS concepts, revealing interrelationships between conditions for completeness.  In Section \ref{sec:Well-foundedness} we focus on what it means for a relation to be well-founded, identifying the classical principles needed to transition between different formulations.
We conclude with a brief summary and suggestions for future work.

\section{Definitions}
\label{sec:Definitions}

The following definitions are provided to ensure clarity and precision 
in our discussion. When useful, we also include the name and type of
definition as we have implemented it in Agda, allowing the report to guide the code
and vice versa.

\begin{definition}
    An abstract rewrite system (\emph{ARS}) is a structure $\mathcal{A} = (A, R_\alpha)$ where
     $A$ is a set of elements and $R_\alpha$ is a set of binary relations on $A$.
\end{definition}

We denote a relation $R$ (from the set $R_\alpha$) from an element $a \in A$ to an element $b \in A$ as $a\rstep b$.
A multi-step relation is denoted $a \mstep b$.

\begin{definition}
    An element $a \in A$ is \emph{R-weakly confluent} ($WCR_R$) if all single step relations from $a$ can always converge
    via multi-step relations to some common reduct.
    
    A relation $R$ is \emph{weakly confluent} (AKA weakly Church-Rosser (WCR)) if every $a \in A$ is R-weakly confluent.
\end{definition}

\begin{definition}
    An element $a \in A$ is \emph{R-confluent} ($CR_R$) if all multi-step relations from $a$ can always converge
    via multi-step relations to some common reduct.
    A relation $R$ is \emph{confluent} (AKA Church-Rosser (CR)) if every $a \in A$ is R-confluent.
\end{definition}

\begin{definition}
    \emph An element $a \in A$ is a {normal form} if there exists no relation to any other element in $A$.
\end{definition}
  
\begin{definition}
    An element $a \in A$ is \emph{$R$-weakly normalizing} ($WN_{R}$)if $a \mstep b$ for some normal form $b \in A$.
  
    A relation $R$ is weakly normalizing (WN) if every $a \in A$ is $R$- weakly normalizing.
\end{definition}

\begin{definition}
    An element $a \in A$ is \emph{$R$-strongly normalizing} ($SN_R$) if every sequence of relations starting from $a$ is finite.
  
    A relation $R$ is strongly normalizing (SN) if every
    $a \in A$ is $R$-strongly normalizing.
\end{definition}

The above definition of SN is taken from [TeReSe]. 
When working in Agda, we make use of the alternative definition presented in [TeReSe] ($\leftarrow$ is accessibly well-founded). 
The \ref{sec:Well-foundedness} section clarifies this definition. 

\begin{definition}
    An element $a \in A$ has the \emph{$R$-weak normal form property} ($WNFP_{R}$) if $a \mstep b$ and 
    $a \mstep c \implies c \mstep b$, for all $c \in A$ and all normal form $b \in A$.
    
    A relation $R$ has the \emph{weak normal form property} (WNFP) if every $a \in A$ has the $R$-weak normal form property.    
\end{definition}

\begin{definition}
    The \emph{equivalence relation} of $R$ is the smallest relation $\estep$ that contains $R$ and is reflexive, transitive, and symmetric.
\end{definition}

\begin{definition}
An element $a \in A$ has the \emph{$R$-normal form property} ($NFP_{R}$) if
$a \estep b$  $\implies$
$a \mstep b$, for any normal form $b \in A$.

A relation $R$ has the \emph{normal form property} (NFP) if every $a \in A$ has the $R$-normal form property.
\end{definition}

In ARS-Implications.agda we show the equivalence of WNFP and NFP when applied globally in the function $\mathtt{NP \leftrightarrow WNFP}$.

In the next definition we deonte two elements $a, b \in A$ being equivalent with $a \equiv b$.
\begin{definition}
An element $a \in A$ has the \emph{{R}-unique normal form property} ($UN_{R}$) if
$a \estep b$  $\implies$
$a \equiv b$, for any normal form $b \in A$.

A relation $R$ has the \emph{unique normal form property} ($UN$) if every $a \in A$ has the $R$-unique normal form property.
\end{definition}


\begin{definition}
An element $a \in A$ has the \emph{${R}$-unique normal form property with respect to reduction} ($UN^ \to _{R}$) if
$b \bmstep  \cdot  \mstep c$  $\implies$ $b \equiv c$, for all normal forms $b,c \in A$.

A relation $R$ has the \emph{unique normal form property with respect to reduction} ($UN^\to$) if every $a \in A$ has the
$R$-unique normal form property with respect to reduction.
\end{definition}
In ARS-Implications.agda we show that $UN$ implies $UN\to$ but the inverse implication doesn't hold, as seen 
in counterexample (\sacomment{TODO: put counterexample here when section is complete}).

\begin{definition}
    A \emph{sequence (in $A$)} is a function from the natural numbers to $A$:
    \begin{align*}
      &s : \nat \to A \\
      &s = (s_0,s_1,s_2,\dots)
    \end{align*}
\end{definition}
  
\begin{definition}
    A sequence is \emph{$R$-increasing} if every term is $R$-related to its preceding term.
\end{definition}
  
\begin{definition}
    A sequence is \emph{bounded} if there exists an element $a \in A$ to which all elements of the sequence reduce.
\end{definition}

\begin{definition}
    An element $a \in A$ has the \emph{minimal form property} ($MF$) if $a \mstep b \implies b \mstep a$ for all $b \in A$.
\end{definition}

Note that being a normal form is a trivial case of the minimal form property (as we prove in ARS-implications.agda 
function $\mathtt{NF\subseteq MF : \forall {x} \to NF x \to MF x}$).

\begin{definition}
    An element $a \in A$ is \emph{$R$-weakly minimal} ($WM_{R}$)if $a \mstep b$ for some minimal form $b \in A$.
  
    A relation $R$ is weakly minimal (WM) if every $a \in A$ is $R$-weakly minimal.
\end{definition}

\begin{definition}
    An element $a \in A$ is \emph{$R$-strongly minimal} ($SM_R$) if either $a$ itself is a minimal form or every element one $R$ step
    from $a$ is also strongly minimal.
  
    A relation $R$ is strongly minimal (SM) if every
    $a \in A$ is $R$-strongly minimal.
\end{definition}

\begin{definition}
   An element $a \in A$ has the \emph{${R}$-weak minimal form property} ($WMFP_R$) if
   $a \mstep b$ and $a \mstep c$  $\implies$
   $c \mstep b$, for all $c \in A$ and all $b \in A$ with the minimal form property.

   A relation $R$ has the \emph{weak minimal form property} (WMFP) if every $a \in A$ has the $R$-weak minimal form property.
\end{definition}

Definitions to add:
\begin{enumerate}
    \item Finite branching
    \item Any other terms from well foundedness
    \item The other terms we have for strongly 
\end{enumerate}







\section{Hierarchy}
\label{sec:Hierarchy}

\newenvironment{counterexample}[1][]{%
    \refstepcounter{CEcounter} % Increment the counter
    \noindent \scriptsize\textbf{{\theCEcounter } }  #1\par
}

% From the perspective of programming language theory, the two properties of abstract rewriting 
% systems that are most interesting are $\SN$ and $\CR$. The following hierarchies 
% show chains of implications starting from these properties.
This section examines the hierarchy of ARS properties. 
In programming language theory, $\SN$ and $\CR$ constitute the essential properties that 
jointly establish completeness. 
Consequently, our analysis centers on two primary categories of properties: those associated 
with termination and those associated with confluence.
% Possibly redundant introductory paragraph

As discussed in Subsection \ref{subsec:def} we put forward $\RP$ as an alternative for $\Inc$. It is through examining $\RP$ that 
we came to the `minimal' family of properties below. These properties help to augment our ARS hierarchy. 

\begin{definition}\label{def:mf} \hfill
    \begin{description}
        \item[\ule{$a \in \MF_R$}] \emph{$a$ is a minimal form} if $a \mstep b \implies b \mstep a$ for any $b \in A$.
        \item[\ule{$a \in \WM_R$}] \emph{$a$ is weakly minimalizing} if $a \mstep b$ for some $b \in \MF_R$.
        \item[\ule{$a \in \SMseq_R$}] \emph{$a$ is sequentially strongly minimalizing} if for every $R$-increasing sequence $s$ starting at $a$, there exists 
        an $i \in \nat$ such that $s (i) \in \MF$. 
        \item[\ule{$a \in \SM_R$}]  \emph{$a$ is strongly minimalizing} if either $a \in MF_R$, or every element one $R$ step from $a$ is strongly minimalizing. (This 
        definition is to be understood inductively).
        \item[\ule{$a \in \MP_R$}] \emph{$a$ has the minimal form property} if $c \bmstep a \mstep b \implies c \mstep b$ for any $b \in \MF$.

    \end{description} 
\end{definition}

Trivially, the following holds. 
\begin{proposition}\label{prop:nftomf}
    $\NF \implies \MF$ 
\end{proposition}    
\begin{proof}
    See, \verb|NF ⊆ MF : ∀x → NFx → MFx| in \texttt{Implications.agda}.
\end{proof}

$\SMseq$ is only classically equivalent to $\SM$. The 
following proposition relates these new definitions with our previous $\RP$ definitions.
\begin{proposition}\label{prop:SMRP}
    $SMseq \leftrightarrow RP \land BP$
\end{proposition}
\begin{proof}
    See,\\ 
    \verb|RP∧BP→SMseq : R isRP → R isBP → ∀ {x : A} → SMseq R x| \\ 
    \verb|RisSMseq→RisRP : (∀ {x : A} → SMseq R x) → R isRP|\\ 
    \verb|RisSMseq→RisBP : (∀ {x : A} → SMseq R x) → R isBP| \\
    in \texttt{Implications.agda}.
\end{proof}


Proposition \ref{prop:nftomf} suggests that we consider $\MF$ as a termination property in its own right. This leads us to 
explore the following broader taxonomy of concepts relating to termination and confluence.

\begin{center}
    \begin{tikzpicture}[auto,
      arrowstyle/.style={->, line width=1pt, >={Latex[length=3mm, width=2mm]}, shorten >=2pt}]
      % A box style
      \tikzstyle{boxnode} = [draw, rectangle, text centered, minimum width=1.2cm, inner sep=3pt]
      
      % Title nodes
      \node[text width=4cm, align=center, font=\bfseries] at (-1.5,1) {Confluence Hierarchy};
      \node[text width=4cm, align=center, font=\bfseries] at (5,1) {Termination Hierarchy};
  
      % Place the nodes vertically (top to bottom)
      \node (Confluent) [boxnode] at (-4,0) {$\CR$};
      \node (RP)        [boxnode, right=.5cm of Confluent] {$\MP$};
      \node (WN)        [boxnode, right=.5cm of RP] {$\NP$};
      \node (UN)        [boxnode, right=.5cm of WN] {$\UNto$};
  
      % Draw arrows downwards
      \draw[arrowstyle] (Confluent) -- (RP);
      \draw[arrowstyle] (RP) -- (WN);
      \draw[arrowstyle] (WN) -- (UN);
   
        
      % Define the nodes
      \node (NF) [boxnode] at (3,0) {$\NF$};
      \node (MF) [boxnode] at (3,-1) {$\MF$};
      \node (SN) [boxnode] at (5,0) {$\SN$};
      \node (SR) [boxnode] at (5,-1) {$\SM$};
      \node (WN) [boxnode] at (7,0) {$\WN$};
      \node (WR) [boxnode] at (7,-1) {$\WM$};
      
      % Draw the arrows to form a diamond
      \draw [arrowstyle] (NF) -- (SN);
      \draw [arrowstyle] (NF) -- (MF);
      \draw [arrowstyle] (MF) -- (SR);
      \draw [arrowstyle] (SN) -- (WN);
      \draw [arrowstyle] (SN) -- (SR);
      \draw [arrowstyle] (WN) -- (WR);
      \draw [arrowstyle] (SR) -- (WR);
      
      % Optional: Add a separation line
      \draw[dashed] (2.0,-1.5) -- (2.0,1.5);
    \end{tikzpicture}
\end{center}

The implications in this diagram are formalized in \texttt{Implications.agda}. 
The confluence hierarchy requires no classical assumptions, however this is not the case for the hierarchy of terminating properties.
The implication from $\SN$ to $\WN$ relies on the classical decidability of the property 
of being $R$-minimal, as demonstrated in our formalization:

\verb|SNdec→WN : (~R R) isMinDec → SN ⊆ WN| 
 
The same property is similarly required in order to show that $\SM$ implies $\WM$.

Having established the hierarchies of ARS properties, we now examine which property combinations are sufficient to reverse these implications and achieve completeness. The following tables illustrate, for each property combination, whether confluence (left box) and strong normalization (right box) are obtained. Where properties cannot be derived, we provide counterexamples demonstrating the necessity of additional conditions.

The first table explores where the properties hold locally, the second explores where the properties hold globally. Counterexamples in the 
first table assume $\WCR$ and in the second table assume $\gWCR$ (as a reminder, bold text is used to indicate a property holds globally). 

\clearpage
\renewcommand*{\thefootnote}{\fnsymbol{footnote}}



\begin{table}[h!]
    \centering
    \renewcommand\arraystretch{1.2}
    \begin{tabular}{!{\vrule width 1.5pt}
        >{\columncolor{gray!30}}l
        !{\vrule width 1.5pt}c|c
        !{\vrule width 1.5pt}c|c
        !{\vrule width 1.5pt}c|c
        !{\vrule width 1.5pt}c|c
        !{\vrule width 1.5pt}c|c!{\vrule width 1.5pt}}
        \Xhline{1.5pt}
        \rowcolor{gray!30} 
        & \multicolumn{2}{c!{\vrule width 1.5pt}}{$\WM$}
        & \multicolumn{2}{c!{\vrule width 1.5pt}}{$\WN$}
        & \multicolumn{2}{c!{\vrule width 1.5pt}}{$\SM$}
        & \multicolumn{2}{c!{\vrule width 1.5pt}}{$\SMandWN$}
        & \multicolumn{2}{c!{\vrule width 1.5pt}}{$\SN$} \\
        \Xhline{1.5pt}
        $\UNto$ & CE-\ref{CE:4} & CE-\ref{CE:11} & CE-\ref{CE:4} & CE-\ref{CE:6} & $\CR$\footnotemark[1] & CE-\ref{CE:8} & $\CR$\footnotemark[1] & CE-\ref{CE:4} & $\CR$\footnotemark[1] & $\SN$ \\
        \hline
        $\NP$ & CE-\ref{CE:9} & CE-\ref{CE:8} & $\CR$ & CE-\ref{CE:6} & $\CR$\footnotemark[1] & CE-\ref{CE:8} & $\CR$ & $\SN$ & $\CR$\footnotemark[2] & $\SN$ \\
        \hline
        $\MP$ & $\CR$ & CE-\ref{CE:8} & $\CR$ & CE-\ref{CE:6} & $\CR$\footnotemark[2] & CE-\ref{CE:8} & $\CR$ & $\SN$ & $\CR$\footnotemark[2] & $\SN$ \\
        \hline
        $\CR$ & $\CR$ & CE-\ref{CE:8} & $\CR$ & CE-\ref{CE:6} & $\CR$ & CE-\ref{CE:8} & $\CR$ & $\SN$ & $\CR$ & $\SN$ \\
        \hline
        \Xhline{1.5pt}
    \end{tabular}
    \caption{Local implications}
\end{table}

\vspace{-1cm}
% \\[-0.5cm]
\begin{table}[h!]
    \centering
    \renewcommand\arraystretch{1.2}
    \begin{tabular}{!{\vrule width 1.5pt}
        >{\columncolor{gray!30}}l
        !{\vrule width 1.5pt}c|c
        !{\vrule width 1.5pt}c|c
        !{\vrule width 1.5pt}c|c
        !{\vrule width 1.5pt}c|c
        !{\vrule width 1.5pt}c|c!{\vrule width 1.5pt}}
        \Xhline{1.5pt}
        \rowcolor{gray!30} 
        & \multicolumn{2}{c!{\vrule width 1.5pt}}{$\gWM$}
        & \multicolumn{2}{c!{\vrule width 1.5pt}}{$\gWN$}
        & \multicolumn{2}{c!{\vrule width 1.5pt}}{$\gSM$}
        & \multicolumn{2}{c!{\vrule width 1.5pt}}{$\gSMandWN$}
        & \multicolumn{2}{c!{\vrule width 1.5pt}}{$\gSN$} \\
        \Xhline{1.5pt}
        $\gUNto$ & CE-\ref{CE:2} & CE-\ref{CE:8} & $\gCR$ & CE-\ref{CE:6} & $\gCR$\footnotemark[1] & CE-\ref{CE:8} & $\gCR$ & $\gSN$ & $\gCR$\footnotemark[2] & $\gSN$ \\
        \hline
        $\gNP$ & CE-\ref{CE:3} & CE-\ref{CE:8} & $\gCR$ & CE-\ref{CE:6} & $\gCR$\footnotemark[1] & CE-\ref{CE:8} & $\gCR$ & $\gSN$ & $\gCR$\footnotemark[2] & $\gSN$ \\
        \hline
        $\gMP$ & $\gCR$ & CE-\ref{CE:8} & $\gCR$ & CE-\ref{CE:6} & $\gCR$\footnotemark[2] & CE-\ref{CE:8} & $\gCR$ & $\gSN$ & $\gCR$\footnotemark[2] & $\gSN$ \\
        \hline
        $\gCR$ & $\gCR$ & CE-\ref{CE:8} & $\gCR$ & CE-\ref{CE:7} & $\gCR$ & CE-\ref{CE:8} & $\gCR$ & $\gSN$ & $\gCR$ & $\gSN$ \\
        \Xhline{1.5pt}
    \end{tabular}
    \caption{Global implications}
\end{table}
\footnotetext[1]{This implication also requires $\gWCR$.}
\footnotetext[2]{This implication also requires either $\gWCR$ or the classical property required to go from 
$\SN \to \WN$ or $\SM \to \WM$.}
\renewcommand*{\thefootnote}{\arabic{footnote}}

The table should be interpreted as follows: for example, if a relation possesses the property $\NP$, then, according to the table, obtaining the property $\CR$ requires, at a minimum, the additional property $\WN$. Counterexample~\ref{CE:9} demonstrates that the combination of $\WM$ and $\NP$ is insufficient to guarantee $\CR$.

\subsection{Counterexamples}\label{subsec:counterexamples}
% Setting up the ce counter 
\newcounter{CEcounter}
\renewcommand{\theCEcounter}{\arabic{CEcounter}}

    \begin{center}
        \begin{tabular}{ccc}  % Three columns per row
            % First Counterexample
            \begin{minipage}{0.3\textwidth}
                \centering
                \begin{counterexample}\label{CE:1}
                    \begin{tikzcd}[row sep=small, column sep=small]
                        a & b \arrow[l] \arrow[r, bend left] & c \arrow[l, bend left] \arrow[r] & d
                    \end{tikzcd} \\
                \end{counterexample}
            \end{minipage}
            &
            % Second Counterexample
            \begin{minipage}{0.3\textwidth}
                \centering
                \begin{counterexample}\label{CE:2}
                    \begin{tikzcd}[row sep=small, column sep=small]
                        a & b \arrow[l] \arrow[r, bend left] & c \arrow[l, bend left] \arrow[r] & d \arrow[loop right]
                    \end{tikzcd} \\
                \end{counterexample}
            \end{minipage}
            &
            % Third Counterexample
            \begin{minipage}{0.3\textwidth}
                \centering
                \begin{counterexample}\label{CE:3}
                    \begin{tikzcd}[row sep=small, column sep=small]
                        a \arrow[loop left] & b \arrow[l] \arrow[r, bend left] & c \arrow[l, bend left] \arrow[r] & d \arrow[loop right]
                    \end{tikzcd} \\
                \end{counterexample}
            \end{minipage}
            \\  % New row
            \\
            
            % Fourth Counterexample
            \begin{minipage}{0.3\textwidth}
                \centering
                \begin{counterexample}\label{CE:4}
                    \begin{tikzcd}[row sep=small, column sep=small]
                        & a \arrow[d] \arrow[dl] \\ e & b \arrow[l] \arrow[r] & c \arrow[r, bend left] & d \arrow[l, bend left]
                    \end{tikzcd} \\
                \end{counterexample}
            \end{minipage}
            &
            % Fifth Counterexample
            % \begin{minipage}{0.3\textwidth}
            %     \centering
            %     \begin{counterexample}\label{CE:5}
            %         \begin{tikzcd}[row sep=small, column sep=small]
            %             d & a \arrow[l, two heads] \arrow[r, two heads] & b \arrow[r, bend left] & c \arrow[l, bend left]
            %         \end{tikzcd} \\
            %     \end{counterexample}
            % \end{minipage}
            
            % Sixth Counterexample - This one shows why UN -> doesn't imply UN
            \begin{minipage}{0.3\textwidth}
                \centering
                \begin{counterexample}\label{CE:5}
                    \begin{tikzcd}[row sep=small, column sep=small]
                        n & c \arrow[l] \arrow[r] & a \arrow[r, bend left] & b \arrow[l, bend left]
                        & d \arrow[l] \arrow[r] & m
                    \end{tikzcd} \\
                \end{counterexample}
            \end{minipage}
        % \\ % New row
        % \\
        &
        \begin{minipage}{0.3\textwidth}
            \centering
            \begin{counterexample}\label{CE:6}
            \begin{tikzcd}[row sep=small, column sep=small]
                f_0 \arrow[r] \arrow[dr] & f_1 \arrow[r] \arrow[d] & f_2 \arrow[dl] \arrow[r] & \dots \arrow[dll] \\
                & n 
            \end{tikzcd} \\
            \end{counterexample}
        \end{minipage}
        \\ % new row
        \\
        
        \begin{minipage}{0.3\textwidth}
            \centering
            \begin{counterexample}\label{CE:7}
            \begin{tikzcd}[row sep=small, column sep=small]
                f_0 \arrow[r] \arrow[d] & f_1 \arrow[r] \arrow[d] & f_2 \arrow[r] \arrow[d] & \dots \arrow[d] \\
                n_0  & n_1  & n_2 & \dots 
            \end{tikzcd} \\
            \end{counterexample}
        \end{minipage}
        
        &
        \begin{minipage}{0.3\textwidth}
            \centering
            \begin{counterexample}\label{CE:8}
            \begin{tikzcd}[row sep=small, column sep=small]
                a \arrow[r, bend left] & b \arrow[l, bend left]
            \end{tikzcd} \\
        \end{counterexample}
    \end{minipage}
    % \\ % New row
    % \\
    &
    \begin{minipage}{0.3\textwidth}
        \centering
        \begin{counterexample}\label{CE:9}
            \begin{tikzcd}[row sep=small, column sep=small]
                d  \arrow[d] \arrow[rr, bend left] & a \arrow[r] \arrow[l]  &  
                b \arrow[r, bend left] & c \arrow[l, bend left] \\  
                f_0 \arrow[r] & f_1 \arrow[r] & f_2 \arrow[r] & \dots  
            \end{tikzcd} \\
        \end{counterexample}
    \end{minipage}
    % New row
    \\
    \\
    \begin{minipage}{0.3\textwidth}
        \centering
        \begin{counterexample}\label{CE:10}
            \begin{tikzcd}[row sep=small, column sep=small]
                c \arrow[r, bend left] & b \arrow[l, bend left]  & a \arrow[l] \arrow[r] 
                & d \arrow[r, bend left] & e \arrow[l, bend left]
            \end{tikzcd} \\
        \end{counterexample}
    \end{minipage}
    &
    &
    \begin{minipage}{0.3\textwidth}
        \centering
        \begin{counterexample}\label{CE:11}
            \begin{tikzcd}[row sep=small, column sep=small]
                c & a \arrow[r, bend left] \arrow[l] & b \arrow[l, bend left]
            \end{tikzcd} \\
        \end{counterexample}
    \end{minipage}
    
    
\end{tabular}
\end{center}

\begin{remark} The following are the key takeaways from the implication tables. \hfill
    \begin{enumerate}
        \item The property $\gWCR$ can sometimes be a substitute for the classical 
        property required for $\SN \implies \WN$ (and similarly $\SM \implies \WM$).   
        \item $\gWN \land \gUNto \implies \gCR$ does not require any classical assumptions
         or $\gWCR$ for the implication to $\gCR$, unlike $\gSN \land \gUNto$. 
        \item $\gWN \land \gUNto \implies \gCR$ only holds when the properties are global. The 
        counterexample provided illustrates why $\WCR \land \WN \land \UNto \nRightarrow \CR$.
        \item We have a generalization of Newman's Lemma as $\SM \land \gWCR \implies \CR$. This is explored in Subsection \ref{subsec:newnewman}.
        \item Both $\SM$ and $\WN$ are required for $\SN$, this connection is explored in Subsection \ref{subsec:SMWNSN}.
    \end{enumerate}
\end{remark}

\subsection{Generalized Newman's Lemma}\label{subsec:newnewman}
One interesting outcome from investigating the normalization taxonomy was a generalization of Newman's Lemma. 

\begin{proposition}
    $\gWCR \land \SM \implies \CR$
\end{proposition}
\begin{proof}
    See, \verb|LocalNewmansLemmaRecurrent : R isWCR → SM ⊆ CR|
\end{proof}

This proof follows the same steps as the second proof of Newman's Lemma found in \terese, but 
uses the weaker assumption $\SM$ in place of $\SN$. 

% The tables show a number of changes when moving from properties holding locally to globally. 
% Globally $\WN \land \UNto \implies \CR$ holds whereas it does not locally. Similarly, $\SM \land \WN \implies \SN$ holds 
% globally but not locally. 
\subsection{Relation between $\SM$, $\WN$, and $\SN$} \label{subsec:SMWNSN}
One key takeaway from the tables is that we obtain completeness when we have the properties $\SM$, $\WN$, and $\NP$. We also 
obtain completeness when we have $\gWN$ and $\gSM$. 
To show that $\gSM \land \gWN \implies \gSN$ we built on our proof that $\SM \land \WN \land \NP \implies \SN$. If $\gWN$ holds 
then our proof no longer requires the property $\NP$. This progression of proofs can be seen in the functions:

\verb|WN∧NP∧SM→SN : ∀ {x} → WN x → NP x → SM x → SN x| 

\verb|isWN∧SM→SN : R isWN → ∀ {x} → SM x → SN x|

\verb|isWN∧isSM→isSN : R isWN → R isSM → R isSN| 



\section{Counterexamples}
\label{sec:Counterexamples}

% Setting up the ce counter 
\newcounter{CEcounter}
\renewcommand{\theCEcounter}{\arabic{CEcounter}}

\newenvironment{counterexample}[1][]{%
    \refstepcounter{CEcounter} % Increment the counter
    \noindent \scriptsize\textbf{{Counterexample \theCEcounter.}} #1\par
}

This section collects the useful counterexamples we have encountered whilst undertaking this project. 
The counterexamples rule out certain implications between properties.

\begin{center}
\begin{tabular}{ccc}  % Three columns per row
    % First Counterexample
    \begin{minipage}{0.3\textwidth}
        \centering
        \begin{counterexample}\label{CE:1}
        \begin{tikzcd}[row sep=small, column sep=small]
            a & b \arrow[l] \arrow[r, bend left] & c \arrow[l, bend left] \arrow[r] & d
        \end{tikzcd} \\
    \end{counterexample}
    \end{minipage}
    &
    % Second Counterexample
    \begin{minipage}{0.3\textwidth}
        \centering
        \begin{counterexample}\label{CE:2}
        \begin{tikzcd}[row sep=small, column sep=small]
            a & b \arrow[l] \arrow[r, bend left] & c \arrow[l, bend left] \arrow[r] & d \arrow[loop right]
        \end{tikzcd} \\
        \end{counterexample}
    \end{minipage}
    &
    % Third Counterexample
    \begin{minipage}{0.3\textwidth}
        \centering
        \begin{counterexample}\label{CE:e}
        \begin{tikzcd}[row sep=small, column sep=small]
            a \arrow[loop left] & b \arrow[l] \arrow[r, bend left] & c \arrow[l, bend left] \arrow[r] & d \arrow[loop right]
        \end{tikzcd} \\
    \end{counterexample}
    \end{minipage}
    \\  % New row
    \\

    % Fourth Counterexample
    \begin{minipage}{0.3\textwidth}
        \centering
        \begin{counterexample}\label{CE:4}
        \begin{tikzcd}[row sep=small, column sep=small]
            n \in NF & a \arrow[l] \arrow[r] & b \arrow[r, bend left] & c \arrow[l, bend left]
        \end{tikzcd} \\
    \end{counterexample}
    \end{minipage}
    &
    % Fifth Counterexample
    \begin{minipage}{0.3\textwidth}
        \centering
        \begin{counterexample}\label{CE:5}
        \begin{tikzcd}[row sep=small, column sep=small]
            n \in NF & a \arrow[l, two heads] \arrow[r, two heads] & b \arrow[r, bend left] & c \arrow[l, bend left]
        \end{tikzcd} \\
    \end{counterexample}
    \end{minipage}
    &
    % Sixth Counterexample
    \begin{minipage}{0.3\textwidth}
        \centering
        \begin{counterexample}\label{CE:6}
        \begin{tikzcd}[row sep=small, column sep=small]
            a \arrow[r, bend left] & b \arrow[l, bend left]
        \end{tikzcd} \\
        \end{counterexample}
    \end{minipage}
    \\ % New row
    \\
    \begin{minipage}{0.3\textwidth}
        \centering
        \begin{counterexample}\label{CE:7}
        \begin{tikzcd}[row sep=small, column sep=small]
            f_0 \arrow[r] \arrow[dr] & f_1 \arrow[r] \arrow[d] & \dots \arrow[dl] \\
            & n \in NF 
        \end{tikzcd} \\
        \end{counterexample}
    \end{minipage}
    &
    \begin{minipage}{0.3\textwidth}
        \centering
        \begin{counterexample}\label{CE:8}
        \begin{tikzcd}[row sep=small, column sep=small]
            f_0 \arrow[r] \arrow[d] & f_1 \arrow[r] \arrow[d] & f_2 \arrow[r] \arrow[d] & \dots \arrow[d] \\
            n_0  & n_1  & n_2 & \dots 
        \end{tikzcd} \\
        \end{counterexample}
    \end{minipage}
    
    &
    \begin{minipage}{0.3\textwidth}
        \centering
        \begin{counterexample}\label{CE:9}
        \begin{tikzcd}[row sep=small, column sep=small]
            a \arrow[r, bend left] & b \arrow[l, bend left]
        \end{tikzcd} \\
        \end{counterexample}
    \end{minipage}
    
    

\end{tabular}
\end{center}

\section{Well-foundedness}
\label{sec:Well-foundedness}

\section{TODO}

I thought it would be useful to have a final todo in which we can list missing work or questions for this final report.

\begin{enumerate}
    \item Wording question. I'm trying to avoid stating "reduction" when talking about relations. However, can I still refer to a reduct and make sense? See for example the defintion of WCR.
    \item We have shown SN implies SMseq, can we show SN implies SM?
    \item Need to fill in ACM subject classifications
    \item Need to fill in keywords and phrases
    \item Need to check what other mandatory information is required in the file 
    \item Need to fill in Digital object identifier 
    \item Need to sort out a repository for the code and add it to the supplementary material section.  
    \item Add contents sections?
    \item Add bibliography/references.
    \item FINAL Remove this TODO section from the report 
\end{enumerate}

There will also be work we can do once we have submitted the file. We can jot that down here so we don't forget:

\begin{enumerate}
    \item Tidy up the code (fill in or remove goals. Remove unneeded comments. Make sure still compiles after moving it. Make sure it is consistent).
    \item We should create a good clear section of future work so we can pass this project on to future students.
    \item {We should remember that If global WCR and local SM implies local CR, then if we also have local WN, we get local SN.}
    \item \sacomment{Do we include more information on the propositions in the formalization section?}
    \item \sacomment{Proove the forawrd direction of theorem 1.2.3 iv}
    \item Show that SM and SMseq are classically equivalent.
\end{enumerate}


\end{document}

