\section{Formalization of \terese}
\label{sec:Formalization}
This sections acts as a guide of our formalization of \terese and explains the deviations we took.  

\subsection{Formalization of Definitions}
We have formalized those definitions which were necessary for the formalization of the propositions and theorems found 
in Chapter 1 of \terese. We use Agda's built-in \texttt{Set} for sets and define a relation in \texttt{Predicates.agda}.
The closure operators are formalized in \texttt{ARS-Closure.agda}. 
The definitions given in \terese in \texttt{1.1.8 DEFINITION (confluence)} 
are formalized in \texttt{ARS-Base.agda}. We only cover $\WCR$ and $\CR$ of the definitions from this section 
as the other definitions were not involved in the more interesting formalizations.

The definitions given in \terese in \texttt{1.1.13 DEFINITION (normalization)} 
are formalized in \texttt{ARS-Properties.agda}.  


The following is a summary of the differences in definition we have used as mentioned in Section \ref{sec:Definitions}. 
\begin{description}
    \item[$\SN$] We make use of the well-founded defintion, rather than the finite reduction sequence definition.
    \item[$\NPe$] We use the equivalent $\NP$ definition. 
    \item[$\Inc$] We use the more general property $\RP$.      
\end{description}

Our motivation for using $\RP$ rather than $\Inc$ is that we found $\Inc$ to be a particularly difficult property 
to use in a constructive setting. \sacomment{Say more on this motivation and decide where it will be placed.}

\subsection{Propositions}
There are three sets of propositions given in \terese: \texttt{1.1.9}, \texttt{1.1.10}, \texttt{1.1.11}. 
The formalization of the proofs are in \texttt{ARS-Propositions.agda}. These formalizations were straightforward 
and as such are not the focus of this paper. \sacomment{Do we include more information on these propositions?}

\subsection{Theorems}
\subsubsection{Newman's Lemma}
There are three proofs of Newman's Lemma ($\SN \land \WCR \implies \CR$) given in \terese. 
The first and third of these proofs make use of classical properties, 



Three proofs are given in \terese of Newman's Lemma (Theorem 1.2.1). The first and third of 
these proofs we found to rely on classical principles. The second proof was amenablbe to a 
constructive formalization which we carried out in the function 


\verb|NewmansLemma : R isSN → R isWCR → R isCR| \footnotemark[2]
\footnotetext[2]{\texttt{ARS-NewmansLemma.agda}}
We show in section \ref{sec:Implications} that $\SN$ implies $\SM$, and it is from this implication 
that we produce a new generalization of Newman's lemma. Every relation with the properties 
$\SM$ and $\WCR$ has the property of $\CR$. This holds true both locally and globally as we show in:

\verb|LocalNewmansLemmaRecurrent : R isWCR → SM ⊆ CR| \footnotemark[2]

and \verb|GlobalNewmansLemmaRecurrent : R isWCR → R isSM → R isCR| \footnotemark[2]

The formalization of the proof that $\WCR \land \WN \land \Inc \implies \SN$ proved particularly difficult. 
As already mentioned, we substituted out the property $\Inc$ for the property $\RP$ however 
we found that we still require a classical principle in order for the proof to hold (as can be seen in the type of our proof):
\verb|iii : R isWN → R isWCR → R isRP- → dec (SN) → R isSN| \footnotemark[3] \footnotetext[3]{\texttt{ARS-Theorems.agda}}
The classical property $\decSN$ is a property that states it is decidable whether a given 
element has the property of being $\SN$. \sacomment{Should say more on this proof.}

In the next section we take a closer look at the properties we found most crucial for termination and confluence, and the interrelationship between them.

