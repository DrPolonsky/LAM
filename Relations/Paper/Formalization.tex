\section{Formalization of \terese}
\label{sec:Formalization}
This sections is a guide to our formalization of \terese $\,$ and explains the deviations we take from the text.  

\subsection{Formalization of Definitions}\label{subsec:def}
The closure operators are formalized in \texttt{Closure.agda}. 
The definitions grouped as confluent in \terese $\,$ are formalized in \texttt{Base.agda}. 
The definitions grouped as normalizing in \terese $\,$ are formalized in \texttt{Properties.agda}.  

The following is a summary of the differences in definition we have used as mentioned in Section \ref{sec:Definitions}. 
\begin{itemize}
    \item $\SN$ : We define this via the notion of accessibility rather than the non-existence of infinite reduction sequences. 
     See Section \ref{sec:Well-foundedness} for further discussion.
    \item $\NPe$ : We use the equivalent $\NP$ definition. 
    \item $\Inc$ : We use the more general property $\RP$.      
\end{itemize}

As mentioned, the $\Inc$ property can impose a computationally demanding requirement. However, $\Inc$ is used only in Theorem 1.2.3 
and the only crucial aspect of $\Inc$ needed for the proof is that no $R$-increasing 
sequence can have an upper bound.
We establish that the equivalent properties $\RP$ and $\RPm$ are sufficient for replacing $\Inc$ in the theorem (see Subsection \ref{subsec:theorems}), 
which led us to examine the condition on $s (i)$ appearing inside $\RP$ more closely. This condition provides a generalization of the notion of a normal form, 
which we explore in detail in Section \ref{sec:Hierarchy}. 

\subsection{Propositions}
There are three sets of propositions given in \terese: \texttt{1.1.9}, \texttt{1.1.10}, \texttt{1.1.11}. 
The formalization of the proofs are in \texttt{Propositions.agda}. These formalizations are straightforward 
and as such are not the focus of this paper. 

\subsection{Theorems}\label{subsec:theorems}
\subsubsection{Newman's Lemma}
There are three proofs of Newman's Lemma ($\SN \land \WCR \implies \CR$) given in \terese. 
The first and third of these proofs make use of classical reasoning. The second proof is amenable to a 
constructive formalization as shown in the following function in \texttt{ARS-NewmansLemma.agda}:

\verb|NewmansLemma : R isSN → R isWCR → R isCR|.

We show in Subsection \ref{subsec:newnewman} a generalization of Newman's Lemma using a condition distilled from $\RP$. 
The formalizations of the remaining theorems are all found in \texttt{ARS-Theorems.agda}. 
\subsubsection{Theorem 1.2.2}
There are three parts to this theorem. 
\begin{description}
    \item[($i$)] $\gCR \implies \gNPe \implies \gUN$
    \item[($ii$)] $\gWN \land \gUN \implies \gCR$  
    \item[($iii$)] $\gCRs \implies \gCR$
\end{description}

\begin{proof}
    See, \\
    \verb|CR→NP : R isCR → R isNP₌|\\
    \verb|NP→UN : R isNP₌ → R isUN|\\
    \verb|ii : R isWN × R isUN → R isCR|\\
    \verb|UN→∧WN→CR : R isUN→ → R isWN → R isCR|\\
    \verb|iii : subcommutative R → R isCR|\\
    in \texttt{Theorems.agda}.
\end{proof}
\begin{remark}
    In $\mathtt{UN→∧WN→CR}$ a more general proof of ($ii$) is given using $\UNto$.
\end{remark}

\subsubsection{Theorem 1.2.3}
The following definition is required for this theorem.
\begin{definition}\hfill
    \begin{description}
        \item[\ule{$a \in CP_R$}] $a$ has the \emph{cofinality property} if there exists an 
        (not necessarily strictly) increasing sequence starting from $a$ such that every
        reduct of $a$ reduces to some element of the sequence.
    \end{description}
\end{definition}

There are four parts to this theorem. 
\begin{description}
    \item[($i$)] $\gWN \land \gUN \implies \gBP$
    \item[($ii$)] $\gBP \land \gInc \implies \gSN$  
    \item[($iii$)] $\gWCR \land \gWN \land \gInc \implies \gSN$
    \item[($iv$)] $\gCR \iff \gCP$ ($\gCR \implies \gCP$ only holds if the ARS is countable.)
\end{description}
\begin{proof}
    See, \\
    \verb|i : R isWN → R isUN → R isBP|\\
    \verb|i→ : R isWN → R isUN→ → R isBP|\\
    \verb|iiSeq : R isWN → R isUN → R isRP → isWFseq- (~R R)|\\
    \verb|iii : R isWN → R isWCR → R isRP- → dec (SN) → R isSN|\\
    \verb|iv : R isCP → R isCR| \\
    in \texttt{Theroems.agda}.
\end{proof}

\begin{remark} \hfill
    \begin{itemize}        
        \item In $\mathtt{i\to}$ we again provide a slight improvement by using $\gUNto$ rather than $\gUN$.
        \item ($ii$) is inherently classical. In $\mathtt{iiSeq}$, we show that these properties imply $\isWFseqm$; establishing the implication to $\gSN$ would require additional classical assumptions. The definition and further discussion of $\isWFseqm$ are provided in Section~\ref{sec:Well-foundedness}. 
        \item In $\mathtt{iii}$, we replace $\gInc$ with the more general property $\gRPm$. We also assume that $\gSN$ is decidable, as this allows the construction of a witness for whether an element possesses the property $\SN$, which is essential for our proof.
        \item We did not formalize $\gCR \to \gCP$ as it requires cardinality assumptions.
    \end{itemize}
\end{remark}




