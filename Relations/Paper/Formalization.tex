\section{Formalization of Terese}
\label{sec:Formalization}
The original ambition of this project was to formalize the contents of the \emph{abstract reduction 
systems} chapter in Term Rewriting Systems [TeReSe], a standard text in rewriting. 
The motivation for the project was to provide construvtive proofs in Agda 
of the propositions and theorems in the chapter. This in turn provides a foundation for 
future work in programming languages theory. 

In order to formalize the chapter we have aimed to provide constructive counterparts to the 
classical proofs found in the text, and to make explicit any classical properties we depend 
on when a constructive proof could not be found. It is from this formalization that our 
original work has emerged. We have gone beyond \terese in fleshing out the interrelationship 
of the properties most interesting from the perspective of programming languages theory (namely 
being confluent and strongly normalizing). 
For some properties we suggest alternatives which are more conducive to a constructive project. 
We look into many notions of well-foundedness and again explore their interrelationship. 

This section explains the deviations we took from \terese and lays the groundwork for the original work 
that we have carried out. 

\sacomment{Maybe mention where we have stayed true to the book, but I don't think this is key}

Our first deviation from \terese is in defining strong normalization. The book defines 
an element as strongly normalizing if every reduction sequence starting from it is finite. 
We wanted to avoid introducing the notion of finiteness to our definition and so we 
followed the alternative definition given in \terese : a relation is strongly normalizing when its 
converse is well founded. In \terese a particular definition of well-foundedness is given but 
it is one with which we were not satisfied and so we explored different definitions of 
well-foundedness. This exploration is detailed in section \ref{sec:Well-foundedness}. 

Another deviation we have taken from \terese relates to the use of the definition of an 
increasing relation (see dfn:\ref{d:inc}). This property was one we found particularly difficult 
to make use of in a constructive setting (for example showing that it along with the properties 
of being WCR and WN implies SN) and so we suggest an alternative property, $\RP$.
(dfn:\ref{d:RP}). 

\begin{proposition}
    If $R$ is increasing, then no infinite $R$-increasing sequence is bounded.
\end{proposition}
\begin{proof}
    Let $(a_i) = a_0 \to a_1 \to a_2 \to \cdots$ be an $R$-increasing sequence
    and suppose that $(a_i)$ is bounded.  Let $b$ be the bound, so that $a_i \to b$ for all $i$.

    If $m = |b|$ then we have
    \[ 0 \le |a_0| < |a_1| < |a_2| < \cdots < |a_m| < |b| = m \]
    % \[ m > |a_m| > |a_{m-1}| > \cdots > |a_0| \ge 0 \]
    which is a contradiction, since there are only $m-1$ integers strictly less than $m$.
\end{proof}
It follows that every relation that is increasing in the sense of \terese
vacuously satisfies the RP condition.

\begin{corollary}
    If $R$ is increasing, then $R$ has the recurrence property.
\end{corollary}

We investigated $\RP$ and thought that with $\RPm$ we had a candidate for a weaker definition that 
did not rely on the notion of $\MF$. We were able to proove that this weaker definition is 
in fact an equivalent definition in the following proof in Agda: 

\verb|RP-↔RP : R is RP- ↔ R is RP| \footnote{ARS-implications.agda}

Three proofs are given in \terese of Newman's Lemma (Theorem 1.2.1). The first and third of 
these proofs we found to rely on classical principles. The second proof was amenablbe to a 
constructive formalization which we carried out in the function 


\verb|NewmansLemma : R isSN → R isWCR → R isCR| \footnote{ARS-NewmansLemma.agda}

We show in section \ref{sec:Implications} that $\SN$ implies $\SM$, and it is from this implication 
that we produce a new generalization of Newman's lemma. Every relation with the properties 
$\SM$ and $\WCR$ has the property of $\CR$. This holds true both locally and globally as we show in:

\verb|LocalNewmansLemmaRecurrent : R isWCR → SM ⊆ CR| \footnote{ARS-NewmansLemma.agda}

and \verb|GlobalNewmansLemmaRecurrent : R isWCR → R isSM → R isCR| \footnote{ARS-NewmansLemma.agda}

The formalization of the proof that $\WCR \land \WN \land \Inc \implies \SN$ proved particularly difficult. 
As already mentioned, we substituted out the property $\Inc$ for the property $\RP$ however 
we found that we still require a classical principle in order for the proof to hold (as can be seen in the type of our proof):
\verb|iii : R isWN → R isWCR → R isRP- → dec (SN) → R isSN| \footnote{ARS-Theorems.agda}
The classical property $\decSN$ is a property that states it is decidable whether a given 
element has the property of being $\SN$. \sacomment{Should say more on this proof.}

In the next section we take a closer look at the properties we found most crucial for termination and confluence, and the interrelationship between them.

