\section{Formalization of \terese}
\label{sec:Formalization}
This sections is a guide to our formalization of \terese $\,$ and explains the deviations we take from the text.

\subsection{Formalization of Definitions (\texttt{Base.agda})}\label{subsec:def}

\apcomment{Rephrase this?  Or rewrite this subsection.}

As outlined in the previous section, \texttt{Properties.agda}, \texttt{Seq.agda}, and \texttt{ClosureOperators.agda} define most standard termination and confluence properties. The file \texttt{Base.agda} introduces additional defitions which are required for the formalization of the propositions found in Chapter 1 of \terese.



\apcomment{Should articulate the fact that normalization is defined as well-foundedness of the \emph{converse}
relation, per \cite{Terese}.  Will discuss this separately in WF section, but should remind the reader of this
whenever SN/SM comes up. }

\apcomment{When SMseq comes up, also point to section 5, and the need for classical logic in going from
sequential versions to inductive ones.  IOW, elaborate the summary below. }

The following is a summary of the differences in definition we have used as mentioned in Section \ref{sec:Definitions}.
\begin{itemize}
    \item $\SN$ : We define this via the notion of accessibility rather than the non-existence of infinite reduction sequences.
     See Section \ref{sec:Well-foundedness} for further discussion.
    \item $\NPe$ : We use the equivalent $\NP$ definition.
    \item $\Inc$ : We use the more general property $\RP$.
\end{itemize}

As mentioned, the $\Inc$ property can impose a computationally demanding requirement. However, $\Inc$ is used only in Theorem 1.2.3
and the only crucial aspect of $\Inc$ needed for the proof is that no $R$-increasing
sequence can have an upper bound.
We establish that the equivalent properties $\RP$ and $\RPm$ are sufficient for replacing $\Inc$ in the theorem (see Subsection \ref{subsec:theorems}),
which led us to examine the condition on $s (i)$ appearing inside $\RP$ more closely. This condition provides a generalization of the notion of a normal form,
which we explore in detail in Section \ref{sec:Hierarchy}.

\subsection{\texttt{Propositions.agda}}
There are three sets of propositions given in \terese: \texttt{1.1.9}, \texttt{1.1.10}, \texttt{1.1.11}.
The formalization of the proofs are in \texttt{Propositions.agda}. These formalizations are straightforward
and as such are not the focus of this paper.

\subsection{\texttt{Theorems.agda}} \label{subsec:theorems}

\begin{theorem}(Newman's Lemma)
  $\gSN \land \gWCR \implies \gCR$
\end{theorem}

\begin{proof}
  There are three proofs of Newman's Lemma given in \terese.
  The first and third of these proofs make use of classical reasoning. The second proof is amenable to a
  constructive formalization as shown in the following function in \texttt{ARS-NewmansLemma.agda}:

  \verb|NewmansLemma : R isSN → R isWCR → R isCR|.

  We show in Subsection \ref{subsec:newnewman} a generalization of Newman's Lemma using a condition distilled from $\gRP$.
  The formalizations of the remaining theorems are all found in \texttt{ARS-Theorems.agda}.
\end{proof}

\begin{theorem}(Thm 1.2.2 in [Terese])
\begin{description}
  \item[($i$)] $\gCR \implies \gNPe \implies \gUN$
  \item[($ii$)] $\gWN \land \gUN \implies \gCR$
  \item[($iii$)] $\gCRs \implies \gCR$
\end{description}
\end{theorem}

\begin{proof}
    See, \\
    \verb|CR→NP : R isCR → R isNP₌|\\
    \verb|NP→UN : R isNP₌ → R isUN|\\
    \verb|ii : R isWN × R isUN → R isCR|\\
    \verb||\texttt{UN}$^{\to}$ \verb|∧WN→CR : R |\texttt{isUN}$^{\to}$ \verb| → R isWN → R isCR|\\
    \verb|iii : subcommutative R → R isCR|\\
    in \texttt{Theorems.agda}.
\end{proof}
\begin{proof}[Discussion.]
    In $\mathtt{UN^{→}∧WN→CR}$ a more general proof of ($ii$) is given using $\UNto$.
\end{proof}
\apcomment{Perhaps change the fourth implication to use $\times$ instead of $\to$ ??}

\begin{theorem}(Thm. 1.2.3 in [Terese])
  \begin{enumerate}[(i)]
    \item $\gWN \land \gUN \implies \gBP$
    \item $\gBP \land \gInc \implies \gSN$
    \item $\gWCR \land \gWN \land \gInc \implies \gSN$
    \item $\gCR \iff \gCP$ $($the forward implication only holds if the ARS is countable.$)$
  \end{enumerate}
  % \begin{description}
  %   \item[($i$)] $\gWN \land \gUN \implies \gBP$
  %   \item[($ii$)] $\gBP \land \gInc \implies \gSN$
  %   \item[($iii$)] $\gWCR \land \gWN \land \gInc \implies \gSN$
  %   \item[($iv$)] $\gCR \iff \gCP$ ($\gCR \implies \gCP$ only holds if the ARS is countable.)
  % \end{description}
\end{theorem}
\apcomment{Do the same for 1.2.2, and below!}

\begin{proof}
    See \texttt{Theorems.agda},\\
    \verb|i     : R isWN → R isUN → R isBP|\\
    $\mathtt{i^\to}$\hspace{5mm} \verb|: R isWN → R |\texttt{isUN}$^{\to}$ \verb|→ R isBP|\\
    \verb|iiSeq : R isWN → R isUN → R isRP → isWFseq- (~R R)|\\
    \verb|iii   : R isWN → R isWCR → R isRP- → dec SN → R isSN|\\
    \verb|iv    : R isCP → R isCR|
\end{proof}
\apcomment{Apply similar formatting elsewhere.}
\begin{proof}[Discussion.]\hfill
  \begin{description}
    \item[($i$)] In $\mathtt{i^\to}$ we again provide a slight improvement by using $\gUNto$ rather than $\gUN$.
    \item[($ii$)] This proof is inherently classical. In $\mathtt{iiSeq}$, we show that these properties imply $\isWFseqm$; establishing the implication to $\gSN$ would require additional classical assumptions. The definition and further discussion of $\isWFseqm$ are provided in Section~\ref{sec:Well-foundedness}.
    \item[($iii$)] In $\mathtt{iii}$, we replace $\gInc$ with the more general property $\gRPm$. We also assume that $\gSN$ is decidable, as this allows the construction of a witness for whether an element possesses the property $\SN$, which is essential for our proof.
    \item[($iv$)] We did not formalize $\gCR \to \gCP$ as it requires cardinality assumptions. \qedhere
  \end{description}
\end{proof}

% \begin{remark} \hfill
%     \begin{itemize}
%         \item In $\mathtt{i^\to}$ we again provide a slight improvement by using $\gUNto$ rather than $\gUN$.
%         \item ($ii$) is inherently classical. In $\mathtt{iiSeq}$, we show that these properties imply $\isWFseqm$; establishing the implication to $\gSN$ would require additional classical assumptions. The definition and further discussion of $\isWFseqm$ are provided in Section~\ref{sec:Well-foundedness}.
%         \item In $\mathtt{iii}$, we replace $\gInc$ with the more general property $\gRPm$. We also assume that $\gSN$ is decidable, as this allows the construction of a witness for whether an element possesses the property $\SN$, which is essential for our proof.
%         \item We did not formalize $\gCR \to \gCP$ as it requires cardinality assumptions.
%     \end{itemize}
% \end{remark}
