\section{Formalization of \terese}
\label{sec:Formalization}
This sections acts as a guide of our formalization of \terese and explains the deviations we took.  

\subsection{Formalization of Definitions}\label{subsec:def}
We have formalized those definitions which were necessary for the formalization of the propositions and theorems found 
in Chapter 1 of \terese. 
The closure operators are formalized in \texttt{ARS-Closure.agda}. 
The definitions grouped as confluent \terese are formalized in \texttt{ARS-Base.agda}. 
The definitions grouped as normalizing in \terese are formalized in \texttt{ARS-Properties.agda}.  

The following is a summary of the differences in definition we have used as mentioned in Section \ref{sec:Definitions}. 
\begin{itemize}
    \item $\SN$ : We make use of the well-founded defintion, rather than the finite reduction sequence definition.
    \item $\NPe$ : We use the equivalent $\NP$ definition. 
    \item $\Inc$ : We use the more general property $\RP$.      
\end{itemize}

As already mentioned, the $\Inc$ property was one we wanted to move away from. We saw that $\Inc$ was being used only in Theorem 1.2.3 
and that the crucial facet of $\Inc$ for the proof of the theorem was that every sequence has a bound \sacomment{is this the correct characterization?}. 
So we established that the properties in Definition \ref{def:rp} were sufficient for the proving the theorems. This in turn led us to examining 
a generalization of a normal form, the minimal form. We explore this further in Section \ref{sec:Implications}.

\subsection{Propositions}
There are three sets of propositions given in \terese: \texttt{1.1.9}, \texttt{1.1.10}, \texttt{1.1.11}. 
The formalization of the proofs are in \texttt{ARS-Propositions.agda}. These formalizations were straightforward 
and as such are not the focus of this paper. \sacomment{Do we include more information on these propositions?}

\subsection{Theorems}
\subsubsection{Newman's Lemma}
There are three proofs of Newman's Lemma ($\SN \land \WCR \implies \CR$) given in \terese. 
The first and third of these proofs make use of classical properties. The second proof is amenable to a 
constructive formalization as shown in the following function in \texttt{ARS-NewmansLemma.agda}:

\verb|NewmansLemma : R isSN → R isWCR → R isCR|.

\sacomment{This is a point where it may be useful to examine MF earlier. However, the alternative is to discuss the generalization of NL in the section on implications.}

We show in Section \ref{sec:Implications} a generalization of Newman's Lemma. \sacomment{Delete this comment when that generalization is written.}

The formalizations of the remaining theorems are all found in \texttt{ARS-Theorems.agda}. 
\subsubsection{Theorem 1.2.2}
There are three parts to this theorem. 
\begin{description}
    \item[($i$)] $\gCR \implies \gNPe \implies \gUN$
    \item[($ii$)] $\gWN \land \gUN \implies \gCR$  
    \item[($iii$)] $\gCRs \implies \gCR$
\end{description}

($i$) is formalized in:
\begin{itemize}
    \item \verb|CR→NP : R isCR → R isNP₌|
    \item \verb|NP→UN : R isNP₌ → R isUN|
    \item \verb|CP→UN : R isCR → R isUN|
\end{itemize}

($ii$) is formalized in:
\begin{itemize}
    \item \verb|ii : R isWN × R isUN → R isCR|
\end{itemize}
In our formalization we show that $\gWN \land \gUNto \implies \gCR$ is true and so provide a generalization of ($ii$) as $\gUN \implies \gUNto$. 

($iii$) is formalized in:
\begin{itemize}
    \item \verb|iii : subcommutative R → R isCR|
\end{itemize}

\subsubsection{Theorem 1.2.3}
There are four parts to this theorem. 
\begin{description}
    \item[($i$)] $\gWN \and \gUN \implies \gBP$
    \item[($ii$)] $\gBP \land \gInc \implies \gSN$  
    \item[($iii$)] $\gWCR \land \gWN \land \gInc \implies \gSN$
    \item[($iv$)] $\gCR \iff \gCR$ ($\implies$ only holds if the ARS is countable.)
\end{description}
($i$) is formalized in: 
\begin{itemize}
    \item \verb|i : R isWN → R isUN → R isBP|
    \item \verb|i→ : R isWN → R isUN→ → R isBP|
\end{itemize}
In $\mathtt{i\to}$ we again provide a generalization by using $\gUNto$ rather than $\gUN$.

A near formalization of ($ii$) is provided in: 
\begin{itemize}
    \item \verb|iiSeq : R isWN → R isUN → R isRP → isWFseq- (~R R)|
\end{itemize}
\sacomment{Explain here why have we provided a near formalization rather than sticking to the same properties}

($iii$) is formalized in: 
\begin{itemize}
    \item \verb|iii : R isWN → R isWCR → R isRP- → dec (SN) → R isSN|
\end{itemize}
As can be seen we have substituted $\gInc$ with $\gRPm$ which generalizes this theorem. We have also included the assumption that $\gSN$ 
is decidable \sacomment{Is decidability being discussed elsewhere in the paper that we can reference here.} as this property enables us 
to construct a witness for an element that either does or does not have the property $\SN$. Such a witness was necessary for our proof. \sacomment{More to say on this theorem?}


The reverse implication of ($iv$) is formalized in: 
\begin{itemize}
    \item \verb|iv : R isCP → R isCR|
\end{itemize}
We did not formalize $\gCR \to gCP$ as it requires the ARS to be countable \sacomment{More to say?}.


