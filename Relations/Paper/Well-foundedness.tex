\section{Well-foundedness}

\newcommand{\then}{\Longrightarrow}
\label{sec:Well-foundedness}

\newcommand{\accCor}{\textbf{accCor}}
\newcommand{\accDNE}{\textbf{accDNE}}

% We use Well-founded not Wellfounded

An abstract reduction $\to_R$ is strongly normalizing
if and only if the converse relation is well-founded.
In this section, we will compare several definitions of well-foundedness, which depend on the following properties.
\begin{definition}
  Let $R \subseteq A \times A$.  We write $Rxy$ when $(x,y) \in R$.
  \begin{enumerate}
    \item $P \subseteq A$ is \emph{inductive}
    if, for each $x \in A$, $(\forall y. Ryx \to y \in P)$ implies $x \in P$.

    \item $x \in A$ is \emph{accessible} if $x \in P$ for each inductive $P$.

    \item $x \in A$ is \emph{$P$-minimal} if $x \in P$ and for all $y$,
    $Ryx$ implies $y \notin P$.

    \item $s : \nat \to A$ is \emph{$R$-decreasing} if for all $k$, $(s(k+1),s(k)) \in R$.
  \end{enumerate}
\end{definition}
\apcomment{Add the definitions of well-foundedness here and then continue with the text as below.}
\subsection{New version: Synopsis and Outline}

Classically, the above notions of well-foundedness are all equivalent.  Constructively, only the first two are equivalent; the remaining definitions are not. Moreover, even individual classical definitions can be given constructive
interpretations in varying degrees of logical strength.

For example, the implication from ``every non-empty subset has a minimal element'' to
``every element is accessible''  is a classical proof by contradiction.
(Assuming that there exists an inaccessible element, the hypothesis implies that
there must be a minimal such element, which immediately yields a contradiction:
every minimal element is accessible.)

Constructively, this argument only goes so far as to show that, if every non-empty
subset has a minimal element, then no element is inaccessible.
That is, the universal statement being proved is that every $x \in A$
is not-not-accessible. The final step needed to infer that $x$ is indeed accessible
is not constructively provable in general, therefore it needs to be assumed as
an additional hypothesis in order to recover the full implication.

On the other hand, assuming accessibility is $\lnot\lnot$-closed, the other
hypothesis can now be weakened, to only require existence of minimal elements
for $\lnot\lnot$-closed subsets.  This is a significant restriction, which may
be validated in many cases of interests.  In contrast, the original hypothesis
is generally equivalent to every subset being decidable.
(The file WFCounters.agda shows that assuming the full minimality hypothesis
for the standard strict order on the natural numbers entails decidability
of every subset of $\nat$, clearly incompatible with effective semantics.)

One could also consider proving well-foundedness ``up to $\lnot\lnot$'',
asserting that every element is $\lnot\lnot$-accessible,
every sequence $\lnot\lnot$-contains a non-decreasing index,
for every non-empty subset there $\lnot\lnot$-exists a minimal element, etc.
This cuts down some, but not all, of the complexity.
For example, the two minimality principles
described above are equivalent with this formulation.
% The implication from the inductive to sequential definition no longer requires
% decidability of $R$.
% On the other hand, the reverse
At the same time, the
implication from sequential to inductive wellfoundedness
still requires an additional hypothesis, which may be described as the
constructive contrapositive of the definition of accessibility.
Specifically, we call {\accCor} the statement that, if an element $x$
is \emph{not} accessible, then there (strongly) exists a $y$ with $Ryx$
and $y$ not accessible.  (In the context of rewriting systems,
this condition asserts the existence of a function which maps any
non-strongly normalizing element to a one-step reduct with the same property.)
Assuming \accCor, all notions of well-foundedness
are constructively equivalent up to $\lnot\lnot$.

As this discussion illustrates, the network of relationships between the
constructive notions of well-foundedness is quite rich.  The purpose of
this section is to present the basic picture of these implications,
and elaborate on some of them.  Our key observations are summarized
in the following list.
\begin{enumerate}
  \item The implication from the inductive to the sequential definition
  requires the underlying relation to be decidable. This assumption is no longer
  necessary on the $\lnot\lnot$-translated side.
  \item WFmin is a very strong assumption and implies WFseq unconditionally.
  The more reasonable WFminDNE requires Markov's Principle
  (and decidability of $R$) to yield the same consequence.
  \item On the $\lnot\lnot$-side, WFmin- and WFminDNE- are equivalent,
  and imply WFacc- via WFseq-, requiring {\accCor} assumption in the final inference.
  \item At the same time, if the relation is finitely branching, then WFminDNE-
  implies WFacc- directly.
  \item Finite branchingness together with (weak) decidability of minimality
  implies \accCor.
  % \apcomment{The previous observation suggests the question of whether isWFseq- might
  % imply isWFminDNE- via decidability of minimal elements.\\
  % However, in trying to prove this implication, the hypothesis asserting non-existence of
  %  a minimal element of some given non-empty $\lnot\lnot$-closed predicate
  %  automatically entails that any particular element of interest is not normal.\\
  %  Unless it's not satisfying $P$? \emph{We need to decide whether it is P-minimal as well.}}
  \item Assuming \accCor, isWFacc- can also be derived from a weaker condition, isWFminCor(-).
  (Just like isWFminDNE is a natural weakening of isWFmin that is sufficient to recover isWFacc
  in the presence of {\accDNE}.)
  It is implied by isWFseq-, but does not appear to imply anything on its own.

  \apcomment{(This seems easy: The converse implication isWFminCor to isWFseq- requires the sequence in question to
  have decidable image, leading to a further weakening of isWFseq-.)}
  \item Since isWFminDNE- implies isWFseq-, the former also implies isWFminCor-.
  But the converse fails: If a subset/predicate $P$ is $\lnot\lnot$-closed,
  we cannot conclude it is coreductive.

  \item In general, the coreductive properties speak to the ability to choose
  a successor (strong existence).  This ability is central to being able to
  define sequences.  Therefore, sequential definitions generally require
  some coreductive situation to be applicable.

  \item Assuming normal forms are decidable,
  isWFseq implies a very weak version of minimality, isWFminEM.
  On its own this property is insufficient to imply anything else,
  so we have nothing else to say about it.
\end{enumerate}



In \cite{Terese}, well-foundedness is defined as in the classical mathematics textbooks,
as the absence of an infinite $R$-decreasing sequence.
(Again, well-foundedness of $R$ corresponds to strong normalization of its converse.)
Translating this statement directly into type theory yields the formula

\[\tag{$\isWFseqm$} \lnot \big(\exists s : \nat \to A \forall k : \nat. \  R \,(s\,(k+1)) \, (s\, k) \big) \]

(The label $\isWFseqm$ refers to the name of the corresponding term in \texttt{Wellfoundedness.agda}.)
Constructively, this is not a useful hypothesis, as it does not provide any positive computational content and requires an explicit sequence to be defined.

A stronger formulation asserts that, for any sequence, there exists an element
in the sequence that does not reduce to the next one.  This yields the statement
\[\tag{$\isWFseq$} \forall s : \nat \to A \exists k : \nat.\, \lnot \, R \,(s\,k)\,(s\,(k+1)) \]

These two conditions are classically equivalent, but not constructively so.
Indeed, $\isWFseq$ certainly implies $\isWFseqm$, but the converse only holds classically.

Another classical formulation of well-foundedness
asserts that every non-empty subset
contains a minimal element.  That is,
\[ \tag{$\isWFmin$} \forall P \subseteq A. P \neq \emptyset \to \exists x \in P. (\forall y. Ryx \to y\notin P) \]

While classically equivalent to the previous one, see \cite[Exercise A.1.8]{Terese},
constructively we only have the implication $\isWFmin \then \isWFseq$.

The canonical constructive definition of well-foundedness is that
every inductive predicate is universally true.
This is constructively equivalent to the statement that every element
is accessible.  In Agda, these concepts are encoded as follows.
{\small
\begin{verbatim}
  _-inductive_ : 𝓟 A → Set
  _-inductive_ φ = ∀ x → (∀ y → R y x → φ y) → φ x

  _isWFind : Set₁
  _isWFind = ∀ (φ : 𝓟 A) → _-inductive_ φ → ∀ x → φ x

  data _-accessible : 𝓟 A where
    acc : ∀ {x : A} → (∀ y → R y x → _-accessible y) → _-accessible x

  _isWFacc : Set
  _isWFacc = ∀ (x : A) → x ∈ _-accessible
\end{verbatim}
}
Although the two notions are equivalent, the latter definition allows us to do induction on the proof of accessibility of an individual element $x \in A$.  This makes such proofs easier to write in Agda, thus we take
\verb|_isWFacc| as \emph{the} standard notion of well-foundedness for relations.

\subsection{Relations between the main definitions}

We have now seen five notions of a well-founded relation, only two of which
are seen to be logically equivalent.

This raises several questions.
\begin{itemize}
\item Are all these notions truly distinct?  If one cannot furnish a proof of implication,
can we provide a counterexample?
\item Under what hypotheses do the implications become provable?
\item How ``admissible'' are such hypotheses, and how often can we hope for them to be valid?
\end{itemize}

The situation of separating these notions is complicated by the fact that, for all
\emph{definable} predicates, the classical and constructive notions do coincide.
(We have not proved this, but see \cite{Berardi} for a related result.)
So there is no hope to discriminate these notions via explicit counterexamples.
Rather, we could attempt to show that one does not imply the other by proving that
such an implication would yield a ``constructive taboo'': a proposition known to be
unprovable without additional classical assumptions.

For example, the archetype of a well-founded relation is the strict order \verb|<|
of natural numbers.  The usual induction readily validates $\isWFacc$ for \verb|<|.
  At the same time, the file \texttt{WFCounters.agda} shows that
asserting $\isWFmin$ for \verb|<| implies that every subset of the natural numbers is decidable.  Since this is clearly not provable in Agda without classical axioms,
this argument shows that the implication from $\isWFacc$ to $\isWFmin$ is not provable.

Such distinctions have consequences for rewriting theory.
For example, $x\in \SN_R$ iff $x$ is $\bar{R}$-accessible (where $\bar{R}$ denotes
the converse of $R$).  Thus, if $\bar{R}$ is well-founded, $R \models \gSN$.
Yet actually computing the normal form of a given $x$ seems to require a
strong form of decidability of being $R$-minimal, namely:
\[
\tag{$\isMinDec$} \forall x. \left(\exists y. Ryx\right) \lor \left(\forall y. \lnot Ryx\right)
\]
(Note that asking whether $x$ is a normal form constitutes the weaker disjunction
$\lnot\lnot\left(\exists y. Ryx\right) \lor \left(\forall y. \lnot Ryx\right)$.)

On the other hand, $\isWFmin$ trivially implies that each $x \in A$ is weakly normalizing.
Indeed, given $x \in A$, consider the predicate $P_x(y) = x \mstep y$.
This predicate is non-empty, since the empty reduction yields a proof of $P_x(x)$.
Let $n$ be a $P_x$-minimal element.  Then $x \mstep n$.  Moreover, for any $y$,
If $n \rstep y$, then $\lnot(P_x(y))$.  But since reductions from $x$ are closed downward,
no such $y$ can exist, and hence $n$ is a normal form of $x$.

% \greyout{From the perspective of rewriting theory, it would seem wild indeed if
% $\SN$ did not imply $\WN$.
%
These questions motivated us to take a closer look at the issue of well-foundedness.
Our analysis revealed a number of additional related hypotheses, having a complex
network of relations.  %These are displayed in Figure \ref{f:wf}.
These are recorded in the file \verb|Wellfounded.agda|
