\section{Well-foundedness}

\newcommand{\isWFacc}{\mathtt{isWFacc}}
\newcommand{\isWFseq}{\mathtt{isWFseq}}
\newcommand{\isWFmin}{\mathtt{isWFmin}}
\newcommand{\isWFaccm}{\mathtt{isWFacc-}}
\newcommand{\isWFseqm}{\mathtt{isWFseq-}}
\newcommand{\isWFminm}{\mathtt{isWFmin-}}
\newcommand{\then}{\Longrightarrow}
\label{sec:Well-foundedness}

An abstract reduction $\to_R$ is strongly normalizing
if and only if the converse relation is well-founded.
In this section, we will compare several definitions of well-foundedness.
They will be based on the following notions.

\begin{definition}
  Let $R \subseteq A \times A$.
  \begin{enumerate}
    \item $P \subseteq A$ is \emph{inductive}
    if, for each $x \in A$, $(\forall y. Ryx \to y \in P)$ implies $x \in P$.

    \item $x \in A$ is \emph{accessible} if $x \in P$ for each inductive $P$.

    \item $x \in A$ is \emph{$P$-minimal} if $x \in P$ and for all $y$,
    $Ryx$ implies $y \notin P$.

    \item $s : \nat \to A$ is \emph{$R$-decreasing} if for all $k$, $(s(k+1),s(n)) \in R$.
  \end{enumerate}
\end{definition}

In TeReSe, well-foundedness is defined in the classical way,
as the absence of an infinite $R$-decreasing sequence.
Translating this directly into type theory yields the condition
\[\isWFseqm : \lnot \big(\exists s : \nat \to A \forall k : \nat. (s(k+1),s(k)) \in R \big) \]

Constructively, this is not a useful hypothesis, as it does not provide any positive computational content and requires an explicit sequence to be given.

A stronger formulation asserts that, for any sequence, there exists an element
in the sequence that does not reduce to the next one.  This yields
\[\isWFseq : \forall s : \nat \to A \exists k : \nat. (s(k),s(k+1)) \notin R \]

These are classically equivalent, but not constructively so.
Indeed, we have $\isWFseq$ implying $\isWFseqm$, but not vice versa.

Another classical formulation of well-foundedness
asserts that every non-empty subset
contains a minimal element.  That is,
\[ \isWFmin : \forall P \subseteq A. P \neq \emptyset \to \exists x \in P. (\forall y. Ryx \to y\notin P) \]

While classically equivalent to the above, see  \cite{Terese, Exercise 1.11?},
constructively we only have the implication $\isWFmin \then \isWFseq$.

The canonical constructive definition is that
every inductive predicate is universally true.
This is constructively equivalent to the statement that every element
is accessible.  In Agda, these concepts are encoded as follows.
{\small
\begin{verbatim}
  data _-accessible : 𝓟 A where
    acc : ∀ {x : A} → (∀ y → R y x → _-accessible y) → _-accessible x

  _isWFacc : Set
  _isWFacc = ∀ (x : A) → x ∈ _-accessible

  _-inductive_ : 𝓟 A → Set
  _-inductive_ φ = ∀ x → (∀ y → R y x → φ y) → φ x

  _isWFind : Set₁
  _isWFind = ∀ (φ : 𝓟 A) → _-inductive_ φ → ∀ x → φ x
\end{verbatim}
}

\greyout{ SN does not imply WN, we can now discuss this in detail.
Notice that $x \in WFmin$ certainly implies WN, because we can ask for
the minimal element of the predicate $P(y) = x \mstep y$
(the reduction graph of $x$).  Note that this subset is nonempty because $P(x)$ holds.
}

We have now seen five notions of a well-founded relations, only two of which
appear to be logically equivalent.

This raises several questions.
\begin{itemize}
\item Are all these notions truly distinct?  If one cannot furnish a proof of implication,
can we provide a counterexample?
\item Under what hypotheses do the implications become provable?
\item How ``admissible'' are these hypotheses, and how often can we hope for them to be valid?
\end{itemize}

The situation of constructing counterexamples is complicated by the fact that, for all
\emph{definable} predicates, the notions do coincide.  (We have not formalized this,
but see for example \cite{Berardi}, $A$-translation.)
So there is little hope to separate these notions with explicit counterexamples.
Rather, we could attempt to show that one does not imply the other by proving that
such an implication would yield a ``constructive taboo'': a proposition known to be
unprovable without addtional classical assumptions.

These questions motivated us to take a closer look at the issue of well-foundedness.
Our analysis revealed a number of additional related hypotheses, having a complex
network of relations.  These are displayed in Figure \ref{f:wf}.

Let us elaborate each of these further.

First, we noticed that, analogously to $\isWFseq$ and $\isWFseqm$, there are weaker notions of
$\isWFacc$ and $\isWFmin$.  These are
\begin{align*}
\isWFaccm &= \cdots \\
\isWFminm &= \cdots \\
\end{align*}

Each notion of those listed trivially implies the weaker one.

Moreover, we can show that \apcomment{add the ``easy'' relationships from the old report.}
These elementary implications are summarized in the following diagram.

\apcomment{insert the diagram with X implying X- downward, and chains horizontally,
as X ranges of isWFacc, isWFind, isWFseq, isWFmin}

Since $\isWFmin$ appears to be stronger than the rest, one may wonder whether
it can be weakened.  Indeed, in the classical proof of the implication
$\isWFacc \to \isWFmin$, the predicate $P$ is instantiated by the \emph{complement}
of a given one.

This yields the following two :
  \begin{align*}
    isWFminDNE
    isWFmiEM
  \end{align*}

Surprisingly, although $\isWFmin$ seems very strong, it is not sufficient to
conclude $\isWFacc$.  The culprit is in the conclusion, that the predicate becomes
negated twice.

This suggests an alternative restriction, where one directly works with the
complement of a given predicate.  This leads us to two more variations.
\begin{align*}
  isWFmin+\\
  isWFminRed+
\end{align*}

Of course, each of these also has weaker versions,
with $\lnot\lnot$ in front of the existential quantifier.

\apcomment{Should collect all these definitions in one table.
Display this table next to the diagram showing all of the established implications.
Perhaps do the same in section 4 as well...     }

\subsection{Other things to say}
\begin{itemize}
  \item $\isWFmin$ splits into the restrictions on the predicate and the complement
  \item The distinction between $\isWFseq$ and $\isWFmin$ concerns
  the non-determinism/choice/coreductivity of the predicate.
  With sequences, the path is deterministic and we don't have to
  search for a reduct where the predicate retains its truth value.

  But in case of minimality, it is not known how to find the successor
  where the truth condition is preserved.

  One remedy is to ask for this choice function directly --- this yields
  the coreductivity hypothesis.

  The other remedy is finite branching.

  \item Generally, iswFMin EM will imply isWFseq.  This should be easy to prove.

  \greyout{It should also imply/be implied by min+cor, maybe }




















\end{itemize}
