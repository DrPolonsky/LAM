\section{Well-foundedness}

\newcommand{\isWFacc}{\mathtt{isWFacc}}
\newcommand{\isWFseq}{\mathtt{isWFseq}}
\newcommand{\isWFmin}{\mathtt{isWFmin}}
\newcommand{\isWFaccm}{\mathtt{isWFacc-}}
\newcommand{\isWFseqm}{\mathtt{isWFseq-}}
\newcommand{\isWFminm}{\mathtt{isWFmin-}}

\newcommand{\isMinDec}{\mathtt{isMinDec}}

\newcommand{\then}{\Longrightarrow}
\label{sec:Well-foundedness}

An abstract reduction $\to_R$ is strongly normalizing
if and only if the converse relation is well-founded.
In this section, we will compare several definitions of well-foundedness.

\subsection{The main definitions}

The various formulations of well-foundedness are based on the following notions.
\begin{definition}
  Let $R \subseteq A \times A$.  We write $Rxy$ when $(x,y) \in R$.
  \begin{enumerate}
    \item $P \subseteq A$ is \emph{inductive}
    if, for each $x \in A$, $(\forall y. Ryx \to y \in P)$ implies $x \in P$.

    \item $x \in A$ is \emph{accessible} if $x \in P$ for each inductive $P$.

    \item $x \in A$ is \emph{$P$-minimal} if $x \in P$ and for all $y$,
    $Ryx$ implies $y \notin P$.

    \item $s : \nat \to A$ is \emph{$R$-decreasing} if for all $k$, $(s(k+1),s(k)) \in R$.
  \end{enumerate}
\end{definition}

In \cite{Terese}, well-foundedness is defined as in the classical mathematics textbooks,
as the absence of an infinite $R$-decreasing sequence.
Translating this statement directly into type theory yields the formula
\[\tag{$\isWFseqm$} \lnot \big(\exists s : \nat \to A \forall k : \nat. (s(k+1),s(k)) \in R \big) \]
(The label $\isWFseqm$ refers to the name of the corresponding term in \texttt{Wellfoundedness.agda}.)
Constructively, this is not a useful hypothesis, as it does not provide any positive computational content and requires an explicit sequence to be defined.

A stronger formulation asserts that, for any sequence, there exists an element
in the sequence that does not reduce to the next one.  This yields the statement
\[\tag{$\isWFseq$} \forall s : \nat \to A \exists k : \nat. (s(k),s(k+1)) \notin R \]

These two conditions are classically equivalent, but not constructively so.
Indeed, $\isWFseq$ certainly implies $\isWFseqm$, but the converse only holds classically.

Another classical formulation of well-foundedness
asserts that every non-empty subset
contains a minimal element.  That is,
\[ \tag{$\isWFmin$} \forall P \subseteq A. P \neq \emptyset \to \exists x \in P. (\forall y. Ryx \to y\notin P) \]

While classically equivalent to the previous one, see \cite[Exercise A.1.8]{Terese},
constructively we only have the implication $\isWFmin \then \isWFseq$.

The canonical constructive definition of well-foundedness is that
every inductive predicate is universally true.
This is constructively equivalent to the statement that every element
is accessible.  In Agda, these concepts are encoded as follows.
{\small
\begin{verbatim}
  _-inductive_ : 𝓟 A → Set
  _-inductive_ φ = ∀ x → (∀ y → R y x → φ y) → φ x

  _isWFind : Set₁
  _isWFind = ∀ (φ : 𝓟 A) → _-inductive_ φ → ∀ x → φ x

  data _-accessible : 𝓟 A where
    acc : ∀ {x : A} → (∀ y → R y x → _-accessible y) → _-accessible x

  _isWFacc : Set
  _isWFacc = ∀ (x : A) → x ∈ _-accessible
\end{verbatim}
}
Although the two notions are equivalent, the latter definition allows us to do induction on the proof of accessibility of an individual element $x \in A$.  This makes such proofs easier to write in Agda, thus we take
\verb|_isWFacc| as \emph{the} standard notion of well-foundedness for relations.

\subsection{Relations between the main definitions}

We have now seen five notions of a well-founded relation, only two of which
are seen to be logically equivalent.

This raises several questions.
\begin{itemize}
\item Are all these notions truly distinct?  If one cannot furnish a proof of implication,
can we provide a counterexample?
\item Under what hypotheses do the implications become provable?
\item How ``admissible'' are such hypotheses, and how often can we hope for them to be valid?
\end{itemize}

The situation of separating these notions is complicated by the fact that, for all
\emph{definable} predicates, the classical and constructive notions do coincide.
(We have not proved this, but see \cite{Berardi} for a related result.)
So there is no hope to discriminate these notions via explicit counterexamples.
Rather, we could attempt to show that one does not imply the other by proving that
such an implication would yield a ``constructive taboo'': a proposition known to be
unprovable without additional classical assumptions.

For example, the archetype of a well-founded relation is the strict order \verb|<|
of natural numbers.  The usual induction readily validates $\isWFacc$ for \verb|<|.
  At the same time, the file \texttt{WFCounters.agda} shows that
asserting $\isWFmin$ for \verb|<| implies that every subset of the natural numbers is decidable.  Since this is clearly not provable in Agda without classical axioms,
this argument shows that the implication from $\isWFacc$ to $\isWFmin$ is not provable.

Such distinctions have consequences for rewriting theory.
For example, $x\in \SN_R$ iff $x$ is $\bar{R}$-accessible (where $\bar{R}$ denotes
the converse of $R$).  Thus, if $\bar{R}$ is well-founded, $R \models \gSN$.
Yet actually computing the normal form of a given $x$ seems to require a
strong form of decidability of being $R$-minimal, namely:
\[
\tag{$\isMinDec$} \forall x. \left(\exists y. Ryx\right) \lor \left(\forall y. \lnot Ryx\right)
\]
(Note that asking whether $x$ is a normal form constitutes the weaker disjunction
$\lnot\lnot\left(\exists y. Ryx\right) \lor \left(\forall y. \lnot Ryx\right)$.)

On the other hand, $\isWFmin$ trivially implies that each $x \in A$ is weakly normalizing.
Indeed, given $x \in A$, consider the predicate $P_x(y) = x \mstep y$.
This predicate is non-empty, since the empty reduction yields a proof of $P_x(x)$.
Let $n$ be a $P_x$-minimal element.  Then $x \mstep n$.  Moreover, for any $y$,
If $x \rstep y$, then $\lnot(P_x)$.  But since reductions from $x$ are closed downward,
no such $y$ can exist, and hence $n$ is a normal form of $x$.

% \greyout{From the perspective of rewriting theory, it would seem wild indeed if
% $\SN$ did not imply $\WN$.
%
These questions motivated us to take a closer look at the issue of well-foundedness.
Our analysis revealed a number of additional related hypotheses, having a complex
network of relations.  %These are displayed in Figure \ref{f:wf}.
These are recorded in the file \verb|Wellfounded.agda|
