\documentclass{article}
\usepackage{hyperref}
% \usepackage{tikz}
\usepackage{amsmath}
\usepackage{amsthm}
\usepackage{pifont}
\usepackage{mathrsfs}
\usepackage{amssymb}
\usepackage{xcolor}
\usepackage{colortbl}
\usepackage{tikz-cd}
\usepackage{caption}
\usepackage{newunicodechar}


\usepackage[utf8]{inputenc}
\usepackage{ucs}
% \DeclareUnicodeCharacter{03A3}{\ensuremath{\Sigma}}



\usetikzlibrary{positioning,shapes.geometric,fit,arrows.meta}


\captionsetup{justification=centering}



\definecolor{darkgreen}{rgb}{0.0, 0.5, 0.0}

\newcommand{\tto}{\twoheadrightarrow}
\newcommand{\sse}{\subseteq}
\newcommand{\bset}{\mathbf{Set}}
\newcommand{\nat}{\mathbb{N}}

% Comments
\newcommand{\sacomment}[1]{\textcolor{green}{#1}}
\newcommand{\apcomment}[1]{\textcolor{blue}{#1}}
\newcommand{\greyout}[1]{\textcolor{gray}{#1}}
\newcommand{\err}[1]{\textcolor{red}{#1}}

% Theorem style
\newtheorem{notation}[theorem]{Notation}
% \newtheorem{thm}{Theorem}
% \newtheorem{dfn}[thm]{Definition}
% \newtheorem{prop}[thm]{Proposition}
% \newtheorem{cor}[thm]{Corollary}
% % \newtheorem{lemma}[thm]{Lemma}
% % \newtheorem{rmk}[thm]{Remark}
% % \newtheorem{expl}[thm]{Example}
% \newtheorem{notn}[thm]{Notation}
% %\theoremstyle{nonumberplain}
% %\theoremsymbol{\Box}
% % \newtheorem{proof}{Proof}

\newcommand{\RP}{\mathrm{RP}}
\newcommand{\gRP}{\mathbf{RP}}
\newcommand{\RPm}{\mathrm{RP{-}}}
\newcommand{\gRPm}{\mathbf{RP{-}}}
\newcommand{\NF}{\mathrm{NF}}
\newcommand{\MF}{\mathrm{MF}}
\newcommand{\UN}{\mathrm{UN}}
\newcommand{\gUN}{\mathbf{UN}}
\newcommand{\UNto}{\mathrm{UN}^{\to}}
\newcommand{\gUNto}{\mathbf{UN}^{\to}}
\newcommand{\SN}{\mathrm{SN}}
\newcommand{\gSN}{\mathbf{SN}}
\newcommand{\decSN}{\mathrm{dec(SN)}}
\newcommand{\SM}{\mathrm{SM}}
\newcommand{\SMseq}{\mathrm{SMseq}}
\newcommand{\gSM}{\mathbf{SM}}
\newcommand{\WN}{\mathrm{WN}}
\newcommand{\gWN}{\mathbf{WN}}
\newcommand{\SMandWN}{\mathrm{SM\land WN}}
\newcommand{\gSMandWN}{\mathbf{SM\land WN}}
\newcommand{\WM}{\mathrm{WM}}
\newcommand{\gWM}{\mathbf{WM}}
\newcommand{\WNFP}{\mathrm{WNFP}}
\newcommand{\NP}{\mathrm{NP}}
\newcommand{\gNP}{\mathbf{NP}}
\newcommand{\NPe}{\mathrm{NP_=}}
\newcommand{\gNPe}{\mathbf{NP_=}}
\newcommand{\WMFP}{\mathrm{WMFP}}
\newcommand{\MP}{\mathrm{MP}}
\newcommand{\gMP}{\mathbf{MP}}
\newcommand{\CR}{\mathrm{CR}}
\newcommand{\CRs}{\mathrm{CR^{\le 1}}}
\newcommand{\gCRs}{\mathbf{CR^{\le 1}}}
\newcommand{\gCR}{\mathbf{CR}}
\newcommand{\WCR}{\mathrm{WCR}}
\newcommand{\gWCR}{\mathbf{WCR}}
\newcommand{\Inc}{\mathrm{Inc}}
\newcommand{\gInc}{\mathbf{Inc}}
\newcommand{\BP}{\mathrm{BP}}
\newcommand{\gBP}{\mathbf{BP}}
\newcommand{\FB}{\mathrm{FB}}
\newcommand{\CP}{\mathrm{CP}}
\newcommand{\gCP}{\mathbf{CP}}



\newcommand{\from}{\leftarrow}



\newcommand{\red}[1]{\textcolor{red}{#1}}
\newcommand{\blue}[1]{\textcolor{blue}{#1}}

% Reduction relation macros
\newcommand{\rstep}{\mathbin{\longrightarrow_R}}
\newcommand{\mstep}{\mathbin{\longrightarrow_R^*}}
\newcommand{\estep}{\mathbin{\longrightarrow_R^=}}
\newcommand{\rrstep}{\mathbin{\longrightarrow_R^r}}
\newcommand{\brstep}{\mathbin{\longleftarrow_R^r}}
\newcommand{\bstep}{\mathbin{\longleftarrow_R}}
\newcommand{\bmstep}{\mathbin{\longleftarrow_R^*}}

% Unicode characters

\newunicodechar{∀}{\ensuremath{\forall}}
\newunicodechar{→}{\ensuremath{\rightarrow}}
\newunicodechar{ℕ}{\ensuremath{\mathbb{N}}}
\newunicodechar{↔}{\ensuremath{\leftrightarrow}}
\newunicodechar{⊆}{\ensuremath{\sse}}
\newunicodechar{∧}{\ensuremath{\land}}
\newunicodechar{₌}{\ensuremath{_=}}
\newunicodechar{𝓟}{\ensuremath{\mathcal{P}}}
\newunicodechar{∈}{\ensuremath{\in}}
\newunicodechar{φ}{\ensuremath{\phi}}
\newunicodechar{Σ}{\ensuremath{\Sigma}}
\newunicodechar{₁}{\ensuremath{{}_1}}


% Misc Macros
\newcommand{\terese}{[TeReSe]}
% \newcommand{\ul}[1]{\underline{#1}}
\newcommand{\ule}[1]{\underline{#1:}}
% \newcommand{\setof}[1]{\{#1\}}
\newcommand{\setof}[1]{\left\{#1\right\}}


\newcommand{\tclos}[1]{{#1}^{\scriptscriptstyle{+}}}

\usepackage{enumitem,amssymb}
\newlist{todolist}{itemize}{2}
\setlist[todolist]{label=$\square$}
\usepackage{pifont}
\newcommand{\cmark}{\ding{51}}%
\newcommand{\xmark}{\ding{55}}%
\newcommand{\done}{\rlap{$\square$}{\raisebox{2pt}{\large\hspace{1pt}\cmark}}%
\hspace{-2.5pt}}
\newcommand{\wontfix}{\rlap{$\square$}{\large\hspace{1pt}\xmark}}

\begin{document}
\title{Summer 2025 ToDo}
\maketitle

\section*{September 2025}
\subsection*{Notes from 7th September}
Thoughts on the current state of the paper.
\begin{description}
\item[The status of wfCor]
  \begin{itemize}
    \item 
      On the ``missing'' implication isWFCor$\lnot\lnot$ to isWFminDNE$\lnot\lnot$.

      The wfCor hypothesis needs a ``choice function'' that will pick a successor $y$ of a given element $x$
      (while still refuting the given predicate). 

      The property wfMinDNE$\lnot\lnot$ is equivalent to the statement that, if a $\lnot\lnot$-closed 
      predicate $P$ has no minimal element, then its complement is universally true.

      The ``no minimal element'' for not-not-closed $P$ is close to saying that its complement is coreductive; 
      the difference is precisely that between the ``strong'' existential in the coreductive statement ---
      and the weak one (with $\lnot\lnot$ applied to it) in the minimality one.

      Assuming finite branchingness and weak decidability of $P$, we could get the strong existential 
      from the weak one.  However, weak decidability and DNE are already equivalent with EM, so that 
      will only apply to decidable predicates.  There may be a relationship to isWFminEM, but that 
      would be a question for another paper.

    \item 
      On the ``strong side'' of the WF diagram, the situation is even worse.
      Neither implication between isWFCor and isWFminDNE is provable, and similarly for isWFCor and isWFSeq. 

    \item 
      At this point, there does not appear to be much more to say isWFCor, so we should consider removing it.

      On Monday, we should quickly discuss the different versions of sequentially,
      and move on to writing the text of the paper.
    \end{itemize}
\end{description} 

\section*{Week starting 21st April}
\begin{itemize}
  \item Items from review

  \begin{todolist}
    \item (Review 1) Elaborate on our motivations in the introduction (reviewer wasn't sure how relevant this is to ARS theory itself/ sees it as too minor a contribution.)
    \item [\done] (Review 1) Typo: incaranations $\to$ incarnations.
    \item [\done] Improve fonts for constant and text (refering to a definition on line 258). General formatting imrpovement for how we set out defns.
    \item [\done] (Review 1) Improve the layout on line 270 where we set out iiSeq. This should get fixed with an overall improvement in how we set out definitions. We should stop using itemize and verb combination.
    \item [\done] (Review 1) Typo: we obtaining $\to$ we obtain (this is in the conclusion which we want to rewrite entirely).
    \item [\done] (Review 1) Citation expansion for program=proof.
    \item (Review 2) Unconvinced by the motivations for the project, believes the same can be achieved in a classical framework. So maybe we should explain our motivations more clearly.
    \item [\done] Not convinced by our definitions, as they are imprecise and lack quantifiers.
    \item [\done] Issue with definition of R|=RP on line 188. Use of bounded is ambiguous. We should clarify which use of bounded.
    \item [\done] Improve proposition 11 on line 194, we want to make the clame that RP- $\to$ RP, and either have a one line proof or refer to the proof in Agda.
    \item [\done] Section 3 is "full of repetitions and paraphrases between the statements in natural language and corresponding formalization". We can try and think of a way to get it to flow better, we do want to provide natural language summaries of what is in the code!
    \item [\done] (Review 2) Struggled to understand the implications tables. We should add a reminder that bold font is global. We should rethink the tables generally to make them clearer. \apcomment{Don't comma seperate and don't denote counterexamples as CE-XX. ?}
    \item [\done] Struggled to find the relevant agda code for the various definitions. We should find a way to better link the two.
    \item [\done] Clarify Notation 7 on line 159: "Choose a uniform notation, X or NP, but not both".
    \item [\done] l 169: An annoying notational collision between the two arrows -> at different heights."I'm not too sure what they're referring to, though there could be some ambiguity with the arrows here, especially with use of $UN\to$.
    \item [\done] (Review 2) "l 361: Swap s(k+1) and s(k) " We should clarify the definition here otherwise people (evidently) get confused.
    \item [\done](Review 2) "l 422: $P_x$ has no arguments?"
    \item (Review 3) Check out and include the following related work:
    \begin{itemize}
        \item Christian Sternagel and René Thieman formalized results on Abstract Rewriting in Isabelle https://www.isa-afp.org/entries/Abstract-Rewriting.html
        \item André Galdino and Mauricio Ayala-Rincón also developed a theory in PVS regarding ARS and TRS  https://github.com/nasa/pvslib/tree/master/TRS
        \item What are the differences and improvements of the formalization in this paper and the ones in the literature?
        \item (Review 2) TeReSe is not state of the art. We need state of the art references. Try this: https://termination-portal.org/wiki/WScT
    \end{itemize}
    \item (Review 3) * From the formalization, the authors could explain some features and syntax of Agda for readers unfamiliar with this proof assistant. What were the main attributes of Agda that helped you in the formalization process? In line 204, the paper brings the definition of an element being finitely branching and states that the formulation of such a definition is given in line 206. FB a seems to be a set (in line 206), not a boolean. Also, in the formalization of FB a, the authors use R b a instead of R a b. What elements of Agda allow such specification capturing exactly Def. 12.
    \item [\done](Review 3) In line 270, isWFseq- is used in Section 3 but defined only in Section 5.
    \item [\done] (Review 3) In line 192, the authors state that (in TeReSe) Ind denotes BP. However, in line 186, BP is defined from an R-increasing sequence, whereas this statement is not required in the definition of Ind by TeReSe. How does this difference impact the formalizations and generalizations pointed out in Section 3.3.3? (Maybe we should explicitly address all difference with TeReSe in one location.)
  \end{todolist}

  \item Other ToDo
  \begin{todolist}
    \item [\done] Make Latex version of WF implications (initial draft).
    \item [\done] Verify which implications have and have not been formalized from the image.
    \item [\done] $\lnot \lnot$ WFacc $\leftrightarrow$ $\lnot \lnot WFind$.
    \apcomment{WFind to be removed, this is addressed in the Friday June 13th notes}
    \item [\done] (Missing imp) WFmin- $\to$ WFDNE-
    \item [\done] (Missing imp) All of the implicaions from the standard WF definition to the $\lnot \lnot$ definition.
    \item [\done] Sam TODO: I still don't have a coherent understanding in my head as to how we relate the WF and ARS sections of the paper. I need to go through it and either understand it or come with questions to Friday's meeting.
  \end{todolist}
\end{itemize}

\section*{Week starting April 28th}
\begin{itemize}
  \item Below notes from meeting on April 25th.
  \begin{todolist}
    \item [\done] Fix Wellfounded.agda after changes made during meeting.
    \item [\done] Think about how to organize the Git repo going forward. For example, could be good to have a parent folder with ARS and WF, with a private dir for paper
    
    
    
  \end{todolist}

\end{itemize}
\section*{Week starting 26th May}
\begin{todolist}
  \item [\done] Double check the counterexamples match in the table to the diagrams.
  \item [\done] Make clear in proof references to agda code how it should be read (especially the first one). Make clear the module as well. After doing once, the others don't need to be as explicit.
  \item [\done] find a seperate symbol for -> in the code sections in the paper for UN$\to$. Superscipt would be perfect. $UN^{\rightarrow}$
  \item [\done] Mention theorem 123 ii and iii are inherently classical. For iii we require strong decidability.
  \item [\done] Working on CR $\to$ CP. Low priority. \apcomment{This is mentioned as one of the research items above}
\end{todolist}

\section*{Week starting 2nd June}
\begin{todolist}
  \item [\done] Should we add quantification to each definition(I think quantification for $a$ is already implied)? E.g.\\
  \item [\ule{$a \in \WCR_R$}] \emph{$a$ is weakly Church-Rosser (or weakly confluent)} if $c \bstep a \rstep b \implies c \mstep d \bmstep b$.\\
   Transforms to
  \item[\ule{$a \in \WCR_R$}] \emph{$a$ is weakly Church-Rosser (or weakly confluent)} if $c \bstep a \rstep b \implies c \mstep d \bmstep b$ for all $b, c, d \in A$.
  \item [\done] For showing inc -> RP we're currently just saying "This is immediate". Is this sufficient? We did have a proof previously but I think we removed it for space reasons.
  \item [\done] Remove "desired implications" and similar organizational issues in the code (general review)
  \item [\done] Consider defining FB in terms of existing list library; especially in the rebooted repo,
  we shouldn't use manual List type just for this.
  \item [\done] Move all definitions and properties from Section 2.3 to Seq.agda, except Inc.
  \item [\done] Revise section 3.1
\end{todolist}
\section*{Week starting 9th June (some of these items taken from Dr. Polonsky's email sent June 6th)}
\begin{todolist}
  \item [\done] Thm. 19(iv), i don't like that the parens are italicized.  maybe it would be okay if instead of CR => CP we just write "the forward implication"
  \item [\done] Proof of (iii), line 296: do we really need () around SN?
  \item [\done] Page 9, second half (the hierarchy): we should add a proof that $\MF \cap \SN \iff \NF$, say this in the text, and note that this does NOT hold for $\SM$ and $\WN$.  Hence, the tables include a separate column for their combination.
  \item [\done] In the par before 4.1 counterexamples, add that in the global table, the counterexample 3 satisfies the hypotheses globally.
(I actually wonder whether we need example 9 at all, if we have a counterexample to the global one.  Am I missing something?). \sacomment{I don't think we need to expand that CE:3 satisfies the global hypotheses. I'm hoping that we've done enough to show to the reader how to read the table at this point. If you disagree I'm happy to add it in though. As to why have local and global counterexamples, good point. Unless either of us can think of a reason we'll remove in the next meeting.}
    \item [\done] the * and dagger in the table should be explained in the caption, or in the text.
    \sacomment{I prefer to have this as a footnote. I have put it as a caption for now but I would revert it to what we had before.  }
\end{todolist}

\subsection*{Notes from meeting on Friday June 13th}

\begin{itemize}
  \item Perhaps we should not create variations and multiple implications of WFind,
  but rather prove that it's equivalent to WFacc, but lands in a higher universe.
  Mention this once at the beginning of section 5 then proceed to analyze
  connections between WFacc and WFseq/WFmin.
\end{itemize}

\section*{July TODO}
\begin{todolist}
  \item Carry on refactoring Wellfounded.agda once fixed, breaking it up if necessary.
  \item To finish the paper: We should do a final walk through and make sure we haven't introduced errors with our edits. In particular, do all the counterexamples line up. Do all the references to different sections and pieces of code in Agda line up. 
  \item Do we remove all our local counterexamples in favour of global ones?
  \item Rework well-foundedness section 
  \item Rework conclusion 
  \item Rework introduction 
  \item Do we introduce a new section for discussing Agda? Talking about why we think there is an advantage to doing this work in Agda vs (say) Isabelle 

\end{todolist}

\section*{Research items}
\begin{todolist}
  \item (Research item) Go deeper in the related works and see which are most relevant. Make a list of works and give summaries to be used in future papers. * Look into the literature and related works. There should be an Agda formulation of rewrite systems (check agda standard library stuff on relations). 
  \item (Research item) Find a concrete application of generalized Newman's Lemma. * Finding a term rewrite system where it's eassier to see that it is weakly cr and that every term is SM (as opposed to being CR).  
  \item (Research item) Formalize a first order term rewriting system (I can't remember what the motivation for this was). Use TeReSe to create a list of rewrite systems.
  \item (Research item) Carry out research on rewriting counterexamples like our own so we can try and put them into some sort of context. Have our counerexamples been shown before?
  \item (Research item) Work on the missing implicaitons from the paper. So far we believe this is just CR $\to$ CP in the ARS section. - Will probably be quite a bit of work (low priority)
  \item (Research item) Do we want to reincorporate Knaster Tarski? - Not this paper 
  \item (Research item) How to go from WFseq to WFacc - possible that the Kleene Tree is a solution. - Not this paper
  
\end{todolist}

\section*{Agda standard library}
Some of the work we have explored is formalized in the Agda standard library.  
% https://agda.github.io/agda-stdlib/v1.7.3/Relation.Binary.Rewriting.html#2096
\begin{todolist}
  \item Write about the standard library formalization in our paper and talk about how ours differs. 
  \item Create an equivalences file showing that our formalization agrees with the ASL (specifically definitions such as confluet, wcr, nf, leading up to showing our formalizations of Newman's lemma agree).
  \item Investigate contributing work to the ASL. This may mean refactoring the work already done to conform with the ASL. Specific contributions would likely be: Generalized NL. Theorem's 122 and 123. Alternative definitions of Wellfounded, the implications for the hierarchy.
\end{todolist}

\section*{20th August todos}
\begin{todolist}
  \item [\done] (Secondary) Find consistency with NP and UN with respect to reduction and conversion. Should the default be by reduction for both. Should we be explicit with $\to$ and = in each case.  \sacomment{Main issue here is that we'll either introduce an inconsistency with our code or we'll have a hard to read symbol. We can do $\to$ as super or sub-script in Agda. If we're happy with $NP\to$ in agda but using a different script in the paper then I think we should go ahead and be explicit for both UN and NP as to whether it is WRT conversion or reduction. We already have this inconsistency in the code with $UN\to$} \apcomment{In the agda code let's not use arrow and have arrow implied as default but be explicit with '=' sign.}
  \item Carry on modifying the names of wfXXX- to wfXXXnotnot with prefix underscore in agda and in latex. \sacomment{Done with the exception of WFseq-. This needs a bit more thought. Similarly for the relation between WFcor and WFminCor.}
  \item [\done] Make a decision regarding WFcor. We use it to imply WFacc using AccCor. But otherwise it is sort of orphaned off. Does it help contribute to the overall story? \sacomment{We've now related it to WFmin by requiring that correductive predicates have double negation elimination. Is this too strong a requirement? Similarly we have linked WFacc to WFcor but this proof requires the same property as well as limiting ourselves to accessible predicates. Again, is this too much?}
  \item [\done] Get the classical hypothesis and relations organized and see how they're used in the implications (organize implications by classical requirement. In the implications file we can import the classical implications. Basically, refactor the code.) \sacomment{Done but could be further refactored.}
  \item Make a cheat sheet for the next meeting regarding to talk about coreductive. How it relates on the weak side and on the strong side. SEQ seq+- , how to present these. Try and clarify what the issues are here and what exactly are the differences between them in terms of definition and implication. Could one be removed and the overall story stay the same? \sacomment{Regarding seq+ (and seq+-) we currently only show that seq+ implies seq and seq+- implies seq- so could be removed without changing any of our other implictions.} 
\end{todolist}

\section*{September 15th todo}
\begin{todolist}
  \item (Minor) We want to have a sentence in our WF section making clear that we can consider local properties as well and maybe make an example from accessible to sequential. 
  \item Sentence explaining that wfacc is equivalent to wfind and therefore isn't included in the diagram (wfind).
\end{todolist}

\section*{September 26th TODO}
\begin{todolist}
  \item Focus for next meeting on writing. 
  \item [\done] Are not not closed predicates over natural numbers dedcidable: If not is this a counterexample for WFminDNE.
\end{todolist}

\end{document}
