\documentclass{scrartcl}

\usepackage{hyperref}
% \usepackage{tikz}
\usepackage{amsmath}
\usepackage{amsthm}
\usepackage{pifont}
\usepackage{mathrsfs}
\usepackage{amssymb}
\usepackage{xcolor}
\usepackage{colortbl}
\usepackage{tikz-cd}
\usepackage{caption}
\usepackage{newunicodechar}


\usepackage[utf8]{inputenc}
\usepackage{ucs}
% \DeclareUnicodeCharacter{03A3}{\ensuremath{\Sigma}}



\usetikzlibrary{positioning,shapes.geometric,fit,arrows.meta}


\captionsetup{justification=centering}



\definecolor{darkgreen}{rgb}{0.0, 0.5, 0.0}

\newcommand{\tto}{\twoheadrightarrow}
\newcommand{\sse}{\subseteq}
\newcommand{\bset}{\mathbf{Set}}
\newcommand{\nat}{\mathbb{N}}

% Comments
\newcommand{\sacomment}[1]{\textcolor{green}{#1}}
\newcommand{\apcomment}[1]{\textcolor{blue}{#1}}
\newcommand{\greyout}[1]{\textcolor{gray}{#1}}
\newcommand{\err}[1]{\textcolor{red}{#1}}

% Theorem style
\newtheorem{notation}[theorem]{Notation}
% \newtheorem{thm}{Theorem}
% \newtheorem{dfn}[thm]{Definition}
% \newtheorem{prop}[thm]{Proposition}
% \newtheorem{cor}[thm]{Corollary}
% % \newtheorem{lemma}[thm]{Lemma}
% % \newtheorem{rmk}[thm]{Remark}
% % \newtheorem{expl}[thm]{Example}
% \newtheorem{notn}[thm]{Notation}
% %\theoremstyle{nonumberplain}
% %\theoremsymbol{\Box}
% % \newtheorem{proof}{Proof}

\newcommand{\RP}{\mathrm{RP}}
\newcommand{\gRP}{\mathbf{RP}}
\newcommand{\RPm}{\mathrm{RP{-}}}
\newcommand{\gRPm}{\mathbf{RP{-}}}
\newcommand{\NF}{\mathrm{NF}}
\newcommand{\MF}{\mathrm{MF}}
\newcommand{\UN}{\mathrm{UN}}
\newcommand{\gUN}{\mathbf{UN}}
\newcommand{\UNto}{\mathrm{UN}^{\to}}
\newcommand{\gUNto}{\mathbf{UN}^{\to}}
\newcommand{\SN}{\mathrm{SN}}
\newcommand{\gSN}{\mathbf{SN}}
\newcommand{\decSN}{\mathrm{dec(SN)}}
\newcommand{\SM}{\mathrm{SM}}
\newcommand{\SMseq}{\mathrm{SMseq}}
\newcommand{\gSM}{\mathbf{SM}}
\newcommand{\WN}{\mathrm{WN}}
\newcommand{\gWN}{\mathbf{WN}}
\newcommand{\SMandWN}{\mathrm{SM\land WN}}
\newcommand{\gSMandWN}{\mathbf{SM\land WN}}
\newcommand{\WM}{\mathrm{WM}}
\newcommand{\gWM}{\mathbf{WM}}
\newcommand{\WNFP}{\mathrm{WNFP}}
\newcommand{\NP}{\mathrm{NP}}
\newcommand{\gNP}{\mathbf{NP}}
\newcommand{\NPe}{\mathrm{NP_=}}
\newcommand{\gNPe}{\mathbf{NP_=}}
\newcommand{\WMFP}{\mathrm{WMFP}}
\newcommand{\MP}{\mathrm{MP}}
\newcommand{\gMP}{\mathbf{MP}}
\newcommand{\CR}{\mathrm{CR}}
\newcommand{\CRs}{\mathrm{CR^{\le 1}}}
\newcommand{\gCRs}{\mathbf{CR^{\le 1}}}
\newcommand{\gCR}{\mathbf{CR}}
\newcommand{\WCR}{\mathrm{WCR}}
\newcommand{\gWCR}{\mathbf{WCR}}
\newcommand{\Inc}{\mathrm{Inc}}
\newcommand{\gInc}{\mathbf{Inc}}
\newcommand{\BP}{\mathrm{BP}}
\newcommand{\gBP}{\mathbf{BP}}
\newcommand{\FB}{\mathrm{FB}}
\newcommand{\CP}{\mathrm{CP}}
\newcommand{\gCP}{\mathbf{CP}}



\newcommand{\from}{\leftarrow}



\newcommand{\red}[1]{\textcolor{red}{#1}}
\newcommand{\blue}[1]{\textcolor{blue}{#1}}

% Reduction relation macros
\newcommand{\rstep}{\mathbin{\longrightarrow_R}}
\newcommand{\mstep}{\mathbin{\longrightarrow_R^*}}
\newcommand{\estep}{\mathbin{\longrightarrow_R^=}}
\newcommand{\rrstep}{\mathbin{\longrightarrow_R^r}}
\newcommand{\brstep}{\mathbin{\longleftarrow_R^r}}
\newcommand{\bstep}{\mathbin{\longleftarrow_R}}
\newcommand{\bmstep}{\mathbin{\longleftarrow_R^*}}

% Unicode characters

\newunicodechar{∀}{\ensuremath{\forall}}
\newunicodechar{→}{\ensuremath{\rightarrow}}
\newunicodechar{ℕ}{\ensuremath{\mathbb{N}}}
\newunicodechar{↔}{\ensuremath{\leftrightarrow}}
\newunicodechar{⊆}{\ensuremath{\sse}}
\newunicodechar{∧}{\ensuremath{\land}}
\newunicodechar{₌}{\ensuremath{_=}}
\newunicodechar{𝓟}{\ensuremath{\mathcal{P}}}
\newunicodechar{∈}{\ensuremath{\in}}
\newunicodechar{φ}{\ensuremath{\phi}}
\newunicodechar{Σ}{\ensuremath{\Sigma}}
\newunicodechar{₁}{\ensuremath{{}_1}}


% Misc Macros
\newcommand{\terese}{[TeReSe]}
% \newcommand{\ul}[1]{\underline{#1}}
\newcommand{\ule}[1]{\underline{#1:}}
% \newcommand{\setof}[1]{\{#1\}}
\newcommand{\setof}[1]{\left\{#1\right\}}


\newcommand{\tclos}[1]{{#1}^{\scriptscriptstyle{+}}}
   


\begin{document}

\textbf{Primary Goal: A technical report that focuses on
Abstract Rewriting in Constructive Setting }

\section{2024.10.31}

Rewriting properties.
Elementwise
\begin{itemize}
  \item NF
  \item WN
  \item SN
  \item WCR
  \item CR
  \item NFP
  \item UN$\to$
  \item UN
  \item RP
  \item RP$-$
  \item $\omega$-bound
  \item Recurrent
  \item CP (Cofinality property)
  \item FB (Finite branching of rewrite steps)
  \item DominatedByWF (Needed?)
  \item Define WNFP to be:
  \[\forall a b c \to b \in NF \to a R^* b \to a R^* c \to c R^* b \]
  \item Does WNFP and WN imply CR?  Looks like it!
  \item Another interesting one? $WN^*$: hereditarily $WN$ ($WN$ for the reduction graph starting from a given element $x$.)

  Could even consider $SN^*$, $CR^*$ etc.
\end{itemize}
Most of these properties $P$ have a ``global'' version $P^\forall$, asserting that $\forall x. P(x)$ holds.

Other properties that show up in classical ARS theory:
\begin{itemize}
  \item Inc
  \item Ind
  \item $\mathrm{CR}^{\le 1}$
\end{itemize}
Want to discuss these, and why we may or may not include them in our
development.

Outline of the document:
\begin{enumerate}
  \item Create a diagram with implications between these, and
  combinations of these, similar to the one in TeReSe.
  \item Formalization of valid implications
  \item Counterexamples: Compile a list/chart/graph of those relationships
  which can be disproved. (A big part!!)
  \item A list of classical properties, and elaboration of their meaning
  and whether it's reasonable to assume them in a rewriting context.
  (Discuss each, and its role in ARS theory.)
  \item Well-foundedness: different constructive formulations, and relationships between them
  \item ``Guide'' to the formalization code, GitHub repo
  \item Discussion of open problems (Add a subsection of these at the end of each section?)
  \item Conclusion
\end{enumerate}

\section{Established implications:}

\begin{itemize}
  \item SN implies $\omega$-bounded
\end{itemize}

\section{2024.10.10}

\begin{itemize}
  \item Look at restricting WFmin to $\lnot\lnot$-closed predicates
  \item Status of important hypotheses:
    \begin{itemize}
      \item Deciding whether a given element $x$ reduces to a given element $y$
      (Is $R$ itself decidable?)
      \item Deciding whether a given element $x$ reduces to some element $y$
      (Is being an $R$-normal form decidable?)

      {\textcolor{red}{This is needed to show that SN $\Rightarrow$ WN.}}

      \item If a given element $x$ is \emph{not} a normal form,
      exhibiting some element $y$ that it reduces to.
      (This is related to DeMorgan/Markov/etc. properties that came up in WF file.)

      \item Does UN-lemma \emph{require} the decidability assumption?
    \end{itemize}
  \item META questions: Which of the ARS theorems can be made ``local/pointwise'',
  for example,
  \begin{itemize}
    \item NL+: $WCR(R) \land SN(x) \to CR(x)$?  (Yes?)
    \item NL++: $WCR(x) \land SN(x) \to CR(x)$? (No.)
  \end{itemize}
  \item When time permits, uniformize the notation (variable names, etc.) to the extent possible.
\end{itemize}

\section{2024.09.19}

TODO item: Compile a list of counterexamples

\section{2024.09.12}
\begin{itemize}
  \item Investigate \texttt{iii-lemma}.  (Is it false, or is there a classical proof?)
  (No constructive proof?)
  \item Look at variations of RP.
  \item What happens if properties involving sequences are reformulated for
  $R^r$? (Cannot decide whether the given sequence is finite or infinite.)
  \item Look at RP- and WCR implies RP!
  \item Can "sequential" properties like RP, RP-, isWFseq, isWFseq-, $\omega$-bounded, etc.,
  be related to the cofinality property.
\end{itemize}

\section{Relations, ARS}


\begin{enumerate}
  \item Define logical and structurual operations on relations, proving laws about them.
  \item Define closure operations on realtions, proving laws about them
  \item Define properties such as transtivity etc.
\end{enumerate}

\subsection{Well-foundedness}

Four notions:
\begin{itemize}
  \item Based on every element being accessible
  \item Every inductive relation is universally true
  \item Every non-empty set has a minimal element
  \item No infinite sequence is decreasing
\end{itemize}



\section{Open research questions}
\begin{enumerate}
  \item Sufficient conditions for reversing the implications between notions of
  well-foundedness?
  \item Knaster-Tarski Lemma: Most general form? Alternative formulations?
  \item Syntax for closure operators allowing to prove the needed properties
  uniformly/generically
\end{enumerate}

\section{To Do}
\begin{enumerate}
  \item Clean up the current development
  \item More laws about closure operations: Monotonicity, idempotency
  \item Refactor the ARS formalized results with the new development
  \item Knaster--Tarski in a well-founded setting: what additional
  hypotheses are necessary?
\end{enumerate}

\subsection{Questions about well-foundedness}

\begin{enumerate}
  \item For a given element $x$, is the property of
  being accessible $\lnot\lnot$-closed? I.E.,
  Does $x$ being $\lnot\lnot$-accessible imply that $x$ is accessible?
  (Conjecture: No.)

  \item As a special case, does being weakly (accessible) well-founded imply being $\lnot\lnot$-wellfounded?
  (Conjecture: No.)

  Also: Same question about inductive notion of well-foundedness.

  \item ($\star$) Given a well-founded relation, does every non-empty subset
  have a minimal element?

  \emph{Problem.} Need to decide whether, for a given $x \in U$,
   the set $\{y | Ryx\}$ is empty.

   \item IF every non-empty subset has a minimal element, does this imply
   either of the weak forms of well-foundedness: $\mathtt{isWFacc-}$ or
   $\mathtt{isWFind-}$ ?

   \item Does $\mathtt{isWFmin-}$ imply $\mathtt{isWFacc-}$ or $\mathtt{isWFind-}$?

   \emph{Problem.} Need to go from $\lnot (\forall y. R y d \to \phi y)$
   and $\forall y. R y d \to \lnot \lnot \phi y$ to $\bot$.

   In terms of accessibility, it should suffices to assume accessibility is
   $\lnot\lnot$-closed, since being related to y is not relevant for that. (?)

   \item Does sequential well-foundedness (no decreasing sequence) imply
   any of the other notions, e.g., \texttt{WFmin-} ?

   \emph{Note.}  This seems to require the most classical assumptions:
   $\lnot\lnot$-closure of $\phi$, relativized De Morgan law,
   Markov's principle, etc.

   \item (2024.10.10) Is being $R$-WN inductive?
   \item (2024.10.10) Does $R$ being finitely branching imply
   that, for an inductive $\phi$, if $\lnot \phi(x)$ then
   $\exists y. Rxy \land \lnot \phi(y)$?

\end{enumerate}

\textsc{Remark.}
$\lnot\lnot$-closure of accessibility should solve problems 1 and 4.

\section{Open problems}

\begin{itemize}
  \item The infinite haircomb refutes $WN \land RP \to \omega$-bounded.  But it is not WCR.

  So, does $WN, RP, WCR$ imply $\omega$-bounded?
  Answer: yes, that is what we have proved classically.

  \item Wait, does RP- actually imply RP????

  \item RP- and $\omega$-bounded implies RP?
  \item RP- and WCR imply RP?
  \item
  \[ WNg \land UN \to CRelem :
  \forall (R : \mathscr{R} A) \to WN R \to \forall x \to is R -UN x \to is R -CR x \]
  \item $WN R \to UN R \to \omega-bdd R \to SN R$ (1.2.3.ii-)
\end{itemize}

\end{document}
